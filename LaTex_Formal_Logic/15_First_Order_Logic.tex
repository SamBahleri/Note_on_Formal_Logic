First Order Logic

Basically, reasoning rules such as Double Negation (\(DN\)), Modus
Ponens (\(MP\)), and Reductio ad Absurdum (\(RAA\)) are the most basic
forms of reasoning. However, as we have learned, these three rules more
often only work on logical patterns between statements. In other words,
although deductively valid, their application is not yet complete enough
to capture deeper mathematical structures.

As an example, consider the following statements:

− \(p\): 2 is prime

− \(q\): 2 \(\in \mathbb{N}\)

If we convert this into symbolic form, we get:

\[p \land q\]

However, it should be noted that \(p\) and \(q\) here are merely
representations of complete sentences in natural language. This
symbolization has not yet touched the predicate level inherent in the
number 2, namely the property of being prime and its membership in the
set \(\mathbb{N}\). In other words, propositional logic only ``packages
statements'', but has not yet ``dissected'' the mathematical content of
the statement itself. Therefore, a richer framework is needed to capture
such expressions. This is where First-Order Logic (FOL) becomes crucial,
as it allows us to express properties, relations, and quantification
over mathematical objects more precisely.

\subsection{1. Sentence of FOL}\label{sentence-of-fol}

Normally, there are six kinds of symbols in FOL, and most of them are
similar to those in propositional logic:

\begin{enumerate}
\def\labelenumi{\arabic{enumi}.}
\item
  Constant symbols
  \[a, b, c, \dots, r, a_1, b_{224}, h_7, m_{32}, \dots\]
\item
  Variable symbols \[s, t, u, v, w, x, y, z, x_1, y_1, z_1, x_2, \dots\]
\item
  Function symbols \[f, g, h, f_1, g_2, h_{37}, \dots\]
\item
  Predicate symbols
  \[A, B, C, \dots, Z, A_1, B_1, Z_1, A_2, A_{25}, J_{375}, \dots, =\]
\item
  Logical connectives \[\lnot, \land, \lor, \to, \leftrightarrow\]
\item
  Brackets \[( \ , \ )\]
\item
  Quantifiers \[\forall, \exists\]
\end{enumerate}

\subsection{2. Terms and formulas}\label{terms-and-formulas}

\begin{enumerate}
\def\labelenumi{\arabic{enumi}.}
\tightlist
\item
  Every atomic formula is a formula.
\item
  If \(a\) and \(b\) are formulas, then so are:

  \begin{enumerate}
  \def\labelenumii{\arabic{enumii}.}
  \tightlist
  \item
    \(\lnot a\)
  \item
    \((a \land b)\)
  \item
    \((a \lor b)\)
  \item
    \((a \to b)\)
  \item
    \((a \leftrightarrow b)\)
  \end{enumerate}
\item
  If \(a\) is a formula and \(x\) is a variable, then:

  \begin{enumerate}
  \def\labelenumii{\arabic{enumii}.}
  \tightlist
  \item
    \(\forall x \, a\)
  \item
    \(\exists x \, a\)
  \end{enumerate}
\end{enumerate}

\subsection{3. Bracketing Conventions in
FOL}\label{bracketing-conventions-in-fol}

\begin{enumerate}
\def\labelenumi{\arabic{enumi}.}
\tightlist
\item
  Outer parentheses may be omitted.
\end{enumerate}

\[
  a \land b \quad \text{instead of} \quad (a \land b)
  \]

\begin{enumerate}
\def\labelenumi{\arabic{enumi}.}
\setcounter{enumi}{1}
\tightlist
\item
  Negation binds most tightly.
\end{enumerate}

\[
  \lnot a \lor b \quad \text{means} \quad (\lnot a) \lor b
  \]

\begin{enumerate}
\def\labelenumi{\arabic{enumi}.}
\setcounter{enumi}{2}
\tightlist
\item
  Quantifiers extend as far to the right as possible.
\end{enumerate}

\[
  \forall x \, a \lor b \quad \text{means} \quad (\forall x \, a) \lor b
  \]

\begin{enumerate}
\def\labelenumi{\arabic{enumi}.}
\setcounter{enumi}{3}
\tightlist
\item
  Other connectives associate to the right unless parentheses indicate
  otherwise.
\end{enumerate}

\[
  a \lor b \lor c \quad \text{means} \quad a \lor (b \lor c)
  \]

\subsection{4. Superscripts on Predicates in
FOL}\label{superscripts-on-predicates-in-fol}

\begin{enumerate}
\def\labelenumi{\arabic{enumi}.}
\tightlist
\item
  Predicate symbols may be written with superscripts to indicate their
  arity.
\end{enumerate}

\[
  P^1 \text{ (unary)}, \quad Q^2 \text{ (binary)}, \quad R^3 \text{ (ternary)}
  \]

\begin{enumerate}
\def\labelenumi{\arabic{enumi}.}
\setcounter{enumi}{1}
\tightlist
\item
  In general, if \(P^n\) is an \(n\)-place predicate and
  \(t_1, \dots, t_n\) are terms, then
\end{enumerate}

\[P^n(t_1, \dots, t_n) \text{ is an atomic formula.}\]

\subsection{5. Name}\label{name}

Names, in general, function to indicate both the existence of an entity
and serve as a reference for a particular object or place. For example,
the word ``Gottlob Frege'' refers specifically to an individual, a great
logician who was instrumental in the development of predicate logic.
However, it must be recognized that if someone has never heard the name
``Gottlob Frege'' before, they may not know who is being referred to. In
fact, that person might even imagine that ``Gottlob Frege'' is not the
name of a human being, but merely a term or particular object. Such
conditions are known in philosophy of language as a problem inherent in
proper names, namely the potential ambiguity in reference.

Within the framework of FOL, such ambiguity cannot be tolerated. The
reason is simple, when we engage in formal reasoning, we require that
the expressions used be clear, syntactically valid, and also sound in
meaning. Therefore, naming in FOL must have definite reference and leave
no room for double interpretation. In FOL, constant symbols usually
function like proper names, pointing specifically to an object within
the domain. For example:

\begin{enumerate}
\def\labelenumi{\arabic{enumi}.}
\item
  Constant \(a\) refers to object \(2\) in \(\mathbb{N}\).
\item
  Constant \(b\) refers to object \(-5\) in \(\mathbb{Z}\).
\end{enumerate}

So when we write the predicate \(Prime(a)\), it must be understood
definitively as ``2 is a prime number.'' There can be no ambiguity about
whether \(a\) refers to the number 2, to a person named ``Gottlob
Frege'' or even to a chair, because in FOL, the interpretation of every
symbol must be clear within the established domain.

\subsection{6. Predicate}\label{predicate}

In everyday language, a predicate is the part of a statement that
provides information or describes the subject. For example, in the
sentence \emph{Snow is white} the phrase ``is white'' is the predicate.

Consider the following example:

\$ \text{Premise 1: All humans are mortal.}\$

\$ \text{Premise 2: Socrates is a human.}\$

\$ \text{Conclusion: Therefore, Socrates is mortal}\$

As we can see, expressions such as \emph{``are mortal''} and \emph{``is
a human''} are predicates. Through such structures, predicate logic
enables us to express relationships and properties of objects in greater
depth than standard propositional logic. Moreover, it is important to
understand that a predicate like \emph{``mortal''} is not a standalone
proposition but a component of one. A predicate can be thought of as a
framework or function with an empty slot waiting to be filled by a
subject. For example:

−\(Mortal(\dots)\)

This structure may feel counterintuitive, since we are accustomed to
complete sentences in natural language. However, there are historical
and logical reasons why this functional form is preferred in formal
logic.

Following the example above, if we fill the blank with \emph{Socrates},
we obtain:

−\(Mortal(Socrates)\)

This indicates that \emph{``Socrates''} is placed in the subject
position of the predicate \emph{``mortal''}, forming a complete
proposition. For simplicity and efficiency, logicians often replace
predicate names with a single capital letter. Thus, \(Mortal(Socrates)\)
is simplified to:

−\(M(Socrates)\) , where \(M\) represents the predicate
\emph{``mortal.''}

We can simplify even further by representing the name
\emph{``Socrates''} with the constant \(S\). Therefore, the whole
expression becomes:

−\(M(S)\)

Furthermore, to express \emph{``Socrates is human and mortal''}, we can
combine the two predicates into a single expression:

− \(H(S) \land M(S)\)

Thus, the expression \(H(S) \land M(S)\) states that Socrates satisfies
both predicates: he is a human and he is mortal. Accordingly, we can
also represent general statements using formulas with \emph{free
variables}. The idea is that the variable serves as a placeholder, or a
``blank,'' that can later be filled by any element from the domain of
discourse. For example:

− \(P(x)\)

− \(O(x)\)

\subsection{7. Quantifiers}\label{quantifiers}

Shortly speaking, when discussing predicate logic, one immediately
thinks of Gottlob Frege (1848--1925). Frege introduced the concept of
quantifiers, namely \(\forall\) and \(\exists\). These concepts were
later popularized in the early 20th century by Bertrand Russell and
Alfred North Whitehead through their seminal work \emph{Principia
Mathematica}.

The use of these symbols is essential to represent not only a single
variable but also many variables. For example:

− Everyone is happy

In this case, we cannot represent the statement merely as \(H(e)\), as
we did in the Socrates'' example. Instead, we represent it as:

\[\forall x \, H(x)\]

In this case, we can represent anything with the predicate \emph{Happy}.
The variable acts as a placeholder, so whatever we choose from the
domain can be tested against the predicate. This shows that the
predicate \emph{Happy} is not tied to a single individual but can be
applied universally to any object in the chosen domain of discourse.

Similarly, consider the example:

\[\text{Someone is unique}\]

We can represent this statement using the existential quantifier as
follows:

\[\exists x \, U(x)\]

Here: − \(\exists x\) intuitively means \emph{``there exists at least
one x''} in the domain.

− \(U(x)\) states that this \(x\) has the property of being unique.

Thus, the formula \(\exists x \, U(x)\) expresses that there is at least
one object in the domain such that the predicate \emph{Unique} applies
to it.

\[\text{Everyone loves someone}\]

\[\forall x \exists y \, L(x,y)\]

This formula uses two variables: \(x\) represents the lover and \(y\)
represents the beloved. The formula states that for every person \(x\)
in the domain, there exists some person \(y\) such that \(x\) loves
\(y\).

\[\text{Someone is loved by everyone}\]

\[\exists y \forall x \, L(x,y)\]

This formula states that there exists a person \(y\) such that for all
persons \(x\), \(x\) loves \(y\), meaning there is one person who is
universally loved.

\[\text{If someone is a teacher, then everyone respects them}\]

\[\exists x (T(x) \land \forall y \, R(y,x))\]

This formula combines both quantifiers with a conditional structure. It
states that there exists a person \(x\) who is a teacher, and for all
persons \(y\), \(y\) respects \(x\).

\[\text{Everyone who teaches logic is respected by all students}\]

\[\forall x ((T(x) \land L(x)) \rightarrow \forall y (S(y) \rightarrow R(y,x)))\]

This more complex formula involves multiple predicates, \(T(x)\) means
``\(x\) teaches'', \(L(x)\) means ``\(x\) teaches logic'', \(S(y)\)
means ``\(y\) is a student'', and \(R(y,x)\) means ``\(y\) respects
\(x\)''. The formula demonstrates how multiple variables can interact
within nested quantifier scopes.

\subsection{8. Domains}\label{domains}

It is also important to comprehend the domain of our reasoning in
predicate logic. The critical point arises when we begin to express more
complex statements, we must be precise about how variables are used
within predicates.

Consider the formula:

\[\forall x \, H(x)\]

which we might translate as ``Everyone is happy.'' But who exactly
counts as ``everyone''? In everyday English, we do not usually mean
literally every person who has ever lived or ever will live. We
typically mean something more limited: everyone in this building,
everyone in a class, everyone on a team, and so on. Predicate logic
removes this ambiguity by requiring us to specify a domain of discourse.
The domain is the set of objects under discussion.

For example, suppose we set our symbolization key as follows:

− Domain: people in Finland

− \(H(x)\): \(x\) is happy

Now the quantifiers range only over that domain.

− \(\forall x \, H(x)\) means ``Every person in Finland is happy.''

− \(\exists x \, H(x)\) means ``Some person in Finland is happy.''

In FOL, the domain must contain at least one member, and each name must
refer to exactly one object in that domain. So if \(S\) names Sam, then
from \(H(S)\) (``Sam is happy'') we can infer \(\exists x \, H(x)\)
(``Someone is happy'').

The choice of domain is crucial. Suppose we instead use this
symbolization key:

− Domain: the Eiffel Tower

− \(P(x)\): \(x\) is in Paris

Then \(\forall x \, P(x)\) would translate as ``Everything is in
Paris.'' But since the domain contains only the Eiffel Tower, what the
sentence really says is simply: ``The Eiffel Tower is in Paris.''

Another example, suppose we want to say:

\[\text{There is exactly one king in the palace.}\]

If we only write:

\[\exists x \, King(x)\]

this means \emph{``there exists at least one \(x\) such that \(x\) is a
king.''} But it does not prevent the possibility of there being two or
more kings. To capture the idea of uniqueness, we need to express that
there is \emph{one and only one}. This is often written with the
uniqueness quantifier \(\exists!\)

\[\exists! x \, King(x)\]

Alternatively, we can express uniqueness using standard quantifiers:

\[\exists x (King(x) \land \forall y (King(y) \rightarrow x = y))\]

This formula states: ``There exists an \(x\) such that \(x\) is a king,
and for all \(y\), if \(y\) is a king, then \(x\) equals \(y\)'',
ensuring there is exactly one king.

\subsection{9. Quantifiers and Scope}\label{quantifiers-and-scope}

When we introduce quantifiers into formal logic, understanding their
scope becomes crucial for accurate translation and interpretation. The
scope of a quantifier determines which variables it binds and how far
its influence extends in a formula.

\begin{enumerate}
\def\labelenumi{\arabic{enumi}.}
\item
  Example 1: \[\text{If anyone is a teacher, they are respected}\] \[
   \color{black}{\overbrace{\color{black}\forall x \, (T(x) \to R(x))}^{\color{black}\text{Scope of } \forall x}}
   \]

  The universal quantifier scopes over the conditional, creating a
  universal generalization. This statement holds vacuously true in
  domains without teachers. \emph{All teachers are respected} yields the
  same logical form.
\item
  Example 2:

  \[\text{Every professor knows a student who failed}\]

  \[
   {\color{black}\overbrace{\color{black}\forall y \, \Big( \text{Professor}(y) \rightarrow
   {\color{black}\overbrace{\color{black}\exists x \, (\text{Student}(x) \wedge \text{Failed}(x) \wedge \text{Knows}(y,x))}^{\color{black}\text{Scope of } \exists x}} \Big)}^{\color{black}\text{Scope of } \forall y}}
   \]
\item
  Example 3:

  \[\text{Every student has submitted some assignment to each professor}\]

  \[
   {\color{black}\overbrace{\color{black}\forall x \, \Big( \text{Student}(x) \rightarrow
   {\color{black}\overbrace{\color{black}\forall z \, (\text{Professor}(z) \rightarrow 
   {\color{black}\overbrace{\color{black}\exists y \, (\text{Assignment}(y) \wedge \text{Submitted}(x,y,z))}^{\color{black}\text{Scope of } \exists y}})}^{\color{black}\text{Scope of } \forall z}} \Big)}^{\color{black}\text{Scope of } \forall x}}
   \]
\end{enumerate}

\subsection{10. The Order of
Quantifiers}\label{the-order-of-quantifiers}

\begin{enumerate}
\def\labelenumi{\arabic{enumi}.}
\item
  Example 1:

  Consider the sentence \emph{Everyone admires some Logician}. This is
  ambiguous and can be represented in two different ways:

  Interpretation 1: For every person \(x\), there exists some Logician
  \(y\) such that \(x\) admires \(y\):

  \[
   {\color{black}\overbrace{\color{black}\forall x \, \Big(
   {\color{black}\overbrace{\color{black}\exists y \, (Person(x) \wedge Logician(y) \wedge Admires(x,y))}^{\color{black}\text{Scope of } \exists y}}
   \Big)}^{\color{black}\text{Scope of } \forall x}}
   \]

  \emph{Interpretation:} Everyone may admire different Logician.

  Interpretation 2: There exists some particular Logician \(y\) such
  that every person \(x\) admires \(y\):

  \[
   {\color{black}\overbrace{\color{black}\exists y \,
   \Big(
   \text{Logician}(y) \wedge
   {\color{black}\overbrace{\color{black}\forall x \,
   ( \text{Person}(x) \wedge \text{Admires}(x,y) )
   }^{\color{black}\text{Scope of } \forall x}}
   \Big)
   }^{\color{black}\text{Scope of } \exists y}}
   \]

  \emph{Interpretation:} There's one particular Logician who everyone
  admires.
\item
  Example 2:

  Consider the sentence \emph{Every student passed some exam}. This has
  two distinct interpretations:

  Interpretation 1: For each student, there exists at least one exam
  they passed \[
   {\color{black}\overbrace{\color{black}\forall x \, \Big(
   \text{Student}(x) \rightarrow
   {\color{black}\overbrace{\color{black}\exists y \, (\text{Exam}(y) \wedge \text{Passed}(x,y))}^{\color{black}\text{Scope of } \exists y}}
   \Big)}^{\color{black}\text{Scope of } \forall x}}
   \]

  \emph{Interpretation:} Each student passed at least one exam (possibly
  different exams).

  Interpretation 2: There exists one particular exam that every student
  passed \[
   {\color{black}\overbrace{\color{black}\exists y \,
   \Big(
   \text{Exam}(y) \wedge
   {\color{black}\overbrace{\color{black}\forall x \, 
   ( \text{Student}(x) \rightarrow \text{Passed}(x,y) )
   }^{\color{black}\text{Scope of } \forall x}}
   \Big)
   }^{\color{black}\text{Scope of } \exists y}}
   \]

  \emph{Interpretation:} There's one specific exam that all students
  passed.

  Consider the sentence \emph{Every patient has some symptom}:

  Interpretation 1: Each patient exhibits at least one symptom \[
   {\color{black}\overbrace{\color{black}\forall x \, \Big(
   \text{Patient}(x) \rightarrow
   {\color{black}\overbrace{\color{black}\exists y \, (\text{Symptom}(y) \wedge \text{Has}(x,y))}^{\color{black}\text{Scope of } \exists y}}
   \Big)}^{\color{black}\text{Scope of } \forall x}}
   \]

  \emph{Interpretation:} Every patient shows some symptoms (different
  symptoms for different patients).

  Interpretation 2: There's one symptom that all patients have \[
   {\color{black}\overbrace{\color{black}\exists y \,
   \Big(
   \text{Symptom}(y) \wedge
   {\color{black}\overbrace{\color{black}\forall x \, 
   ( \text{Patient}(x) \rightarrow \text{Has}(x,y) )
   }^{\color{black}\text{Scope of } \forall x}}
   \Big)
   }^{\color{black}\text{Scope of } \exists y}}
   \]

  \emph{Interpretation:} There's a common symptom shablack by all
  patients.
\item
  Example 3:

  Consider the sentence \emph{Every company hiblack some consultant}:

  Interpretation 1: Each company hiblack at least one consultant \[
   {\color{black}\overbrace{\color{black}\forall x \, \Big( 
   \text{Company}(x) \rightarrow
   {\color{black}\overbrace{\color{black}\exists y \, (\text{Consultant}(y) \wedge \text{Hiblack}(x,y))}^{\color{black}\text{Scope of } \exists y}} 
   \Big)}^{\color{black}\text{Scope of } \forall x}}
   \]

  \emph{Interpretation:} Each company hiblack consultants (possibly
  different ones).

  Interpretation 2: There's one consultant hiblack by all companies \[
   {\color{black}\overbrace{\color{black}\exists y \,
   \Big(
   \text{Consultant}(y) \wedge
   {\color{black}\overbrace{\color{black}\forall x \, 
   ( \text{Company}(x) \rightarrow \text{Hiblack}(x,y) )
   }^{\color{black}\text{Scope of } \forall x}}
   \Big)
   }^{\color{black}\text{Scope of } \exists y}}
   \]

  \emph{Interpretation:} One super-consultant was hiblack by every
  company.

  Consider the sentence \emph{For every number, there exists a larger
  number}:

  Interpretation 1: For each number, we can find a larger one \[
   {\color{black}\overbrace{\color{black}\forall x \,
   {\color{black}\overbrace{\color{black}\exists y \, (x < y)}^{\color{black}\text{Scope of } \exists y}} 
   }^{\color{black}\text{Scope of } \forall x}}
   \]

  \emph{Interpretation:} There's no largest number (true statement).

  Interpretation 2: There exists one number larger than all numbers \[
   {\color{black}\overbrace{\color{black}\exists y \,
   {\color{black}\overbrace{\color{black}\forall x \, (x < y)}^{\color{black}\text{Scope of } \forall x}}
   }^{\color{black}\text{Scope of } \exists y}}
   \]

  \emph{Interpretation:} There's a supremely large number (false
  statement).
\item
  Example 4:

  Consider the sentence \emph{Everyone trusts someone}:

  Interpretation 1: Each person trusts at least one person \[
   {\color{black}\overbrace{\color{black}\forall x \, \Big(
   \text{Person}(x) \rightarrow
   {\color{black}\overbrace{\color{black}\exists y \, (\text{Person}(y) \wedge \text{Trusts}(x,y))}^{\color{black}\text{Scope of } \exists y}}
   \Big)}^{\color{black}\text{Scope of } \forall x}}
   \]

  \emph{Interpretation:} Nobody is completely distrustful; everyone
  trusts someone.

  Interpretation 2: There's one person trusted by everyone \[
   {\color{black}\overbrace{\color{black}\exists y \,
   \Big(
   \text{Person}(y) \wedge
   {\color{black}\overbrace{\color{black}\forall x \, 
   ( \text{Person}(x) \rightarrow \text{Trusts}(x,y) )
   }^{\color{black}\text{Scope of } \forall x}}
   \Big)
   }^{\color{black}\text{Scope of } \exists y}}
   \]

  \emph{Interpretation:} There's a universally trusted person.
\item
  Example 5:

  Consider the sentence \emph{Every device connects to some network}:

  Interpretation 1: Each device connects to at least one network \[
   {\color{black}\overbrace{\color{black}\forall x \, \Big(
   \text{Device}(x) \rightarrow
   {\color{black}\overbrace{\color{black}\exists y \, (\text{Network}(y) \wedge \text{Connects}(x,y))}^{\color{black}\text{Scope of } \exists y}}
   \Big)}^{\color{black}\text{Scope of } \forall x}}
   \]

  \emph{Interpretation:} All devices have network connectivity (possibly
  to different networks).

  Interpretation 2: There's one network that all devices connect to \[
   {\color{black}\overbrace{\color{black}\exists y \,
   \Big(
   \text{Network}(y) \wedge
   {\color{black}\overbrace{\color{black}\forall x \,
   ( \text{Device}(x) \rightarrow \text{Connects}(x,y) )
   }^{\color{black}\text{Scope of } \forall x}}
   \Big)
   }^{\color{black}\text{Scope of } \exists y}}
   \]

  \emph{Interpretation:} There's a universal network that connects all
  devices.
\end{enumerate}

These examples demonstrate how the \emph{quantifier shift fallacy} can
dramatically alter meaning. The pattern \(\forall x \exists y\)
(distributive) versus \(\exists y \forall x\) (collective) represents
fundamentally different logical relationships, making quantifier order
crucial for precise logical reasoning.

\subsection{11. Free and Bound
Variables}\label{free-and-bound-variables}

In the rules of predicate logic, we must also understand the concepts of
free variables and bound variables. In short, a free variable is an
object being referred to but is not within the scope of a universal or
existential quantifier. Whereas a bound variable is an object that is
expressed within the scope of a quantifier, either universal or
existential. Understanding this distinction is essential for correctly
interpreting logical formulas and determining whether a formula is
closed (has no free variables) or open (contains free variables).

\begin{enumerate}
\def\labelenumi{\arabic{enumi}.}
\item
  Free variable

  Consider the formula: \[
   \forall x \, \Big(
   P(x) \wedge 
   {\color{black}\overbrace{\color{black}Q(y)}^{\color{black}\text{Free}}}
   \Big)
   \]

  In this formula, \(x\) is bound by the universal quantifier
  \(\forall x\), but \(y\) is free because it is not within the scope of
  any quantifier. The truth of this formula depends on the value of y,
  while x is quantified over all possible values.
\item
  Bound variable

  Consider the formula:

  \[
   {\color{black}\overbrace{\color{black}\exists x\,
   \big(
   P(x) \wedge Q(x)
   \big)
   }^{\color{black}\text{Scope of } \exists x}}
   \]

  In this formula, \(x\) is bound by the existential quantifier
  \(\exists x\). The truth of the formula does not depend on any
  external value of \(x\), because it is fully specified within the
  scope of the quantifier.
\item
  Mixed free and bound variables

  Consider the formula: \[
   \forall x \, \Big(
   P(x, {\color{black}\overbrace{\color{black}y}^{\color{black}\text{Free}}}) \rightarrow 
   \exists z \, 
   {\color{black}\overbrace{\color{black}Q(x, z)}^{\color{black}\text{Both bound}}}
   \Big)
   \]

  Here:

  − \(x\) is bound by \(\forall x\)

  − \(z\) is bound by \(\exists z\)

  − \(y\) is free (appears in \(P(x,y)\) but isn't quantified)

  The truth value depends on what \(y\) represents, while \(x\) and
  \(z\) are internally determined.
\item
  Nested quantifiers with name collision

  Consider the formula: \[
   \forall x \, \Big(
   P(x) \rightarrow 
   {\color{black}\overbrace{\color{black}\exists x \, R(x, y)}^{\color{black}\text{Inner } x \text{ bound}}}
   \Big)
   \]

  Here we have two different variables both named \(x\):

  − The outer \(x\) is bound by \(\forall x\) in \(P(x)\)

  − The inner \(x\) is bound by \(\exists x\) in \(R(x,y)\)

  − \(y\) is free throughout

  The inner quantifier ``shadows the outer one within its scope.
\item
  Multiple free variables

  Consider the formula: \[
   {\color{black}\overbrace{\color{black}P(a, b)}^{\color{black}\text{Both free}}} \wedge 
   \forall x \,
   {\color{black}\overbrace{\color{black}Q(x, c)}^{\color{black}c \text{ is free}}}
   \]

  Here:

  − \(a\), \(b\), and \(c\) are all free variables

  − \(x\) is bound by \(\forall x\)

  The truth depends on the specific values assigned to \(a\), \(b\), and
  \(c\).
\item
  Complex nesting

  Consider the formula: \[
   \exists x \, \forall y \, \Big(
   {\color{black}\overbrace{\color{black}P(x, y)}^{\color{black}\text{Both bound}}} \rightarrow
   \exists z \,
   \big(
   {\color{black}\overbrace{\color{black}Q(x, z)}^{\color{black}\text{Both bound}}} \wedge
   {\color{black}\overbrace{\color{black}R(w)}^{\color{black}w \text{ free}}}
   \big)
   \Big)
   \]

  Here:

  − \(x\) is bound by \(\exists x\) (outermost)

  − \(y\) is bound by \(\forall y\)

  − \(z\) is bound by the inner \(\exists z\)

  − \(w\) is free (not quantified anywhere)
\item
  No quantifiers (all free)

  Consider the formula: \[
   {\color{black}\overbrace{\color{black}P(a) \wedge Q(b, c) \rightarrow R(a, d)}^{\color{black}\text{All variables free}}}
   \]

  Since there are no quantifiers, all variables \(a\), \(b\), \(c\), and
  \(d\) are free. The truth value depends entirely on the interpretation
  of these variables.
\item
  Sequential quantifiers

  Consider the formula: \[
   \forall x \, \exists y \, \forall z \, 
   {\color{black}\overbrace{\color{black}P(x, y, z)}^{\color{black}\text{All bound}}} \wedge 
   {\color{black}\overbrace{\color{black}Q(w)}^{\color{black}w \text{ free}}}
   \]

  Here:

  − \(x\) is bound by \(\forall x\)

  − \(y\) is bound by \(\exists y\)

  − \(z\) is bound by \(\forall z\)

  − \(w\) is free

  The quantifiers create a chain: ``for all \(x\), there exists a \(y\),
  such that for all \(z\)\ldots''
\item
  Partial binding

  Consider the formula: \[
   \exists x \,
   \Big(
   {\color{black}\overbrace{\color{black}P(x)}^{\color{black}\text{Bound}}} \wedge 
   {\color{black}\overbrace{\color{black}Q(x, y)}^{\color{black}x \text{ bound, } y \text{ free}}}
   \Big) \rightarrow
   {\color{black}\overbrace{\color{black}R(y, z)}^{\color{black}\text{Both free}}}
   \]

  Here:

  − \(x\) is bound by \(\exists x\) within the antecedent

  − \(y\) appears both bound (in the scope of \(\exists x\)) and free
  (in \(R(y,z)\))

  − \(z\) is free throughout

  This shows how the same variable name can have different binding
  status in different parts of a formula.
\end{enumerate}

\subsection{12. Quantifier Equivalences}\label{quantifier-equivalences}

We understand that one logical symbol can often be rewritten in terms of
others. In propositional logic, for instance, the implication symbol
\((\rightarrow)\) can be expressed using disjunction \((\lor)\) and
negation \((\lnot)\). For example:

\[
p \rightarrow q \;\;\equiv\;\; \lnot p \lor q
\]

A similar relationship exists in picate logic between the universal
quantifier \((\forall)\) and the existential quantifier \((\exists)\),
connected through negation:

\[
\forall x \, c(x) \;\;\equiv\;\; \lnot \exists x \, \lnot c(x)
\]

This means: \emph{``For all \(x\), \(c(x)\) holds''} is equivalent to
\emph{``There does not exist an \(x\) such that \(c(x)\) does not
hold.''}

Conversely:

\[
\exists x \, c(x) \;\;\equiv\;\; \lnot \forall x \, \lnot c(x)
\]

This means: \emph{``There exists an \(x\) such that \(c(x)\) holds''} is
equivalent to \emph{``It is not the case that \(c(x)\) fails for all
\(x\).''}

These equivalences also extend naturally to statements beginning with
negation:

\[
\lnot \forall x \, c(x) \;\;\equiv\;\; \exists x \, \lnot c(x)
\]

This means: \emph{``It is not true that \(c(x)\) holds for all \(x\)''}
is logically equivalent to saying \emph{``There exists at least one
\(x\) such that \(c(x)\) does not hold.''}

And similarly:

\[
\lnot \exists x \, c(x) \;\;\equiv\;\; \forall x \, \lnot c(x)
\]

This means: \emph{``It is not true that there exists an \(x\) such that
\(c(x)\) holds''} is logically equivalent to saying \emph{``For every
\(x\), \(c(x)\) does not hold.''}

\subsection{13. Symbolizing Syllogistic
Reasoning}\label{symbolizing-syllogistic-reasoning}

Let's now translate classical syllogisms into predicate logic. We'll use
the following key:

Key for predicates: − \(M(x)\): \(x\) is an \(M\) (middle term) −
\(S(x)\): \(x\) is an \(S\) (subject term) − \(P(x)\): \(x\) is a \(P\)
(predicate term)

\begin{enumerate}
\def\labelenumi{\arabic{enumi}.}
\item
  Barbara (AAA)

  Form: − All \(M\) are \(P\) − All \(S\) are \(M\) − \(\therefore\) All
  \(S\) are \(P\)

  Symbolization: − \(\forall x \, (M(x) \rightarrow P(x))\) −
  \(\forall x \, (S(x) \rightarrow M(x))\) − \(\therefore\)
  \(\forall x \, (S(x) \rightarrow P(x))\)

  Everything \(M\) is included in \(P\). Since all \(S\) fall under
  \(M\), they too must fall under \(P\).
\item
  Celarent (EAE)

  Form: − No \(M\) are \(P\) − All \(S\) are \(M\) − \(\therefore\) No
  \(S\) are \(P\)

  Symbolization: − \(\forall x \, (M(x) \rightarrow \lnot P(x))\) −
  \(\forall x \, (S(x) \rightarrow M(x))\) − \(\therefore\)
  \(\forall x \, (S(x) \rightarrow \lnot P(x))\)

  Being an \(M\) rules out being a \(P\). Since all \(S\) are \(M\),
  none of them can possibly be \(P\).
\item
  Darii (AII)

  Form: − All \(M\) are \(P\) − Some \(S\) are \(M\) − \(\therefore\)
  Some \(S\) are \(P\)

  Symbolization: − \(\forall x \, (M(x) \rightarrow P(x))\) −
  \(\exists x \, (S(x) \land M(x))\) − \(\therefore\)
  \(\exists x \, (S(x) \land P(x))\)

  The universal ensures every \(M\) is also \(P\). Since at least one
  \(S\) is an \(M\), that same \(S\) must also be a \(P\).
\item
  Ferio (EIO)

  Form: − No \(M\) are \(P\) − Some \(S\) are \(M\) − \(\therefore\)
  Some \(S\) are not \(P\)

  Symbolization: − \(\forall x \, (M(x) \rightarrow \lnot P(x))\) −
  \(\exists x \, (S(x) \land M(x))\) − \(\therefore\)
  \(\exists x \, (S(x) \land \lnot P(x))\)

  If being \(M\) excludes being \(P\), and some \(S\) is \(M\), then
  that same \(S\) must also not be \(P\).
\item
  Cesare (EAE)

  Form: − No \(P\) are \(M\) − All \(S\) are \(M\) − \(\therefore\) No
  \(S\) are \(P\)

  Symbolization: − \(\forall x \, (P(x) \rightarrow \lnot M(x))\) −
  \(\forall x \, (S(x) \rightarrow M(x))\) − \(\therefore\)
  \(\forall x \, (S(x) \rightarrow \lnot P(x))\)

  If no \(P\) can be an \(M\), then nothing counted as \(P\) overlaps
  with \(M\). Since all \(S\) are \(M\), none of them can belong to
  \(P\).
\item
  Camestres (AEE)

  Form: − All \(P\) are \(M\) − No \(S\) are \(M\) − \(\therefore\) No
  \(S\) are \(P\)

  Symbolization: − \(\forall x \, (P(x) \rightarrow M(x))\) −
  \(\forall x \, (S(x) \rightarrow \lnot M(x))\) − \(\therefore\)
  \(\forall x \, (S(x) \rightarrow \lnot P(x))\)

  Everything \(P\) is also \(M\). But \(S\) can never be \(M\). So \(S\)
  can never be \(P\) either.
\item
  Festino (EIO)

  Form: − No \(M\) are \(P\) − Some \(S\) are \(M\) − \(\therefore\)
  Some \(S\) are not \(P\)

  Symbolization: − \(\forall x \, (M(x) \rightarrow \lnot P(x))\) −
  \(\exists x \, (S(x) \land M(x))\) − \(\therefore\)
  \(\exists x \, (S(x) \land \lnot P(x))\)

  If \(M\) excludes \(P\), then any \(S\) that happens to be \(M\)
  cannot be \(P\).
\item
  Baroco (AOO)

  Form:

  − All \(M\) are \(P\) − Some \(S\) are not \(M\) − \(\therefore\) Some
  \(S\) are not \(P\)

  Symbolization: − \(\forall x \, (M(x) \rightarrow P(x))\) −
  \(\exists x \, (S(x) \land \lnot M(x))\) −
  \(\therefore \, \exists x \, (S(x) \land \lnot P(x))\)

  Explanation: Every \(M\) is guaranteed to be \(P\). But since there
  are some \(S\) that are outside \(M\), those \(S\) might fall outside
  \(P\) as well.
\item
  Bocardo (OAO)

  Form: − Some \(M\) are not \(P\) − All \(M\) are \(S\) −
  \(\therefore\) Some \(S\) are not \(P\)

  Symbolization: − \(\exists x \, (M(x) \land \lnot P(x))\) −
  \(\forall x \, (M(x) \rightarrow S(x))\) − \(\therefore\)
  \(\exists x \, (S(x) \land \lnot P(x))\)

  The first premise ensures at least one \(M\) is outside \(P\). Since
  all \(M\) are also \(S\), that \(M\) is also an \(S\). Thus, there
  exists an \(S\) that is not \(P\).
\item
  Camenes (AEE)

  Form: − All \(P\) are \(M\) − No \(S\) are \(M\) − \(\therefore\) No
  \(S\) are \(P\)

  Symbolization: − \(\forall x \, (P(x) \rightarrow M(x))\) −
  \(\forall x \, (S(x) \rightarrow \lnot M(x))\) − \(\therefore\)
  \(\forall x \, (S(x) \rightarrow \lnot P(x))\)

  If being \(P\) always means being \(M\), but no \(S\) can ever be
  \(M\), then it follows that no \(S\) can possibly be \(P\).
\item
  Dimaris (IAI)

  Form: − Some \(M\) are \(P\) − All \(S\) are \(M\) − \(\therefore\)
  Some \(S\) are \(P\) Symbolization: −
  \(\exists x \, (M(x) \land P(x))\) −
  \(\forall x \, (S(x) \rightarrow M(x))\) − \(\therefore\)
  \(\exists x \, (S(x) \land P(x))\)

  If at least one \(M\) is a \(P\), and all \(S\) are \(M\), then that
  \(M\) is also an \(S\), making some \(S\) a \(P\).
\item
  Felapton (EIO)

  Form: − No \(M\) are \(P\) − All \(M\) are \(S\) − \(\therefore\) Some
  \(S\) are not \(P\)

  Symbolization: − \(\forall x \, (M(x) \rightarrow \lnot P(x))\) −
  \(\forall x \, (M(x) \rightarrow S(x))\) − \(\therefore\)
  \(\exists x \, (S(x) \land \lnot P(x))\)

  If nothing that is \(M\) can be \(P\), and all \(M\) are also \(S\),
  then some \(S\) must fall outside \(P\).
\item
  Ferison (EIO)

  Form: − No \(M\) are \(P\) − Some \(M\) are \(S\) − \(\therefore\)
  Some \(S\) are not \(P\)

  Symbolization: − \(\forall x \, (M(x) \rightarrow \lnot P(x))\) −
  \(\exists x \, (M(x) \land S(x))\) − \(\therefore\)
  \(\exists x \, (S(x) \land \lnot P(x))\)

  If no \(M\) is \(P\), but some \(M\) is also an \(S\), then that \(S\)
  cannot be a \(P\).
\item
  Disamis (IAI)

  Form: − Some \(M\) are \(P\) − All \(M\) are \(S\) − \(\therefore\)
  Some \(S\) are \(P\)

  Symbolization: − \(\exists x \, (M(x) \land P(x))\) −
  \(\forall x \, (M(x) \rightarrow S(x))\) − \(\therefore\)
  \(\exists x \, (S(x) \land P(x))\)

  If at least one \(M\) is a \(P\), and all \(M\) are also \(S\), then
  that same element is both \(S\) and \(P\).
\item
  Datisi (AII)

  Form: − All \(M\) are \(P\) − Some \(M\) are \(S\) − \(\therefore\)
  Some \(S\) are \(P\)

  Symbolization: − \(\forall x \, (M(x) \rightarrow P(x))\) −
  \(\exists x \, (M(x) \land S(x))\) − \(\therefore\)
  \(\exists x \, (S(x) \land P(x))\)

  If every \(M\) is \(P\), and at least one \(M\) is also an \(S\), then
  that individual is both \(S\) and \(P\).
\item
  Bramantip (AAI)

  Form: − All \(P\) are \(M\) − All \(M\) are \(S\) − \(\therefore\)
  Some \(S\) are \(P\)

  Symbolization: − \(\forall x \, (P(x) \rightarrow M(x))\) −
  \(\forall x \, (M(x) \rightarrow S(x))\) − \(\therefore\)
  \(\exists x \, (S(x) \land P(x))\)

  If everything \(P\) is \(M\), and every \(M\) is \(S\), then the class
  of \(P\) must overlap with \(S\), so some \(S\) are \(P\).
\item
  Camenes (AEE)

  Form: − All \(P\) are \(M\) − No \(S\) are \(M\) − \(\therefore\) No
  \(S\) are \(P\)

  Symbolization: − \(\forall x \, (P(x) \rightarrow M(x))\) −
  \(\forall x \, (S(x) \rightarrow \lnot M(x))\) − \(\therefore\)
  \(\forall x \, (S(x) \rightarrow \lnot P(x))\)

  If being \(P\) always means being \(M\), but no \(S\) can ever be
  \(M\), then it follows that no \(S\) can possibly be \(P\).
\item
  Fesapo (EAO)

  Form: − No \(M\) are \(P\) − All \(M\) are \(S\) − \(\therefore\) Some
  \(S\) are not \(P\)

  Symbolization: − \(\forall x \, (M(x) \rightarrow \lnot P(x))\) −
  \(\forall x \, (M(x) \rightarrow S(x))\) − \(\therefore\)
  \(\exists x \, (S(x) \land \lnot P(x))\)

  If no \(M\) can ever be \(P\), and all \(M\) are also \(S\), then
  there must be some \(S\) that are not \(P\).
\item
  Fresison (EIO)

  Form: − No \(M\) are \(P\) − Some \(M\) are \(S\) − \(\therefore\)
  Some \(S\) are not \(P\)

  Symbolization: − \(\forall x \, (M(x) \rightarrow \lnot P(x))\) −
  \(\exists x \, (M(x) \land S(x))\) − \(\therefore\)
  \(\exists x \, (S(x) \land \lnot P(x))\)

  If no \(M\) is a \(P\), but some \(M\) are also \(S\), then those
  \(S\) cannot be \(P\).
\end{enumerate}
