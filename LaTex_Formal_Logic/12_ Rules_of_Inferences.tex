Rules of Inferences

Rules of inference are a list of rules used for natural deduction.
Therefore, it is crucial to learn these rules before entering the
natural deduction chapter.

\subsection{\texorpdfstring{1. Modus Ponens
(\(MP\))}{1. Modus Ponens (MP)}}\label{modus-ponens-mp}

The modus ponens rule is the process by which we affirm the antecedent.

First, we assume \[a \rightarrow b\]

Then, we consider it true that \[a\]

As a result, by logical consequence, then \[b\]

Overall, it can be written as follows:

\[\frac{a  \rightarrow b, \quad a}{\therefore b}\]

\subsection{\texorpdfstring{2. Modus Tollens
(\(MT\))}{2. Modus Tollens (MT)}}\label{modus-tollens-mt}

The modus tollens rule is the process by which we deny the consequent.

First, we assume \[a \rightarrow  b\]

Then, we consider it true that \(b\) is not true, so \[\neg  b\]

As a result, by logical consequence, then \[\neg a\]

Overall, it can be written as follows:

\[\frac{a \rightarrow b, \quad \neg b}{\therefore \neg a}\]

\subsection{\texorpdfstring{3. Hypothetical Syllogism
(\(HS\))}{3. Hypothetical Syllogism (HS)}}\label{hypothetical-syllogism-hs}

Hypothetical Syllogism is a chain of proof, or often also called
Transitive. The process is as follows:

First, we assume that \[a \rightarrow b\]

Then, we assume \[b \rightarrow c\]

So logically we can assume that \[a \rightarrow c\] can be considered
true

Overall, it can be written as follows:

\[\frac{a \rightarrow b, \quad b \rightarrow c}{\therefore a \rightarrow c}\]

\subsection{\texorpdfstring{4. Disjunctive Syllogism
(\(DS\))}{4. Disjunctive Syllogism (DS)}}\label{disjunctive-syllogism-ds}

The reasoning process of \(DS\) is by eliminating one of the
propositions. For example:

We assume \[a \lor b\]

Then from that statement, we know that \[\neg a\]

So, the final result is \[b\]

Overall, it can be written as follows:

\[\frac{a \lor b, \quad \neg a}{\therefore b}\]

\subsection{\texorpdfstring{5. Constructive Dilemma
(\(CD\))}{5. Constructive Dilemma (CD)}}\label{constructive-dilemma-cd}

Constructive Dilemma is a form of deductive reasoning that combines two
conditional propositions with a disjunction. The pattern is as follows:

\begin{enumerate}
\def\labelenumi{\arabic{enumi}.}
\item
  First premise:\\
  \[a \rightarrow b\]
\item
  Second premise:\\
  \[c \rightarrow d\]
\item
  Third premise (disjunction on the antecedent):\\
  \[a \lor c\]
\item
  Then the conclusion is:\\
  \[b \lor d\]
\end{enumerate}

Overall, the general form of Constructive Dilemma can be written as
follows:

\[
\frac{a \rightarrow b, \quad c \rightarrow d, \quad a \lor c}{\therefore b \lor d}
\]

\subsection{\texorpdfstring{6. Addition
(\(Add\))}{6. Addition (Add)}}\label{addition-add}

The Addition (\(Add\)) pattern is one of the simplest forms of
inference. Essentially, if we have a true proposition, then we can add
another proposition through logical operations (disjunction or
conjunction).

For example, suppose we know that:\\
\[a\]

Then, we can add another proposition and obtain:

\begin{enumerate}
\def\labelenumi{\arabic{enumi}.}
\item
  With disjunction:\\
  \[a \lor b\]

  Overall, it can be written as follows:

  \[
   \frac{a}{\therefore a \lor b}
   \]
\item
  With conjunction:\\
  \[a \land c\]

  Overall, it can be written as follows:

  \[
   \frac{a}{\therefore a \land c}
   \]
\end{enumerate}

\subsection{\texorpdfstring{7. Simplification
(\(Simp\))}{7. Simplification (Simp)}}\label{simplification-simp}

The Simplification (\(Simp\)) pattern, also called Conjunction
Elimination, is an inference rule that allows us to draw conclusions
from a conjunction. If we know that a conjunction is true, then each
part of that conjunction is also true.

For example, suppose we have:\\
\[a \land b\]

Then we can conclude:\\
\[a\]\\
or\\
\[b\]

In general, the Simplification pattern can be written as follows:

\[
\frac{a \land b}{\therefore a}
\qquad \text{or} \qquad
\frac{a \land b}{\therefore b}
\]

\subsection{\texorpdfstring{8. Disjunction Elimination
(\(DE\))}{8. Disjunction Elimination (DE)}}\label{disjunction-elimination-de}

The Disjunction Elimination pattern is an inference rule that allows us
to draw conclusions from a disjunction. If we know that a disjunction is
true, and from each of its alternatives we can conclude the same
proposition, then we can conclude that proposition.

For example, suppose we have:\\
\[a \lor b\]\\
\[a \rightarrow c\]\\
\[b \rightarrow c\]

Then we can conclude:\\
\[c\]

In general, the Disjunction Elimination pattern can be written as
follows:

\[
\frac{a \lor b, \quad a \rightarrow c, \quad b \rightarrow c}{\therefore c}
\]

\subsection{\texorpdfstring{9. Resolution
(\(Res\))}{9. Resolution (Res)}}\label{resolution-res}

The Resolution (\(Res\)) pattern is an inference rule widely used in
formal logic and automated proof. This rule works by eliminating a
proposition and its negation from two disjunctions, to then produce a
new disjunction.

In general, if we have:\\
\[(a \lor b), \quad (\lnot a \lor c)\]

Then we can conclude:\\
\[b \lor c\]

In other words, variable \(a\) is eliminated because it appears in
positive form in the first premise, and in negative form in the second
premise.

The Resolution pattern can be written as follows:

\[
\frac{a \lor b, \quad \lnot a \lor c}{\therefore b \lor c}
\]

\subsection{\texorpdfstring{10. Double Negation
(\(DN\))}{10. Double Negation (DN)}}\label{double-negation-dn}

The Double Negation (\(DN\)) pattern states that the double negation of
a proposition is equivalent to the proposition itself. This means that
if we have a proposition preceded by two negation signs, then both can
be removed without changing the truth value.

In general:\\
\[a \equiv \lnot \lnot a\]

The truth of the statement above can be proven with a truth table:

\(a\)

\(\lnot a\)

\(\lnot \lnot a\)

\(a \equiv \lnot \lnot a\)

1

0

1

1

0

1

0

1

The Double Negation inference rule can be written as follows:

\[
\frac{\lnot \lnot a}{\therefore a}
\]

\subsection{\texorpdfstring{11. Commutation
(\(Comm\))}{11. Commutation (Comm)}}\label{commutation-comm}

The Commutation (\(Comm\)) pattern states that the order of propositions
connected by logical operations of conjunction (\(\land\)) and
disjunction (\(\lor\)) can be exchanged without changing the truth
value.

In general:

\begin{enumerate}
\def\labelenumi{\arabic{enumi}.}
\item
  For conjunction:\\
  \[a \land b \equiv b \land a\]
\item
  For disjunction:\\
  \[a \lor b \equiv b \lor a\]
\end{enumerate}

The Commutation inference rule can be written as follows:

\[
\frac{a \land b}{\therefore b \land a}
\qquad \text{or} \qquad
\frac{a \lor b}{\therefore b \lor a}
\]

\subsection{\texorpdfstring{12. Association
(\(Assoc\))}{12. Association (Assoc)}}\label{association-assoc}

The Association (\(Assoc\)) pattern states that grouping of propositions
connected by conjunction (\(\land\)) and disjunction (\(\lor\)) does not
affect the truth value. In other words, parentheses can be moved without
changing the logical meaning.

In general:

\begin{enumerate}
\def\labelenumi{\arabic{enumi}.}
\item
  For conjunction:\\
  \[(a \land (b \land c)) \equiv ((a \land b) \land c)\]
\item
  For disjunction:\\
  \[(a \lor (b \lor c)) \equiv ((a \lor b) \lor c)\]
\end{enumerate}

The Association inference rule can be written as follows:

\[
\frac{a \land (b \land c)}{\therefore (a \land b) \land c}
\qquad \text{or} \qquad
\frac{a \lor (b \lor c)}{\therefore (a \lor b) \lor c}
\]

\subsection{\texorpdfstring{13. Distribution
(\(Dist\))}{13. Distribution (Dist)}}\label{distribution-dist}

The Distribution (\(Dist\)) pattern states that conjunction (\(\land\))
can be distributed over disjunction (\(\lor\)), and vice versa. This
rule allows us to change the form of a logical proposition without
changing its truth value.

In general:

\begin{enumerate}
\def\labelenumi{\arabic{enumi}.}
\item
  Conjunction over disjunction:\\
  \[a \land (b \lor c) \equiv (a \land b) \lor (a \land c)\]
\item
  Disjunction over conjunction:\\
  \[a \lor (b \land c) \equiv (a \lor b) \land (a \lor c)\]
\end{enumerate}

The Distribution inference rule can be written as follows:

\[
\frac{a \land (b \lor c)}{\therefore (a \land b) \lor (a \land c)}
\qquad \text{or} \qquad
\frac{a \lor (b \land c)}{\therefore (a \lor b) \land (a \lor c)}
\]

\subsection{\texorpdfstring{14. De Morgan's Laws
(\(DeM\))}{14. De Morgan's Laws (DeM)}}\label{de-morgans-laws-dem}

De Morgan's Laws (\(DeM\)) pattern states the relationship between
negation and conjunction (\(\land\)) and disjunction (\(\lor\)). This
law shows how the negation of a conjunction or disjunction can be
rewritten in equivalent form.

In general:

\begin{enumerate}
\def\labelenumi{\arabic{enumi}.}
\item
  Negation of conjunction:\\
  \[\lnot (a \land b) \equiv (\lnot a \lor \lnot b)\]
\item
  Negation of disjunction:\\
  \[\lnot (a \lor b) \equiv (\lnot a \land \lnot b)\]
\end{enumerate}

The truth of the statement above can be proven with truth tables:

Truth table 1:

\(a\)

\(b\)

\(\neg a\)

\(\neg b\)

\(a \land b\)

\(\neg (a \land b)\)

\(\neg a \lor \neg b\)

1

1

0

0

1

0

0

1

0

0

1

0

1

1

0

1

1

0

0

1

1

0

0

1

1

0

1

1

Truth table 2:

\(a\)

\(b\)

\(\neg a\)

\(\neg b\)

\(a \lor b\)

\(\neg (a \lor b)\)

\(\neg a \land \neg b\)

1

1

0

0

1

0

0

1

0

0

1

1

0

0

0

1

1

0

1

0

0

0

0

1

1

0

1

1

De Morgan's Laws inference rule can be written as follows:

\[
\frac{\lnot (a \land b)}{\therefore \lnot a \lor \lnot b}
\qquad \text{or} \qquad
\frac{\lnot (a \lor b)}{\therefore \lnot a \land \lnot b}
\]

\subsection{\texorpdfstring{15. Implication
(\(Impl\))}{15. Implication (Impl)}}\label{implication-impl}

The Implication (\(Impl\)) pattern states that an implication can be
rewritten in disjunctive form.

In general:\\
\[a \rightarrow b \equiv \lnot a \lor b\]

The truth of the statement above can be proven with a truth table:

\(a\)

\(b\)

\(\neg a\)

\(\neg b\)

\(\neg a \lor b\)

\(a \rightarrow b\)

1

1

0

0

1

1

1

0

0

1

0

0

0

1

1

0

1

1

0

0

1

1

1

1

The Implication inference rule can be written as follows:

\[
\frac{a \rightarrow b}{\therefore \lnot a \lor b}
\]

\subsection{\texorpdfstring{16. Exportation
(\(Exp\))}{16. Exportation (Exp)}}\label{exportation-exp}

The Exportation (\(Exp\)) pattern states that a nested implication is
equivalent to an implication from the conjunction of antecedents.

In general (equivalence):
\[(a \rightarrow (b \rightarrow c)) \equiv ((a \land b) \rightarrow c)\]

The truth of the statement above can be proven with a truth table:

\(a\)

\(b\)

\(c\)

\(\neg a\)

\(\neg b\)

\(\neg c\)

\(b \rightarrow c\)

\(a \rightarrow (b \rightarrow c)\)

\(a \land b\)

\((a \land b) \rightarrow c\)

1

1

1

0

0

0

1

1

1

1

1

1

0

0

0

1

0

0

1

0

1

0

1

0

1

0

1

1

0

1

1

0

0

0

1

1

1

1

0

1

0

1

1

1

0

0

1

1

0

1

0

1

0

1

0

1

0

1

0

1

0

0

1

1

1

0

1

1

0

1

0

0

0

1

1

1

1

1

0

1

The Exportation inference rule can be written in both directions as
follows:

\[
\frac{a \rightarrow (b \rightarrow c)}{\therefore (a \land b) \rightarrow c}
\qquad \text{and} \qquad
\frac{(a \land b) \rightarrow c}{\therefore a \rightarrow (b \rightarrow c)}
\]

\subsection{\texorpdfstring{17. Contrapositive
(\(Contra\))}{17. Contrapositive (Contra)}}\label{contrapositive-contra}

The Contrapositive rule states that an implication is equivalent to its
contrapositive. This means that if a proposition has the form:

\[a \rightarrow b\]

then it is logically equivalent to:

\[\lnot b \rightarrow \lnot a\]

The truth of the statement above can be proven with a truth table:

\(a\)

\(b\)

\(\lnot a\)

\(\lnot b\)

\(a \rightarrow b\)

\(\lnot b \rightarrow \lnot a\)

1

1

0

0

1

1

1

0

0

1

0

0

0

1

1

0

1

1

0

0

1

1

1

1

The Contrapositive inference rule can be written as follows:

\[
(a \rightarrow b) \equiv (\lnot b \rightarrow \lnot a)
\]

\subsection{\texorpdfstring{18. Reductio Ad Absurdum
(\(RAA\))}{18. Reductio Ad Absurdum (RAA)}}\label{reductio-ad-absurdum-raa}

Reductio Ad Absurdum (\(RAA\)), often called indirect proof, is an
inference rule that states that if by assuming the negation of a
proposition we arrive at a contradiction, then that proposition must be
true. In other words, if the assumption \(\neg a\) produces a
contradiction (for example \(a \land \neg a\)), then we can conclude
that \(a\) is true.

For example, suppose we assume:\\
\[\neg a\]

and from that assumption we can derive a contradiction:\\
\[a \land \neg a\]

Then we can conclude that:\\
\[a\]

In general, the Reductio Ad Absurdum pattern can be written as follows:

\[
\frac{\neg a \;\; \vdash \;\; (a \land \neg a)}{\therefore a}
\]

\subsection{\texorpdfstring{19. Conditional Proof
(\(CP\))}{19. Conditional Proof (CP)}}\label{conditional-proof-cp}

Conditional Proof (\(CP\)) is an inference rule used to prove
implications. Essentially, if by assuming premise \(a\) we can derive
conclusion \(b\), then we can conclude that \(a \rightarrow b\) is true.

For example, suppose we want to prove:\\
\[a \rightarrow b\]

From the following complex series of statements:

\begin{enumerate}
\def\labelenumi{\arabic{enumi}.}
\tightlist
\item
  \(a \rightarrow (c \land d)\)\\
\item
  \(c \rightarrow e\)\\
\item
  \(d \rightarrow f\)\\
\item
  \(e \land f \rightarrow b\)
\end{enumerate}

Steps using Conditional Proof:

\begin{longtable}[]{@{}
  >{\raggedright\arraybackslash}p{(\linewidth - 4\tabcolsep) * \real{0.1875}}
  >{\raggedright\arraybackslash}p{(\linewidth - 4\tabcolsep) * \real{0.3438}}
  >{\raggedright\arraybackslash}p{(\linewidth - 4\tabcolsep) * \real{0.4688}}@{}}
\toprule\noalign{}
\begin{minipage}[b]{\linewidth}\raggedright
Step
\end{minipage} & \begin{minipage}[b]{\linewidth}\raggedright
Statement
\end{minipage} & \begin{minipage}[b]{\linewidth}\raggedright
Justification
\end{minipage} \\
\midrule\noalign{}
\endhead
\bottomrule\noalign{}
\endlastfoot
1. & \(a\) & Assumption (for \(CP\)) \\
2. & \(c \land d\) & Modus Ponens: Step 1, Premise 1
(\(a \rightarrow (c \land d)\)) \\
3. & \(c\) & Simplification: Step 2 \\
4. & \(d\) & Simplification: Step 2 \\
5. & \(e\) & Modus Ponens: Step 3, Premise 2 (\(c \rightarrow e\)) \\
6. & \(f\) & Modus Ponens: Step 4, Premise 3 (\(d \rightarrow f\)) \\
7. & \(e \land f\) & Conjunction: Step 5, Step 6 \\
8. & \(b\) & Modus Ponens: Step 7, Premise 4
(\(e \land f \rightarrow b\)) \\
9. & \(a \rightarrow b\) & Conditional Proof: Steps 1-8 \\
\end{longtable}

If from assumption \(a\) we can derive \(b\), then we can conclude
\(a \rightarrow b\).

In general, the Conditional Proof pattern can be written as follows:

\[
\frac{a \;\; \vdash \;\; b}{\therefore a \rightarrow b}
\]
