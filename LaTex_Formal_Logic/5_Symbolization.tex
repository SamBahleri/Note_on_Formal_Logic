Symbolization

Before we symbolizing statements using formal logical symbols, its
important to clarify two key concepts. Firts, The object Language, and
second, The Metalanguage:

\begin{enumerate}
\def\labelenumi{\arabic{enumi}.}
\item
  Th Object Language

  The object Language is the formal language we analyze, consiting
  symbols like \(p, q, \wedge, \vee, \neg\), and sentences we form with
  these symbols
\item
  The Metalanguage

  The Meta Language is the language we use to talk about this object
  language, wich in this article is mostly English. We sue the
  Metalanguage to explain the meanning and rules of the Object Language,
  such as what makes a formulawell-formed or how the connectives
  function.
\end{enumerate}

When talking about Symbolization, we are dealing with the role of
symbols, their purpose, and their essence as a formal system in
analyzing and representing the truth of what is conveyed through
symbols. However, once again, we must remember that in the previous
material, we learned that the formal and informal, where the processes
are deduction and induction, complement each other. Then, arguments
formed from premises that are not logically connected can result in
invalid forms, even though the content of those premises may be
factually true. Furthermore, sound reasoning is reasoning based on the
following points:

\begin{enumerate}
\def\labelenumi{\arabic{enumi}.}
\item
  The initial stage of reasoning is empirical experience
\item
  The second stage is representation in natural language
\item
  The third stage is symbolic abstraction
\end{enumerate}

\subsubsection{1. Role}\label{role}

\begin{enumerate}
\def\labelenumi{\arabic{enumi})}
\item
  Proposition

  Remember that a proposition like ``All humans are living beings'' can
  be represented simply by \(p\), and the same applies to other
  propositions, which can be symbolized as \(p\), \(q\), \(r\), \(s\),
  etc.
\item
  Relation

  To express logical relations between propositions in symbolic form, we
  first need to understand how these relations are conveyed through
  logical symbols. These symbols represent the connections between
  propositions and serve as the foundation for constructing arguments in
  propositional logic (PL). The key types of logical symbols used to
  represent these connections are called truth-functional connectives.
  These connectives determine the truth value of a compound proposition
  solely based on the truth values of its component propositions. Some
  commonly used truth-functional connectives include:

  \begin{enumerate}
  \def\labelenumii{\arabic{enumii})}
  \item
    Disjunction (\(\lor\))

    In everyday language, the word ``or'' can represent two meanings:
    first, inclusive or, and second, exclusive or. At this initial
    stage, we will focus on learning inclusive or first. It is important
    to remember that the word ``or'' in formal logic is not the same as
    the word ``or'' used to indicate equivalence. For example:

    \[
     \text{Orang utan or Pongo pygmaeus}
     \]

    In this context, `or' is used in a definitional or equivalence
    sense, which is different from the logical connective disjunction.
    Logical equivalence is expressed with the biconditional
    (\(\leftrightarrow\))

    This kind of ``or'':

    \begin{enumerate}
    \def\labelenumiii{\alph{enumiii}.}
    \item
      It does not represent a truth value (true or false), meaning there
      is no choice involved.
    \item
      It is definitional in nature.
    \end{enumerate}

    The representation of the relation expressed by logical ``or'' is as
    follows:

    Example 1:

    \[ \text {1+1 = 2 } \quad \text{or} \quad \text{3-2=1} \]

    Let:

    − \(p\): \(1 + 1 = 2\)

    − \(q\): \(3 - 2 = 1\)

    Then the symbolic form is:

    \[
     p \lor q
     \]

    Example 2:

    \[ \text {Koala } \quad \text{and} \quad \text{panda} \]

    Let:

    − \(k\): Koala

    − \(p\): Panda

    Then the symbolic form is:

    \[
     k \lor p
     \]
  \item
    Conjunction (\(\land\))

    In everyday language, the word ``and'' is used to connect two
    propositions or statements, both of which are expected to be true or
    to occur simultaneously. In symbolic logic, this relation is
    represented by the symbol \(\land\), known as conjunction.

    Example 1:

    \[ \text {1 < 2} \quad \text{and} \quad \text{3 > 2} \]

    Let:

    − \(x\): \(1 < 2\)

    − \(y\): \(3 > 2\)

    Then the symbolic form is:

    \[
     x \land y
     \]

    Example 2:

    \[ \text {Sun} \quad \text{and} \quad \text{Moon} \]

    Let:

    − \(s\): Sun

    − \(m\): Moon

    Then the symbolic form is:

    \[
     s \land m
       \]
  \item
    Conditional (\(\rightarrow\))

    Conditional logic is a form of logic that represents the logical
    relationship between two events, the first is called the antecedent,
    and the second is the result or consequent.

    Example 1:

    \[\text {If it rains, then the ground will be wet.}\]

    Let:

    − \(r\): It rains

    − \(w\): The ground is wet

    Then, in symbolic form:

    \[
     r \rightarrow w
     \]
  \item
    Negation (\(\neg\))

    Negation is a form of logic that represents the denial or rejection
    of a proposition, regardless of whether the proposition is true or
    false. In formal logic notation, negation is typically represented
    by the symbol ¬ or \textasciitilde. Negation plays an important role
    in various forms of logical reasoning, one of which is Reductio ad
    absurdum (reduction to absurdity). In this method, we deliberately
    assume that the proposition we want to prove is false (i.e., it is
    negated), and then derive various logical consequences from that
    assumption. If the result leads to a contradiction or absurdity,
    then the initial assumption (the negation of the proposition) must
    be false, which means that the original proposition is true.

    Example:

    Let:

    − \(p\): All prime numbers are odd

    We know that the statement above is false, so in symbolic form:

    \[
     \neg p
     \]

    This means \$ \text{not p}\$, because the number 2 is a prime number
    and even, which contradicts the original statement. Therefore, we
    negate the statement.
  \item
    Biconditional (\$\leftrightarrow \$)

    Biconditional logic is a type of relational logic, similar to
    implication. However, the difference is that biconditional logic
    requires both propositions to have the same truth value, either both
    true or both false.

    Example 1:

    \[\text {Laika is a dog if and only if she is a mammal.}\]

    Let:

    − \(l\): Laika is a dog

    − \(m\): She is a mammal

    Then, in symbolic form:

    \[
     l \leftrightarrow m
     \]

    Example 2:

    \[\text {3 is a prime number if and only if it has exactly two positive divisors.}\]

    Given:

    − \(p\) : 3 is a prime number

    − \(d\): 3 has exactly two positive divisors

    Then, in symbolic form:

    \[ p\leftrightarrow d\]
  \end{enumerate}
\end{enumerate}

\subsubsection{2. Purpose}\label{purpose}

The purpose of using logical symbols is to represent the truth value of
a statement in a formal and systematic way. By using symbols, we can
express and analyze the relationships between propositions accurately,
consistently, and free from the ambiguities of natural language. To draw
logical conclusions from a combination of several propositions (\(p\),
\(q\), \(r\), \(s\)), we must first know the truth value of each
individual proposition. In propositional logic, each proposition has
only two possible truth values:

\begin{enumerate}
\def\labelenumi{\arabic{enumi}.}
\item
  True, represented by \(T\) or the number \(1\)
\item
  False, represented by \(F\) or the number \(0\)
\end{enumerate}

Each proposition represents a single statement and has its own truth
value, depending on the context or situation being discussed. To
represent all possible truth values of one or more propositions, as well
as the results of logical operations, we use a tool called a truth
table. We will learn it in the next chapter
