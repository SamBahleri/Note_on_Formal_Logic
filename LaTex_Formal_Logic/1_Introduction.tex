
\subsection{1. Etymology and
Terminology}\label{etymology-and-terminology}

Normally, in the study of logic, the first thing we need to understand
is the question: what is logic in terms of its definition?
Etymologically, the word logic comes from the Greek word λόγος (logos).
The word logos carries various meanings, including ``word,'' ``speech,''
``reason,'' ``explanation,'' and ``principle.'' Over time, this term was
adopted into Latin as logica, which means the art or science of
reasoning. On the other hand, in terms of terminology, logic is the
systematic study of valid inference and correct reasoning. Therefore, it
can be understood that when we study logic, what we are learning is how
to evaluate things rationally and systematically to reach a solid
understanding and draw a sound conclusion. One striking statement about
logic comes from the philosopher John Locke, who said that ``logic is
the anatomy of thought.'' In this metaphorical statement, we can also
understand that by studying logic, we are essentially learning about the
structure of thought itself.

\subsection{2. Logic in History}\label{logic-in-history}

\subsubsection{2.1. Philosophical
revolution}\label{philosophical-revolution}

In this era, humans began to think not only about how to live in their
environment, but also about themselves, truth, and ideas. One figure
whose statement represents this revolution is
\href{https://en.wikipedia.org/wiki/Socrates}{Socrates}, with his famous
quote: ``The unexamined life is not worth living.''. To understand the
systematic development of logic, we can follow the stages of history
below:

\paragraph{2.1.1. Pre-Aristotelian Era}\label{pre-aristotelian-era}

In this era, humans tried to understand the world through myth,
narratives shaped by imagination and traditional constructions. Thinkers
of this era include:

\begin{enumerate}
\def\labelenumi{\alph{enumi})}
\tightlist
\item
  \href{https://en.wikipedia.org/wiki/Thales_of_Miletus}{Thales}
\end{enumerate}

He believed the world originated from water. This was considered
rational at the time and reflected the knowledge of that era, water was
everywhere. Thales is often regarded as the first philosopher and the
first to consider the problem of the one and the many. This concept is
still relevant today, as we see that everything, despite its many forms,
shares common substances.

\begin{enumerate}
\def\labelenumi{\alph{enumi})}
\setcounter{enumi}{1}
\tightlist
\item
  \href{https://en.wikipedia.org/wiki/Anaximander}{Anaximander}
\end{enumerate}

Anaximander questioned Thales' idea: If everything comes from water,
then where does water come from? He introduced the idea of the Apeiron
(the indefinite/infinite), beginning the philosophical search for a
first principle (archê).

\begin{enumerate}
\def\labelenumi{\alph{enumi})}
\setcounter{enumi}{2}
\tightlist
\item
  \href{https://en.wikipedia.org/wiki/Xenophanes}{Xenophanes}
\end{enumerate}

As humans began to question truth and divinity, Xenophanes criticized
anthropomorphic portrayals of gods. Here, monotheistic ideas began to
appear.

\begin{enumerate}
\def\labelenumi{\alph{enumi})}
\setcounter{enumi}{3}
\tightlist
\item
  \href{https://en.wikipedia.org/wiki/Heraclitus}{Heraclitus}
\end{enumerate}

Famous for the quote: ``You cannot step into the same river twice.'' He
believed everything is in constant flux, reality is constant change. He
introduced the concept of Logos as the rational structure behind the
universe, implying that nature can be understood through patterns. He
also introduced dialectical thinking: things are defined by their
opposites, a foundation for binary thinking.

\begin{enumerate}
\def\labelenumi{\alph{enumi})}
\setcounter{enumi}{4}
\tightlist
\item
  \href{https://en.wikipedia.org/wiki/Parmenides}{Parmenides}
\end{enumerate}

Opposing Heraclitus, Parmenides argued that opinions do not necessarily
reflect truth. In his view: What is, is; what is not, is not. Truth lies
in existence, while change is an illusion.

\begin{enumerate}
\def\labelenumi{\alph{enumi})}
\setcounter{enumi}{5}
\tightlist
\item
  \href{https://simple.wikipedia.org/wiki/Zeno_of_Elea}{Zeno}
\end{enumerate}

In the realm of logic, Zeno can be considered one of the first figures
to raise the issue of the relationship between logic and sensory data.
He questioned: Which should we trust, logic or what we see? One of his
most famous arguments is the Achilles and the Tortoise paradox.

In short, in the story, the swift Achilles agrees to give the slower
tortoise a head start in a race. However, Zeno argues that, logically,
Achilles will never be able to catch up with the tortoise. The reasoning
is as follows: every time Achilles reaches the point where the tortoise
previously was, the tortoise has already moved a bit farther ahead. To
reach this new point, Achilles must again travel a slightly smaller
distance, and so on, infinitely. As a result, according to Zeno's logic,
Achilles will always remain behind, even if only slightly. This clearly
contradicts our sensory experience, as in reality we can observe that
Achilles would easily overtake the tortoise. However, logically and
mathematically, Zeno's argument forms a paradox that is difficult to
refute within the boundaries of formal logic.

This paradox became one of the earliest indications of the debate about
how we should treat mathematical logic: Is syntax (formal structure)
more important than semantics (the meaning or reality we observe)? In
this way, Zeno raises an important question about the limits and role of
logic in explaining reality.

\begin{enumerate}
\def\labelenumi{\alph{enumi})}
\setcounter{enumi}{5}
\tightlist
\item
  \href{https://simple.wikipedia.org/wiki/Plato}{Plato}
\end{enumerate}

Known for distinguishing between the world of ideas and the actual
world. He believed that numbers and mathematical concepts do not exist
in the physical world, but in the world of forms (ideas).

Through the early history of the philosophical revolution, we realize
that even in the contemporary era, humans are still driven by the same
fundamental questions. a. Aren't we still asking the same things today,
about the origin of the universe? b. Aren't we still questioning what is
the source of all matter? c.~Aren't we still contemplating binary
concepts, such as only knowing that something is true because something
else is false, and vice versa?

And isn't it true that through the development of mathematical logic,
paradoxes arise not from reality itself, but from our conceptual
frameworks? The Pre-Aristotelian questions were clearly questions that
mostly belonged to the transcendental realm. Thus, when entering the
Aristotelian era, these transcendental questions began to be addressed
in a more rational manner, by developing a system of thinking focused on
what can be observed by humans.

\paragraph{2.1.2. Aristotelian era}\label{aristotelian-era}

In short, \href{https://id.wikipedia.org/wiki/Aristoteles}{Aristotle} is
regarded as the Father of Logic, primarily because he developed a system
of categories, which classifies everything. Specifically, these
categories are: Substance (Ousia), Quantity (Poson), Quality (Poion),
Relation (Pros ti), Place (Pou), Time (Pote), Position (Keisthai), State
(Echein), Action (Poiein), and Passion (Paschein).He is also well known
for his method of syllogistic reasoning, for example: All men are
mortal. Socrates is a man. Therefore, Socrates is mortal. This is one of
the most famous examples when first learning about syllogistic logic. In
addition, Aristotle is recognized as someone who developed a system of
reasoning that also aimed to understand fallacies within arguments.

\subsubsection{2.2. Modern era}\label{modern-era}

In the modern era, the development of logic can be understood in three
main points:

\begin{enumerate}
\def\labelenumi{\alph{enumi})}
\tightlist
\item
  Enlightenment
\end{enumerate}

During the Enlightenment, philosophy grew in popularity, especially in
terms of rationality, scientific methods, and freedom of thought.
Philosophers and mathematicians like
\href{https://en.wikipedia.org/wiki/Ren\%C3\%A9_Descartes}{René
Descartes},
\href{https://en.wikipedia.org/wiki/Gottfried_Wilhelm_Leibniz}{Gottfried
Wilhelm Leibniz}, and
\href{https://en.wikipedia.org/wiki/Immanuel_Kant}{Immanuel Kant} began
developing more systematic and reflective approaches to logic,
emphasizing reason as the means to acquire knowledge.

\begin{enumerate}
\def\labelenumi{\alph{enumi})}
\setcounter{enumi}{1}
\tightlist
\item
  19 Century
\end{enumerate}

``Logic'' experienced a revolution. Thinkers such as
\href{https://simple.wikipedia.org/wiki/George_Boole}{George Boole},
\href{https://en.wikipedia.org/wiki/Augustus_De_Morgan}{Augustus De
Morgan}, and \href{https://id.wikipedia.org/wiki/Gottlob_Frege}{Gottlob
Frege} developed symbolic and mathematical logic, which was far more
precise than traditional Aristotelian logic. Frege introduced predicate
logic, leading to non-classical
logic.\href{https://id.wikipedia.org/wiki/Georg_Cantor}{Georg Cantor}
founded set theory, now a foundation of modern mathematics. The dominant
school of thought: Logicism, which holds that mathematics can be reduced
to logic.

\begin{enumerate}
\def\labelenumi{\alph{enumi})}
\setcounter{enumi}{2}
\tightlist
\item
  20 Century
\end{enumerate}

``Logic'' continued to grow and was applied in mathematics, linguistics,
and computer science. This era saw the emergence of modal logic and its
branches, such as:

\begin{enumerate}
\def\labelenumi{\arabic{enumi})}
\tightlist
\item
  Alethic logic\\
\item
  Deontic logic\\
\item
  Epistemic logic\\
\item
  Doxastic logic
\item
  Temporal logic
\item
  Dynamic logic
\item
  Action logic
\item
  Intuitionistic logic
\item
  Multi-modal logic
\item
  Provability logic
\end{enumerate}

Meanwhile, mathematics faced a foundational crisis, initiated by
\href{https://id.wikipedia.org/wiki/Kurt_G\%C3\%B6del}{Kurt Gödel}.
Where the 19th century saw Logicism, the 20th century saw new ``isms'':

\begin{enumerate}
\def\labelenumi{\arabic{enumi})}
\tightlist
\item
  Formalism: Logic is symbol manipulation based on formal rules.
\item
  Intuitionism: Mathematical truths are mental constructions.
\end{enumerate}
