Truth Value for FOL

In propositional logic (PL), every atomic sentence is directly assigned
a truth value (true or false). On the other hand, in first-order logic
(FOL), the truth value of a predicate is determined relative to an
interpretation and a variable assignment.

For example:

Let domain \(D = \{1,2,3\}\), for \(E(x)\) = ``\(x\) is even''.

Hence:

− \(E(1) =\) False

− \(E(2) =\) True

− \(E(3) =\) False

\subsubsection{1. Example 1}\label{example-1}

Let the domain: \[D = \{\langle x,y \rangle \in \mathbb{N}^2\}\]

Evaluate: \[\forall x,y  \in D \;(x \cdot y > x + y)\]

However, by antisymmetric schema, we can find a counterexample:

\[
\exists x \exists y \;\big((\langle x, y \rangle \in \mathbb{N}^2) \;\land\; x = y \;\land\; x = 2 \;\land\; y = 2 \;\land\; (x \cdot y = x + y) \;\land\; \neg(x \cdot y > x + y)\big)
\]

Therefore: \[\boxed{\neg \forall x,y \in D \;(x \cdot y > x + y)}\]

Note that other counterexamples include \(\langle 1,1 \rangle\),
\(\langle 1,2 \rangle\), \(\langle 2,1 \rangle\), and
\(\langle 1,n \rangle\) for any \(n \in \mathbb{N}\).

\subsubsection{2. Example 2}\label{example-2}

Let the domain: \[D = \mathbb{Z}\]

Evaluate:

\[\forall x \forall y \forall z \;(x, y, z \in \mathbb{Z} \to ((x + y = z) \to z > 0))\]

However, we can find a counterexample:

\[\exists x \exists y \exists z \;(x, y, z \in \mathbb{Z} \;\land\; x = -1 \;\land\; y = -2 \;\land\; z = -3 \;\land\; (x + y = z) \;\land\; \neg(z > 0))\]

Since \(-3 \not> 0\), we have shown that there exist integers where the
sum equals \(z\) but \(z\) is not positive.

Consequently:

\[\boxed{\neg \forall x \forall y \forall z \;((x + y = z) \to z > 0)}\]

\subsubsection{3. Example 3}\label{example-3}

Let the domain:

\[
D = \displaystyle\left\{ x \in \mathbb{Z} \;\middle|\; \frac{3x}{2} + 1 + \frac{4x}{3} = -\frac{31}{8} + \frac{2x}{3} \right\}.
\]

Evaluate:

\[
\exists x \,(x \in D).
\]

Since \(-\tfrac{9}{4} \notin \mathbb{Z}\),

Thus:

\[
\boxed{D = \varnothing}
\]

And the statement is false, because the domain is empty.

\subsubsection{4. Example 4}\label{example-4}

Let the domain:
\[D = \{\langle x,y \rangle \in \mathbb{R}^2 \mid x + 2y = 16\} \]

Evaluate:

\[ \forall  x, y \in D \, ((x \in \mathbb{R} \land y \in \mathbb{Z}) \to x > y) \]

Since we can find a counterexample:

\[\exists x \exists y \;(x \in \mathbb{R} \;\land\; y \in \mathbb{Z} \;\land\; x = 0 \;\land\; y = 8 \;\land\; (x + 2y = 16) \;\land\; \neg(x > y))\]

Therefore:

\[\boxed{\neg \forall x, y \in D \, ((x \in \mathbb{R} \land y \in \mathbb{Z}) \to x > y)}\]

\subsubsection{5. Example 5}\label{example-5}

Let the domain: \[D = \{\langle x,y \rangle \in \mathbb{Z}^2\}\]

Evaluate:
\[\forall \langle x,y \rangle \in D \;(\text{even}(x) \leftrightarrow \text{even}(y))\]

However, we can find a counterexample:

\[
\exists x \exists y \;\big((\langle x, y \rangle \in \mathbb{Z}^2) \;\land\; x = 4 \;\land\; y = 3 \;\land\; \text{even}(x) \;\land\; \neg\text{even}(y) \;\land\; \neg(\text{even}(x) \leftrightarrow \text{even}(y))\big)
\]

Therefore:

\[\boxed{\neg \forall \langle x,y \rangle \in D \;(\text{even}(x) \leftrightarrow \text{even}(y))}\]

Note that other counterexamples include \(\langle 2,5 \rangle\),
\(\langle 7,8 \rangle\), and any pair where one number is even and the
other is odd.
