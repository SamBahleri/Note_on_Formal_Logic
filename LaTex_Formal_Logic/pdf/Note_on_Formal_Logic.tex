\documentclass[12pt,a4paper]{book}

\usepackage{amsmath,amssymb}
\usepackage{geometry}
\geometry{margin=1in}
\usepackage{graphicx}
\usepackage{xcolor}
\usepackage{titlesec}
\usepackage{enumitem}

% Hyperref tanpa kotak merah
\usepackage[hidelinks]{hyperref}

% Nomor tanpa leading zero untuk section
\renewcommand{\thesection}{\arabic{section}}
\renewcommand{\thesubsection}{\thesection.\arabic{subsection}}
\renewcommand{\thesubsubsection}{\thesubsection.\arabic{subsubsection}}

% Nomor chapter dengan angka Romawi
\renewcommand{\thechapter}{\Roman{chapter}}

\title{Note on Formal Logic}
\author{Samena Bahleri}
\date{\today}

\begin{document}

\frontmatter
\maketitle
\tableofcontents

\mainmatter

% ===== PART I =====
\chapter{Introduction}

\section{Etymology and Terminology}\label{etymology-and-terminology}
Normally, in the study of logic, the first thing we need to understand
is the question: what is logic in terms of its definition?
Etymologically, the word logic comes from the Greek word (logos).
The word logos carries various meanings, including ``word,'' ``speech,''
``reason,'' ``explanation,'' and ``principle.'' Over time, this term was
adopted into Latin as \emph{logica}, which means the art or science of
reasoning. On the other hand, in terms of terminology, logic is the
systematic study of valid inference and correct reasoning. Therefore, it
can be understood that when we study logic, what we are learning is how
to evaluate things rationally and systematically to reach a solid
understanding and draw a sound conclusion. One striking statement about
logic comes from the philosopher John Locke, who said that ``logic is
the anatomy of thought.'' In this metaphorical statement, we can also
understand that by studying logic, we are essentially learning about the
structure of thought itself.

\section{Logic in History}\label{logic-in-history}

\subsection{Philosophical revolution}\label{philosophical-revolution}
In this era, humans began to think not only about how to live in their
environment, but also about themselves, truth, and ideas. One figure
whose statement represents this revolution is
\href{https://en.wikipedia.org/wiki/Socrates}{Socrates}, with his famous
quote: ``The unexamined life is not worth living.''. To understand the
systematic development of logic, we can follow the stages of history
below:

\paragraph{2.1.1. Pre-Aristotelian Era}\label{pre-aristotelian-era}\

In this era, humans tried to understand the world through myth,
narratives shaped by imagination and traditional constructions. Thinkers
of this era include:

\begin{enumerate}

\item \href{https://en.wikipedia.org/wiki/Thales_of_Miletus}{Thales}
  
He believed the world originated from water. This was considered rational
at the time and reflected the knowledge of that era. Thales is often
regarded as the first philosopher and the first to consider the problem
of the one and the many.

\item \href{https://en.wikipedia.org/wiki/Anaximander}{Anaximander}
  
He questioned Thales' idea: If everything comes from water, then where
does water come from? He introduced the idea of the \emph{Apeiron} (the
indefinite/infinite), beginning the philosophical search for a first
principle (\emph{archê}).

\item \href{https://en.wikipedia.org/wiki/Xenophanes}{Xenophanes} 

As humans began to question truth and divinity, Xenophanes criticized
anthropomorphic portrayals of gods. Monotheistic ideas began to appear.

\item \href{https://en.wikipedia.org/wiki/Heraclitus}{Heraclitus}
  
Famous for the quote: ``You cannot step into the same river twice.'' He
believed everything is in constant flux, reality is constant change. He
introduced the concept of Logos as the rational structure behind the
universe, implying that nature can be understood through patterns.

\item \href{https://en.wikipedia.org/wiki/Parmenides}{Parmenides}

Opposing Heraclitus, Parmenides argued that opinions do not necessarily
reflect truth. In his view: What is, is; what is not, is not. Truth lies
in existence, while change is an illusion.

\item \href{https://simple.wikipedia.org/wiki/Zeno_of_Elea}{Zeno} 
 
He raised questions about the relationship between logic and sensory
data. One of his most famous arguments is the Achilles and the Tortoise
paradox.
\end{enumerate}

\paragraph{2.1.2. Aristotelian era}\label{aristotelian-era}\

\href{https://id.wikipedia.org/wiki/Aristoteles}{Aristotle} is regarded
as the Father of Logic. He developed a system of categories, e.g.,
Substance, Quantity, Quality, Relation, Place, Time, Position, State,
Action, Passion. He also developed syllogistic reasoning:  
\emph{All men are mortal. Socrates is a man. Therefore, Socrates is
mortal.} This illustrates the basics of formal logic.

\subsection{Modern era}\label{modern-era}
In the modern era, logic evolved in three main phases:

\begin{enumerate}
\item Enlightenment

During the Enlightenment, philosophy grew in popularity, especially in
terms of rationality, scientific methods, and freedom of thought.
Philosophers and mathematicians like
\href{https://en.wikipedia.org/wiki/Ren\%C3\%A9_Descartes}{René
Descartes},
\href{https://en.wikipedia.org/wiki/Gottfried_Wilhelm_Leibniz}{Gottfried
Wilhelm Leibniz}, and
\href{https://en.wikipedia.org/wiki/Immanuel_Kant}{Immanuel Kant} began
developing more systematic and reflective approaches to logic,
emphasizing reason as the means to acquire knowledge.

\item 19 Century

``Logic'' experienced a revolution. Thinkers such as
\href{https://simple.wikipedia.org/wiki/George_Boole}{George Boole},
\href{https://en.wikipedia.org/wiki/Augustus_De_Morgan}{Augustus De
Morgan}, and \href{https://id.wikipedia.org/wiki/Gottlob_Frege}{Gottlob
Frege} developed symbolic and mathematical logic, which was far more
precise than traditional Aristotelian logic. Frege introduced predicate
logic, leading to non-classical
logic.\href{https://id.wikipedia.org/wiki/Georg_Cantor}{Georg Cantor}
founded set theory, now a foundation of modern mathematics. The dominant
school of thought: Logicism, which holds that mathematics can be reduced
to logic.

\item 20 Century

``Logic'' continued to grow and was applied in mathematics, linguistics,
and computer science. This era saw the emergence of modal logic and its
branches, such as:

\begin{enumerate}
\item Alethic logic
\item Deontic logic
\item Epistemic logic
\item Doxastic logic
\item Temporal logic
\item Dynamic logic
\item Action logic
\item Intuitionistic logic
\item Multi-modal logic
\item Provability logic
\end{enumerate}

Meanwhile, mathematics faced a foundational crisis, initiated by \href{https://id.wikipedia.org/wiki/Kurt_G\%C3\%B6del}{Kurt Gödel}.
Where the 19th century saw Logicism, the 20th century saw new ``isms'':

\begin{enumerate}
\item
  Formalism: Logic is symbol manipulation based on formal rules.
\item
  Intuitionism: Mathematical truths are mental constructions.
\end{enumerate}
\end{enumerate}

\chapter{Arguments}
\section{Definition of Arguments}\label{definition-of-arguments}


\end{document}
