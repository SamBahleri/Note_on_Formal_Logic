Natural Deduction Rules

After comprehending the rules of inference, we will have a better grasp
of what natural deduction is. The most well-known idea of natural
deduction is the work of
\href{https://en.wikipedia.org/wiki/Frederic_Fitch}{Frederic Fitch}. He
created a system known as the \emph{Fitch-style calculus}. With this
system, we can mechanically derive propositions to reach a valid
conclusion, based on the initial premises or statements.

\subsection{1. Conjunction}\label{conjunction}

\begin{enumerate}
\def\labelenumi{\alph{enumi}.}
\tightlist
\item
  Conjunction Introduction
\end{enumerate}

\begin{verbatim}
<img src="/2_Logic_Blog/Branch_of_Logic/Formal_Logic/Natural_Deduction/1_And_Intro.png" alt="Diagram" class="responsive-image">
<div class="image-caption">
    Picture 1:  Conjunction Introduction
</div>
\end{verbatim}

\begin{enumerate}
\def\labelenumi{\alph{enumi}.}
\setcounter{enumi}{1}
\tightlist
\item
  Conjunction Elimination
\end{enumerate}

\begin{verbatim}
<img src="/2_Logic_Blog/Branch_of_Logic/Formal_Logic/Natural_Deduction/1_And_Elim.png" alt="Diagram" class="responsive-image">
<div class="image-caption">
    Picture 2:  Conjunction Elimination
</div>
\end{verbatim}

\subsection{2. Disjunction}\label{disjunction}

\begin{enumerate}
\def\labelenumi{\alph{enumi}.}
\tightlist
\item
  Disjunction Introduction
\end{enumerate}

\begin{verbatim}
<img src="/2_Logic_Blog/Branch_of_Logic/Formal_Logic/Natural_Deduction/2_Or_Intro.png" alt="Diagram" class="responsive-image">
<div class="image-caption">
    Picture 3:  Disjunction Introduction
</div>
\end{verbatim}

\begin{enumerate}
\def\labelenumi{\alph{enumi}.}
\setcounter{enumi}{1}
\tightlist
\item
  Disjunction Elimination
\end{enumerate}

\begin{verbatim}
<img src="/2_Logic_Blog/Branch_of_Logic/Formal_Logic/Natural_Deduction/2_Or_Elim.png" alt="Diagram" class="responsive-image">
<div class="image-caption">
    Picture 4:  Disjunction Elimination
</div>
\end{verbatim}

\subsection{3. Negation}\label{negation}

\begin{enumerate}
\def\labelenumi{\alph{enumi}.}
\tightlist
\item
  Negation Introduction
\end{enumerate}

\begin{verbatim}
<img src="/2_Logic_Blog/Branch_of_Logic/Formal_Logic/Natural_Deduction/3_Neg_intro.png" alt="Diagram" class="responsive-image">
<div class="image-caption">
    Picture 5:  Negation Introduction
</div>
\end{verbatim}

\begin{enumerate}
\def\labelenumi{\alph{enumi}.}
\setcounter{enumi}{1}
\tightlist
\item
  Negation Elimination
\end{enumerate}

\begin{verbatim}
<img src="/2_Logic_Blog/Branch_of_Logic/Formal_Logic/Natural_Deduction/3_Neg_Elim.png" alt="Diagram" class="responsive-image">
<div class="image-caption">
    Picture 6:  Negation Elimination
</div>
\end{verbatim}

\subsection{4. Implication}\label{implication}

\begin{enumerate}
\def\labelenumi{\alph{enumi}.}
\tightlist
\item
  Implication Introduction
\end{enumerate}

\begin{verbatim}
<img src="/2_Logic_Blog/Branch_of_Logic/Formal_Logic/Natural_Deduction/4_Imp_Intro.png" alt="Diagram" class="responsive-image">
<div class="image-caption">
    Picture 7:  Implication Introduction
</div>
\end{verbatim}

\begin{enumerate}
\def\labelenumi{\alph{enumi}.}
\setcounter{enumi}{1}
\tightlist
\item
  Implication Elimination
\end{enumerate}

\begin{verbatim}
<img src="/2_Logic_Blog/Branch_of_Logic/Formal_Logic/Natural_Deduction/4_Imp_Elim.png" alt="Diagram" class="responsive-image">
<div class="image-caption">
    Picture 8:  Implication Elimination
</div>
\end{verbatim}

\subsection{5. Bottom}\label{bottom}

\begin{verbatim}
<img src="/2_Logic_Blog/Branch_of_Logic/Formal_Logic/Natural_Deduction/5_Bot.png" alt="Diagram" class="responsive-image">
<div class="image-caption">
    Picture 9:  Bottom
</div>
\end{verbatim}

\subsection{\texorpdfstring{5. Reiteration
(\(R\))}{5. Reiteration (R)}}\label{reiteration-r}

\begin{verbatim}
<img src="/2_Logic_Blog/Branch_of_Logic/Formal_Logic/Natural_Deduction/6_R.png" alt="Diagram" class="responsive-image">
<div class="image-caption">
    Picture 10:  Reiteration
</div>
\end{verbatim}

\subsection{\texorpdfstring{7. Reductio Ad Absurdum
(\(RAA\))}{7. Reductio Ad Absurdum (RAA)}}\label{reductio-ad-absurdum-raa}

\begin{verbatim}
<img src="/2_Logic_Blog/Branch_of_Logic/Formal_Logic/Natural_Deduction/7_RAA.png" alt="Diagram" class="responsive-image">
<div class="image-caption">
    Picture 11:  RAA
</div>
\end{verbatim}

\subsection{8. Biconditional}\label{biconditional}

\begin{enumerate}
\def\labelenumi{\alph{enumi}.}
\tightlist
\item
  Biconditional Introduction
\end{enumerate}

\begin{verbatim}
<img src="/2_Logic_Blog/Branch_of_Logic/Formal_Logic/Natural_Deduction/8_Bicon_Intro.png" alt="Diagram" class="responsive-image">
<div class="image-caption">
    Picture 12:  Biconditional Introduction
</div>
\end{verbatim}

\begin{enumerate}
\def\labelenumi{\alph{enumi}.}
\setcounter{enumi}{1}
\tightlist
\item
  Biconditional Elimination
\end{enumerate}

\begin{verbatim}
<img src="/2_Logic_Blog/Branch_of_Logic/Formal_Logic/Natural_Deduction/8_Bicon_elim.png" alt="Diagram" class="responsive-image">
<div class="image-caption">
    Picture 13:  Biconditional Elimination
</div>
\end{verbatim}

\subsection{\texorpdfstring{9. Double Negation
(\(DN\))}{9. Double Negation (DN)}}\label{double-negation-dn}

\begin{enumerate}
\def\labelenumi{\alph{enumi}.}
\tightlist
\item
  Double Negation Introduction
\end{enumerate}

\begin{verbatim}
<img src="/2_Logic_Blog/Branch_of_Logic/Formal_Logic/Natural_Deduction/9_DN_Intro.png" alt="Diagram" class="responsive-image">
<div class="image-caption">
    Picture 14  Biconditional Introduction
</div>
\end{verbatim}

\begin{enumerate}
\def\labelenumi{\alph{enumi}.}
\setcounter{enumi}{1}
\tightlist
\item
  Double Negation Elimination
\end{enumerate}

\begin{verbatim}
<img src="/2_Logic_Blog/Branch_of_Logic/Formal_Logic/Natural_Deduction/9_DN_Elim.png" alt="Diagram" class="responsive-image">
<div class="image-caption">
    Picture 15:  Biconditional Elimination
</div>
\end{verbatim}

\subsection{\texorpdfstring{10. De Morgan's Laws
(\(DM\))}{10. De Morgan's Laws (DM)}}\label{de-morgans-laws-dm}

\[\neg (A\land B) \equiv \neg A \lor \neg B\]

\begin{verbatim}
<img src="/2_Logic_Blog/Branch_of_Logic/Formal_Logic/Natural_Deduction/10_DM.png" alt="Diagram" class="responsive-image">
<div class="image-caption">
    Picture 16:  De Morgan’s Laws
</div>
\end{verbatim}

\[\neg (A\lor B) \equiv \neg A \land \neg B\]

\begin{verbatim}
<img src="/2_Logic_Blog/Branch_of_Logic/Formal_Logic/Natural_Deduction/10_DM2.png" alt="Diagram" class="responsive-image">
<div class="image-caption">
    Picture 17:  De Morgan’s Laws
</div>
\end{verbatim}

\subsection{11. Material Conditional}\label{material-conditional}

\[ A \rightarrow B \equiv \neg A \lor B\]

\begin{verbatim}
<img src="/2_Logic_Blog/Branch_of_Logic/Formal_Logic/Natural_Deduction/11_Material.png" alt="Diagram" class="responsive-image">
<div class="image-caption">
    Picture 18:  Material Conditional
</div>
\end{verbatim}

\[ \neg A \lor B  \equiv A \rightarrow B\]

\begin{verbatim}
<img src="/2_Logic_Blog/Branch_of_Logic/Formal_Logic/Natural_Deduction/11_Material2.png" alt="Diagram" class="responsive-image">
<div class="image-caption">
    Picture 19:  Material Conditional
</div>
\end{verbatim}

\subsection{\texorpdfstring{13. Conditional Proof
(\(CP\))}{13. Conditional Proof (CP)}}\label{conditional-proof-cp}

\begin{verbatim}
<img src="/2_Logic_Blog/Branch_of_Logic/Formal_Logic/Natural_Deduction/13_Cp_Intro.png" alt="Diagram" class="responsive-image">
<div class="image-caption">
    Picture 22:  Conditional Proof
</div>
\end{verbatim}

\subsection{12. Contrapositive}\label{contrapositive}

\[ A \rightarrow B \equiv \neg B \rightarrow \neg A\]

\begin{verbatim}
 Picture 20:  Contrapositive
</div>
\end{verbatim}

\[ \neg B \rightarrow \neg A \equiv  A \rightarrow B \]

\begin{verbatim}
    Picture 21:  Contrapositive
\end{verbatim}
