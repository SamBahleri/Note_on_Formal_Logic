Semantic concepts

In the previous chapter, we learned about symbols such as \(p\), \(q\),
\(\neg\), \(\land\), \(\lor\), \(\to\), and \(\leftrightarrow\), as well
as the syntax rules that define well-formed formulas (wff), for example,
\(a \lor b\). In this chapter, we will explore semantics, which deals
with the meaning and interpretation of these formulas.

\subsubsection{1. Tautology}\label{tautology}

A formula is a tautology if it is always true, regardless of the truth
values of its components. For example, \(a \lor \neg a\) is a tautology;
it remains true whether \(a\) is \emph{True} or \emph{False}. Another
example is \(a \to a\), which is always \emph{True} by the definition of
implication. The symbol for tautology is usually denoted by \(\top\).
Hence, we can write tautologies such as:

\[
a \lor \neg a \quad (\top) \quad \text{and} \quad a \to a \quad (\top)
\]

Truth Tables:

\(a\)

\(\neg a\)

\(a \lor \neg a\)

\(a \to a\)

1

0

1

1

0

1

1

1

\subsubsection{2. Contradiction}\label{contradiction}

A contradiction, frequently denoted by \(\bot\), is a formula that is
always \emph{False}, no matter the truth values assigned. For example,
\(a \land \neg a\) is a contradiction because it can never be true.
Another example is \((a \land \neg a) \lor (b \land \neg b)\), which is
also always \emph{False}.

Truth Table:

\(a\)

\(b\)

\(\neg a\)

\(\neg b\)

\(a \land \neg a\)

\(b \land \neg b\)

\((a \land \neg a) \lor (b \land \neg b)\)

1

1

0

0

0

0

0

1

0

0

1

0

0

0

0

1

1

0

0

0

0

0

0

1

1

0

0

0

\subsubsection{3. Equivalence}\label{equivalence}

In propositional logic, equivalence
\((\equiv \text{ or } \leftrightarrow)\) between two well-formed
formulas (wffs) means that they have the same truth value for all
possible truth assignments to their propositions. If two formulas are
logically equivalent, they will always evaluate to the same truth value,
regardless of the truth values of the individual propositions.

For example, the formula \(a \land b\) is equivalent to
\(\neg(\neg a \lor \neg b)\).

Truth Table:

\(a\)

\(b\)

\(\neg a\)

\(\neg b\)

\(\neg a \lor \neg b\)

\(\neg(\neg a \lor \neg b)\)

\(a \land b\)

1

1

0

0

0

1

1

1

0

0

1

1

0

0

0

1

1

0

1

0

0

0

0

1

1

1

0

0

Common Forms of Logical Equivalence

\begin{enumerate}
\def\labelenumi{\arabic{enumi}.}
\item
  Double Negation\\
  \[\neg(\neg a) \equiv a\]
\item
  De Morgan's Laws\\
  \[\neg(a \land b) \equiv \neg a \lor \neg b\]\\
  \[\neg(a \lor b) \equiv \neg a \land \neg b\]
\item
  Commutative Laws\\
  \[a \land b \equiv b \land a\]\\
  \[a \lor b \equiv b \lor a\]
\item
  Associative Laws\\
  \[(a \land b) \land r \equiv a \land (b \land r)\]\\
  \[(a \lor b) \lor r \equiv a \lor (b \lor r)\]
\item
  Distributive Laws\\
  \[a \land (b \lor r) \equiv (a \land b) \lor (a \land r)\]\\
  \[a \lor (b \land r) \equiv (a \lor b) \land (a \lor r)\]
\item
  Implication Equivalences\\
  \[a \to b \equiv \neg a \lor b\]\\
  \[\neg(a \to b) \equiv a \land \neg b\]
\item
  Biconditional (Equivalence)\\
  \[a \leftrightarrow b \equiv (a \to b) \land (b \to a)\]\\
  \[a \leftrightarrow b \equiv (a \land b) \lor (\neg a \land \neg b)\]
\item
  Contrapositive\\
  \[a \to b \equiv \neg b \to \neg a\]
\end{enumerate}

\subsubsection{4. Contingency}\label{contingency}

A well-formed formula (wff) is called a contingency if it is \emph{True}
in some cases and \emph{False} in others. In other words, it is neither
a tautology nor a contradiction. For instance, \(a \lor b\) is a
contingency; it can be \emph{True} or \emph{False} depending on the
values of \(a\) and \(b\).

Truth Table:

\(a\)

\(b\)

\(a \lor b\)

0

0

0

0

1

1

1

0

1

1

1

1

\subsubsection{5. Satisfiability}\label{satisfiability}

Satisfiability refers to whether there exists \emph{at least one}
assignment of truth values to the variables in a formula that makes the
formula \emph{True}. A formula is satisfiable if it can be true for some
assignment of its variables; otherwise, it is unsatisfiable. For
example, the formula \(a \land b\) is satisfiable because there is at
least one assignment (when both \(a\) and \(b\) are \emph{True}) that
makes the formula true.

\begin{longtable}[]{@{}llll@{}}
\toprule\noalign{}
\(a\) & \(b\) & \(a \land b\) & \(Satisfies?\) \\
\midrule\noalign{}
\endhead
\bottomrule\noalign{}
\endlastfoot
0 & 0 & 0 & \(No\) \\
0 & 1 & 0 & \(No\) \\
1 & 0 & 0 & \(No\) \\
1 & 1 & 1 & \(Yes\) \\
\end{longtable}

\subsubsection{6. Validity}\label{validity}

An argument is called \emph{valid} if its conclusion necessarily follows
from its premises, no matter whether the premises are true or false in
the actual world. That is, in every case where the premises are true,
the conclusion must also be true.

For example, consider the statement:

\[
\neg a \to \neg b
\]

Truth Table:

\(a\)

\(b\)

\(\neg a\)

\(\neg b\)

\(\neg a \to \neg b\)

0

0

1

1

1

0

1

1

0

0

1

0

0

1

1

1

1

0

0

1

In the last row, both \(\neg a\) and \(\neg b\) evaluate to
\emph{False}. Since the antecedent (\(\neg a\)) is \emph{False}, the
entire implication \(\neg a \to \neg b\) evaluates to \emph{True}
according to the truth-table definition of implication. This truth
arises because an implication with a \emph{False} antecedent is always
\emph{True}, regardless of the consequent.

\subsubsection{7. Soundness}\label{soundness}

An argument is sound if: 1. \(\Gamma \vdash A\) (syntactic validity -
provable) 2. \(\Gamma \models A\) (semantic validity - tautological
consequence) 3. All premises in \(\Gamma\) are true

Soundness Theorem:

If \(\Gamma \vdash A\), then \(\Gamma \models A\)

Equivalently: 1. If \(\Gamma\) is satisfiable, then \(\Gamma\) is
consistent 2. Provability implies tautological consequence

Example

Given premises \(\Gamma = \{a, b\}\) and conclusion \(c\):

\[(a \land b) \to c\]

where:

− \(a\): all birds have feathers\\
− \(b\): a robin is a bird\\
− \(c\): a robin has feathers

Truth table:

\begin{longtable}[]{@{}lllll@{}}
\toprule\noalign{}
\(a\) & \(b\) & \(c\) & \(a \land b\) & \((a \land b) \to c\) \\
\midrule\noalign{}
\endhead
\bottomrule\noalign{}
\endlastfoot
1 & 1 & 1 & 1 & 1 \\
1 & 1 & 0 & 1 & 0 \\
1 & 0 & 1 & 0 & 1 \\
1 & 0 & 0 & 0 & 1 \\
0 & 1 & 1 & 0 & 1 \\
0 & 1 & 0 & 0 & 1 \\
0 & 0 & 1 & 0 & 1 \\
0 & 0 & 0 & 0 & 1 \\
\end{longtable}

In detail: 1. \(\{a, b\} \vdash c\) (provable from premises) 2.
\(\{a, b\} \models c\) (tautological consequence) 3.
\(\varphi(a) = 1, \varphi(b) = 1, \varphi(c) = 1\) (all true in reality)
4. Therefore: the argument is sound

\subsubsection{8. Entailment}\label{entailment}

If validity refers to reasoning that is structurally correct but not
necessarily factually accurate, and soundness refers to reasoning that
is both valid and factually true, then \emph{entailment} is a logical
relationship where the truth of one statement necessarily guarantees the
truth of another within a formal system. In other words, entailment
shows that a conclusion must be true if the premises are true, based
solely on the rules of logic.

For example:

\[ a \land b \models a \]

This means that the statement \(a \land b\) (both \(a\) and \(b\) are
true) \emph{entails} \(a\) (that \(a\) is true).

Truth Table:

\(a\)

\(b\)

\(a \land b\)

\(a\)

1

1

1

1

1

0

0

1

0

1

0

0

0

0

0

0

Notice that in every case where \(a \land b\) is true (the first row),
the conclusion \(a\) is also true. This illustrates the entailment:
\(a \land b\) logically entails \(a\).

\subsubsection{9. Completeness}\label{completeness}

Definition

A system is complete if

\[
\Gamma \models A \to \Gamma \vdash A
\]

Equivalently:

\begin{enumerate}
\def\labelenumi{\arabic{enumi}.}
\tightlist
\item
  \(\text{Cons}(\Gamma) \to \text{Sat}(\Gamma)\)\\
\item
  \(\models \text{ then } \vdash\)
\end{enumerate}

Completeness Theorem

\[
\text{If } \Gamma \models A, \text{ then } \Gamma \vdash A
\]

Example

Let

\[
\Gamma = \{a \to b, \; a\}, \quad A = b
\]

Then

\[
\Gamma \models b
\]

By (MP):

\[
\frac{a \to b, \quad a}{b}
\]

Hence

\[
\Gamma \vdash b
\]
