Argument

In our daily lives, we undoubtedly use logic to analyze things, whether
it's someone's statement or a natural phenomenon we observe. Then, in
analyzing something, we naturally hope that our analysis is correct, or
that we can identify something was wrong, avoid it, or correct it so
that it becomes right. In this section, we will specifically learn about
arguments: what an argument is, the components that make up an argument,
and what makes an argument valid or invalid. Studying arguments is
certainly beneficial, whether in the social, academic, or interpersonal
realm.

\subsection{1. Component}\label{component}

\subsubsection{1.1. Statement}\label{statement}

A statement or proposition is a declarative sentence that has a definite
truth value; it can be either true or false. A mathematical statement is
one example of a statement.

Example 1:

\[
\begin{align*}
1 &.\ \displaystyle\bigl[2<4 \ \wedge \ 4<7 \ \to \ 2<7 \bigr]
& 6 &.\ \left[\frac{5}{6}\right]^7 = \neg \neg \left[\frac{5^7}{6^7}\right] \\[2mm]
2 &.\ s =2 \to s^3 \cdot s^5 = s^8
& 7 &.\ \left[6^{1/2}\right]\left[6^{3/2}\right] \lor \neg \neg 6^2 \\[2mm]
3 &.\ a = 3 \to \sqrt[a]{8^2} \equiv 8^{2/3}
& 8 &.\ \log(4 \cdot 5) \equiv \log 4 + \log 5 \\[2mm]
4 &.\ \left[4 \cdot 5\right]^6 \equiv 4^6 \cdot 5^6
& 9 &.\ m = 3, e = 2 \to (m+e)^2 \equiv m^2 + 2(me) + e^2 \\[2mm]
5 &.\ \left[4^{5/1}\right]^6 \lor 4^{30}
& 10 &.\ n = 5, a = 2 \to (n-a)^2 \equiv n^2 - 2(na) + a^2 \\[4mm]
\end{align*}
\]

All of the examples above are mathematical statements with truth values
that can be verified.

Example 2:

\begin{enumerate}
\def\labelenumi{\arabic{enumi}.}
\tightlist
\item
  \$ \text{All prime numbers are odd}\$\\
\item
  \$\sqrt{a+b} = \sqrt{a} + \sqrt{b} \$
\item
  \({}^{9}\!log\,81 > {}^{2}\!log\,64 - {}^{2}\!log\,16\)
\end{enumerate}

The second set of examples are false statements. However, since each can
be evaluated to a definite conclusion, Example 2 is also a statement.

\subsubsection{1.2. Non-Statement}\label{non-statement}

Consequently, any expression that does not have a definite truth value
is not considered a statement.

For example:

\begin{enumerate}
\def\labelenumi{\arabic{enumi}.}
\item
  ``What time is it?''
\item
  ``Close the door.''
\item
  ``Wow, that's amazing!''
\item
  ``Who is running?''
\end{enumerate}

\subsubsection{1.3. Open sentence}\label{open-sentence}

We also need to understand that an argument can contain open sentences,
statements whose truth value is not yet determined.

For example:

\begin{enumerate}
\def\labelenumi{\arabic{enumi}.}
\item
  \$ x \text{ is a prime number}\$
\item
  \$ \forall x, y, z (x, y, z \in \mathbb{Z} \mid x +y = z)\$
\item
  \$ n\^{}2 + 1\$ is divisible by \(3\)
\item
  \$ p \textgreater{} 10\$
\end{enumerate}

Until we specify what \(p\), \(x\), \(z\), or \(y\) refer to, the truth
cannot be judged. Once the variable is assigned, the open sentence
becomes a proposition (a definite true/false statement). This
distinction is central in predicate logic.

\subsubsection{1.4. Conclusion}\label{conclusion}

Intuitively, we all understand what a conclusion is. A conclusion is the
result derived from a set of premises, or in this case, when premises
present facts, the conclusion is what we can logically derive from those
facts.

Example 1:

\$

\begin{aligned}
\text{Premise 1:} \ & \text{All humans are living beings} \\
\text{Premise 2:} \ & \text{Socrates is a human} \\
\text{Conclusion:} \ & \text{Therefore, Socrates is a living being}
\end{aligned}

\$

As we can observe, a conclusion is a deduction drawn from the facts. And
in this example, the conclusion is clearly correct.

Example 2:

\$

\begin{aligned}
\text{Premise 1:} \ & 2, 3 \in \mathbb{R} \\
\text{Premise 2:} \ & 2 < 3 \\
\text{Conclusion:} \ & \text{Hence, } 2 = 3
\end{aligned}

\$

In this example, it is true that 2 and 3 are elements of the real
numbers, and it is also true that 2 is less than 3. However, the
conclusion drawn 2 = 3 is false because it does not logically follow
from the premises.

\subsubsection{1.5. Inference}\label{inference}

Inference is not the same as a conclusion. A conclusion is the specific
claim that logically follows from the premises, whereas an inference
refers to the entire reasoning process that connects the premises to the
conclusion.

For example:

\$

\begin{aligned}
\text{Premise 1:} \ & \text{All students must take the final exam} \\
\text{Premise 2:} \ & \text{Elica is a student} \\
\text{Conclusion:} \ & \text{Therefore, Elica must take the final exam}
\end{aligned}

\$

Here, the conclusion is simply:\\
\[\text{“Elica must take the final exam.”}\]

The inference is the reasoning as a whole:

\[\text{“Because all students must take the final exam, and Elica is a student, it follows that Elica must take the final exam.”}\]

\subsection{2. Formal and Informal}\label{formal-and-informal}

When analyzing an argument, we can approach it from two perspectives:

\subsubsection{2.1. Formal Logic}\label{formal-logic}

As we will continue to study in this course, formal logic is the
discipline that focuses on the form of the argument itself, regardless
of the content or meaning of the statements. For example:

Recall:

\$

\begin{aligned}
\text{Premise 1:} \ & \text{All humans are living beings} \\
\text{Premise 2:} \ & \text{Socrates is a human}
\end{aligned}

\$

Previously, we drew a conclusion by reading the content of these
premises (statements in natural language). In formal logic, however, we
represent these premises using symbols. For instance, we can represent
the entire natural language statement with symbols like \(p, q, r, s\)
and so on.

For example:

− \(p\): All humans are living beings

− \(q\): Socrates is a human

From these two premises, we can combine them using the conjunction
symbol (\$ \land \$), resulting in: \(p \land q\) . If we translate this
back into natural language, it becomes: ``All humans are living beings
and Socrates is a human.''

\subsubsection{2.2. Informal Logic}\label{informal-logic}

If formal logic focuses on the form of the premises, informal logic
focuses on the content of those premises. We've already seen examples of
this above.

Recall:

\$

\begin{aligned}
\text{Premise 1:} \ & \text{All humans are living beings} \\
\text{Premise 2:} \ & \text{Socrates is a human} \\
\text{Conclusion:} \ & \text{Therefore, Socrates is a living being}
\end{aligned}

\$

The focus of informal logic is:

\begin{enumerate}
\def\labelenumi{\arabic{enumi}.}
\item
  Are the premises reasonable?
\item
  Are the terms used clear?
\item
  Who is Socrates?
\end{enumerate}

\subsubsection{2.3. Formal vs Informal}\label{formal-vs-informal}

It is important to understand that formal and informal logic each have
their own limitations. Specifically, formal logic allows us to draw
conclusions systematically and with certainty, because the rules of
inference have already been clearly defined. In this kind of reasoning,
we do not focus on the content of the statements, we focus solely on the
process of drawing conclusions.

For example:

\[ \text{If } A \text{ is true and } B \text{ is true, then } A \land B \text{ (A and B) is true.}\]

In this case, the conclusion is valid because the rule of conjunction
says so. However, we do not know what \$ A \$ or \$ B \$ actually
represent. We don't evaluate whether \$ A \$ or \$ B \$ are reasonable
or relevant in real-life contexts. On the other hand, informal logic is
more contextual, as it reflects what is actually happening in the real
world. However, it is more difficult to evaluate objectively and is more
vulnerable to bias or logical fallacies. Simply put, informal logic
represents a form of reasoning that we often encounter in everyday life.

For example:

\(\text{“Climate change is real because almost every scientist agrees on it, and we can see unusual weather patterns happening more frequently.”}\)

In this case, the reasoning isn't framed in formal symbols like \(𝐴\)
and \(𝐵\). Instead, it relies on contextual evidence (scientific
consensus, observed weather) and appeals to what is happening in the
real world. While it can be persuasive, it is also open to subjective
interpretation, selective use of evidence, or possible fallacies (e.g.,
appeal to authority, hasty generalization).

\subsection{3. Syllogism}\label{syllogism}

As we have seen above, an argument is a collection of statements, and
its conclusion, whether true or false, depends not only on the truth
value of the statements, but also on whether the conclusion logically
follows from the premises. Below are some examples of drawing
conclusions from two premises and a conclusion, presented in both formal
notation and natural language, which we understand as the process of
syllogism.

\begin{enumerate}
\def\labelenumi{\arabic{enumi}.}
\item
  Barbara (AAA)

  \$

  \begin{aligned}
   \text{Premise 1:} \ & \text{All } M \text{ are } P \\
   \text{Premise 2:} \ & \text{All } S \text{ are } M \\
   \text{Conclusion:} \ & \text{Therefore, All } S \text{ are } P
   \end{aligned}

  \$

  Example:

  \$

  \begin{aligned}
   \text{Premise 1:} \ & \text{All mammals are warm-blooded} \\
   \text{Premise 2:} \ & \text{All dogs are mammals} \\
   \text{Conclusion:} \ & \text{Therefore, All dogs are warm-blooded}
   \end{aligned}

  \$
\item
  Celarent (EAE)

  \$

  \begin{aligned}
   \text{Premise 1:} \ & \text{No } M \text{ are } P \\
   \text{Premise 2:} \ & \text{All } S \text{ are } M \\
   \text{Conclusion:} \ & \text{Therefore, No } S \text{ are } P
   \end{aligned}

  \$

  Example:

  \$

  \begin{aligned}
   \text{Premise 1:} \ & \text{No reptiles are warm-blooded} \\
   \text{Premise 2:} \ & \text{All snakes are reptiles} \\
   \text{Conclusion:} \ & \text{Therefore, No snakes are warm-blooded}
   \end{aligned}

  \$
\item
  Darii (AII)

  \$

  \begin{aligned}
   \text{Premise 1:} \ & \text{All } M \text{ are } P \\
   \text{Premise 2:} \ & \text{Some } S \text{ are } M \\
   \text{Conclusion:} \ & \text{Therefore, Some } S \text{ are } P
   \end{aligned}

  \$

  Example:

  \$

  \begin{aligned}
   \text{Premise 1:} \ & \text{All books are sources of knowledge} \\
   \text{Premise 2:} \ & \text{Some objects are books} \\
   \text{Conclusion:} \ & \text{Therefore, Some objects are sources of knowledge}
   \end{aligned}

  \$
\item
  Ferio (EIO)

  \$

  \begin{aligned}
   \text{Premise 1:} \ & \text{No } M \text{ are } P \\
   \text{Premise 2:} \ & \text{Some } S \text{ are } M \\
   \text{Conclusion:} \ & \text{Therefore, Some } S \text{ are not } P
   \end{aligned}

  \$

  Example:

  \$

  \begin{aligned}
   \text{Premise 1:} \ & \text{No cats are reptiles} \\
   \text{Premise 2:} \ & \text{Some pets are cats} \\
   \text{Conclusion:} \ & \text{Therefore, Some pets are not reptiles}
   \end{aligned}

  \$
\item
  Cesare (EAE)

  \$

  \begin{aligned}
   \text{Premise 1:} \ & \text{No } P \text{ are } M \\
   \text{Premise 2:} \ & \text{All } S \text{ are } M \\
   \text{Conclusion:} \ & \text{Therefore, No } S \text{ are } P
   \end{aligned}

  \$

  Example:

  \$

  \begin{aligned}
   \text{Premise 1:} \ & \text{No reptiles are mammals} \\
   \text{Premise 2:} \ & \text{All snakes are reptiles} \\
   \text{Conclusion:} \ & \text{Therefore, No snakes are mammals}
   \end{aligned}

  \$
\item
  Camestres (AEE)

  \$

  \begin{aligned}
   \text{Premise 1:} \ & \text{All } P \text{ are } M \\
   \text{Premise 2:} \ & \text{No } S \text{ are } M \\
   \text{Conclusion:} \ & \text{Therefore, No } S \text{ are } P
   \end{aligned}

  \$

  Example:

  \$

  \begin{aligned}
   \text{Premise 1:} \ & \text{All birds are animals} \\
   \text{Premise 2:} \ & \text{No insects are animals} \\
   \text{Conclusion:} \ & \text{Therefore, No insects are birds}
   \end{aligned}

  \$
\item
  Festino (EIO)

  \$

  \begin{aligned}
   \text{Premise 1:} \ & \text{No } M \text{ are } P \\
   \text{Premise 2:} \ & \text{Some } S \text{ are } M \\
   \text{Conclusion:} \ & \text{Therefore, Some } S \text{ are not } P
   \end{aligned}

  \$

  Example:

  \$

  \begin{aligned}
   \text{Premise 1:} \ & \text{No fish are mammals} \\
   \text{Premise 2:} \ & \text{Some pets are fish} \\
   \text{Conclusion:} \ & \text{Therefore, Some pets are not mammals}
   \end{aligned}

  \$
\item
  Baroco (AOO)

  \$

  \begin{aligned}
   \text{Premise 1:} \ & \text{All } P \text{ are } M \\
   \text{Premise 2:} \ & \text{Some } S \text{ are not } M \\
   \text{Conclusion:} \ & \text{Therefore, Some } S \text{ are not } P
   \end{aligned}

  \$

  Example:

  \$

  \begin{aligned}
   \text{Premise 1:} \ & \text{All animals are living beings} \\
   \text{Premise 2:} \ & \text{Some creatures are not animals} \\
   \text{Conclusion:} \ & \text{Therefore, Some creatures are not living beings}
   \end{aligned}

  \$
\item
  Bocardo (OAO)

  \$

  \begin{aligned}
   \text{Premise 1:} \ & \text{Some } M \text{ are not } P \\
   \text{Premise 2:} \ & \text{All } M \text{ are } S \\
   \text{Conclusion:} \ & \text{Therefore, Some } S \text{ are not } P
   \end{aligned}

  \$

  Example:

  \$

  \begin{aligned}
   \text{Premise 1:} \ & \text{Some books are not fiction} \\
   \text{Premise 2:} \ & \text{All books are readable materials} \\
   \text{Conclusion:} \ & \text{Therefore, Some readable materials are not fiction}
   \end{aligned}

  \$
\item
  Camenes (AEE)

  \$

  \begin{aligned}
  \text{Premise 1:} \ & \text{All } P \text{ are } M \\
  \text{Premise 2:} \ & \text{No } S \text{ are } M \\
  \text{Conclusion:} \ & \text{Therefore, No } S \text{ are } P
  \end{aligned}

  \$

  Example:

  \$

  \begin{aligned}
  \text{Premise 1:} \ & \text{All birds are animals} \\
  \text{Premise 2:} \ & \text{No reptiles are animals} \\
  \text{Conclusion:} \ & \text{Therefore, No reptiles are birds}
  \end{aligned}

  \$
\item
  Dimaris (IAI)

  \$

  \begin{aligned}
  \text{Premise 1:} \ & \text{Some } M \text{ are } P \\
  \text{Premise 2:} \ & \text{All } S \text{ are } M \\
  \text{Conclusion:} \ & \text{Therefore, Some } S \text{ are } P
  \end{aligned}

  \$

  Example:

  \$

  \begin{aligned}
  \text{Premise 1:} \ & \text{Some fruits are sweet} \\
  \text{Premise 2:} \ & \text{All apples are fruits} \\
  \text{Conclusion:} \ & \text{Therefore, Some apples are sweet}
  \end{aligned}

  \$
\item
  Felapton (EIO)

  \$

  \begin{aligned}
  \text{Premise 1:} \ & \text{No } M \text{ are } P \\
  \text{Premise 2:} \ & \text{All } M \text{ are } S \\
  \text{Conclusion:} \ & \text{Therefore, Some } S \text{ are not } P
  \end{aligned}

  \$

  Example:

  \$

  \begin{aligned}
  \text{Premise 1:} \ & \text{No dogs are reptiles} \\
  \text{Premise 2:} \ & \text{All dogs are animals} \\
  \text{Conclusion:} \ & \text{Therefore, Some animals are not reptiles}
  \end{aligned}

  \$
\item
  Ferison (EIO)

  \$

  \begin{aligned}
  \text{Premise 1:} \ & \text{No } M \text{ are } P \\
  \text{Premise 2:} \ & \text{Some } M \text{ are } S \\
  \text{Conclusion:} \ & \text{Therefore, Some } S \text{ are not } P
  \end{aligned}

  \$

  Example:

  \$

  \begin{aligned}
  \text{Premise 1:} \ & \text{No reptiles are mammals} \\
  \text{Premise 2:} \ & \text{Some reptiles are snakes} \\
  \text{Conclusion:} \ & \text{Therefore, Some snakes are not mammals}
  \end{aligned}

  \$
\item
  Disamis (IAI)

  \$

  \begin{aligned}
  \text{Premise 1:} \ & \text{Some } M \text{ are } P \\
  \text{Premise 2:} \ & \text{All } M \text{ are } S \\
  \text{Conclusion:} \ & \text{Therefore, Some } S \text{ are } P
  \end{aligned}

  \$

  Example:

  \$

  \begin{aligned}
  \text{Premise 1:} \ & \text{Some fruits are sweet} \\
  \text{Premise 2:} \ & \text{All fruits are consumable items} \\
  \text{Conclusion:} \ & \text{Therefore, Some consumable items are sweet}
  \end{aligned}

  \$
\item
  Datisi (AII)

  \$

  \begin{aligned}
  \text{Premise 1:} \ & \text{All } M \text{ are } P \\
  \text{Premise 2:} \ & \text{Some } M \text{ are } S \\
  \text{Conclusion:} \ & \text{Therefore, Some } S \text{ are } P
  \end{aligned}

  \$

  Example:

  \$

  \begin{aligned}
  \text{Premise 1:} \ & \text{All books are sources of knowledge} \\
  \text{Premise 2:} \ & \text{Some books are novels} \\
  \text{Conclusion:} \ & \text{Therefore, Some novels are sources of knowledge}
  \end{aligned}

  \$
\item
  Bramantip (AAI)

  \$

  \begin{aligned}
  \text{Premise 1:} \ & \text{All } P \text{ are } M \\
  \text{Premise 2:} \ & \text{All } M \text{ are } S \\
  \text{Conclusion:} \ & \text{Therefore, Some } S \text{ are } P
  \end{aligned}

  \$

  Example:

  \$

  \begin{aligned}
  \text{Premise 1:} \ & \text{All birds are animals} \\
  \text{Premise 2:} \ & \text{All animals are living beings} \\
  \text{Conclusion:} \ & \text{Therefore, Some living beings are birds}
  \end{aligned}

  \$
\item
  Fesapo (EAO)

  \$

  \begin{aligned}
  \text{Premise 1:} \ & \text{No } M \text{ are } P \\
  \text{Premise 2:} \ & \text{All } M \text{ are } S \\
  \text{Conclusion:} \ & \text{Therefore, Some } S \text{ are not } P
  \end{aligned}

  \$

  Example:

  \$

  \begin{aligned}
  \text{Premise 1:} \ & \text{No cats are reptiles} \\
  \text{Premise 2:} \ & \text{All cats are pets} \\
  \text{Conclusion:} \ & \text{Therefore, Some pets are not reptiles}
  \end{aligned}

  \$
\item
  Fresison (EIO)

  \$

  \begin{aligned}
  \text{Premise 1:} \ & \text{No } M \text{ are } P \\
  \text{Premise 2:} \ & \text{Some } M \text{ are } S \\
  \text{Conclusion:} \ & \text{Therefore, Some } S \text{ are not } P
  \end{aligned}

  \$

  Example:

  \$

  \begin{aligned}
  \text{Premise 1:} \ & \text{No mammals are fish} \\
  \text{Premise 2:} \ & \text{Some mammals are pets} \\
  \text{Conclusion:} \ & \text{Therefore, Some pets are not fish}
  \end{aligned}

  \$
\end{enumerate}
