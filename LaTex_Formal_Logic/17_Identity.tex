Identity and Quantity

\subsection{1. Identity}\label{identity}

Simply put, in logic, something is considered identical if \(a\) is the
same as \(b\), symbolically written as \(a = b\). However, we must be
careful in understanding these two constants. Specifically, the term
``identical`` here does not refer to two objects that are very similar,
but rather to one and the same object. Empirically, because two
different objects can never be truly identical. Identity does not speak
of similarity, but of sameness, that \(a\) and \(b\) are not merely
alike, but actually refer to the exact same entity. Consider the classic
example from the philosophy of language, ``Morning Star'' and ``Evening
Star'', these are both names that refer to the same object, Venus.
Another example, consider this expression, \(\frac{1}{2} = 0.5\),
logically we agree that this expression is same.

For example, Consider the ``Morning Star'' and ``Evening Star'', we can
symbolize these statements like this:

Let:

− \(m\) = the Morning Star

− \(e\) = the Evening Star

Then the identity claim is:

\[ m = e \]

This states that both names refer to one and the same object.

If we treat them as predicates:

− \(MS(x)\) = ``\(x\) is the Morning Star''

− \(ES(x)\) = ``\(x\) is the Evening Star''

Then we can express:

\[\exists x \, (MS(x) \land ES(x))\]

Meaning: \emph{there exists an object \(x\) such that \(x\) is both the
Morning Star and the Evening Star}.

Moreover, identity is not just a symbol but follows specific rules that
govern how it works.

For example:

\begin{enumerate}
\def\labelenumi{\arabic{enumi}.}
\item
  Reflexivity

  Every object is identical to itself. This is written as

  \[\forall x (x = x)\]

  which expresses the obvious fact that no matter what object we choose,
  it is always equal to itself.

  For example, let \(P(x)\) mean ``\(x\) is a prime number.'' If
  \(x = 7\), then it must also be true that \(7 = 7\). In predicate
  form, we can write this as

  \[\forall x \, (x = 7 \to x = x)\]

  And if we combine reflexivity with a property, we get

  \[\forall x \, (x = 7 \to (P(x) \leftrightarrow P(7)))\]

  This means that if \(x = 7\), then whatever property \(P\) applies to
  \(x\) must also apply to \(7\). For instance, if \(P(x)\) means
  ``\(x\) is prime,'' then since \(x = 7\), \(P(x)\) holds if and only
  if \(P(7)\) holds.
\item
  Leibniz's Law Leibniz's Law, also called the principle of
  substitutivity of identicals. This states that if \(x = y\), then
  whatever is true of \(x\) is also true of \(y\), symbolized as
  \[x = y \rightarrow (P(x) \leftrightarrow P(y))\] If two names refer
  to the same object, then they are interchangeable in any true
  statement. For example, if \(m = e\) (Morning Star = Evening Star),
  and the statement ``\(m\) is visible at dawn'' is true, then it must
  also be true that ``\(e\) is visible at dawn.''
\end{enumerate}

\subsection{2. Quantity}\label{quantity}

In everyday language, we often encounter vagueness, even ambiguity, when
interpreting someone's statements. For example, consider the sentence:
\emph{``All humans love some dogs.''} The word \emph{some} here is
vague, because it is unclear how many dogs are being referred to, or
whether all humans love the same dogs or different ones. This ambiguity
can lead to different interpretations. For instance, the sentence can be
understood in two distinct ways:

All humans love at least one dog. In predicate logic, this is written as

\[
\forall x \, (Human(x) \rightarrow \exists y \, (Dog(y) \land Loves(x,y)))
\]

This means: every human loves at least one dog, though the specific dog
may differ from person to person.

There is some dog that all humans love. In predicate logic, this is
written as

\[
\exists y \, (Dog(y) \land \forall x \, (Human(x) \rightarrow Loves(x,y)))
\]

This means: there exists a particular dog that is loved by all humans.

To avoid ambiguity, we can express the specific quantity of objects we
are talking about. Predicate logic allows us to distinguish between
\emph{at least}, \emph{at most}, and \emph{exactly} a certain number of
objects satisfying a condition \(A(x)\).

\begin{enumerate}
\def\labelenumi{\arabic{enumi}.}
\item
  At least (minimum number). This means there are at least \(n\)
  different objects that satisfy \(A(x)\), possibly more.

  \begin{enumerate}
  \def\labelenumii{\alph{enumii})}
  \tightlist
  \item
    At least one:
  \end{enumerate}

  \[
   \exists x \, A(x)
   \]

  This means: \emph{``There exists at least one \(x\) such that \(A(x)\)
  is true.''}

  For example:

  Let \(A(x)\) = ``\(x\) is a prime number.'' Then the statement says:
  \emph{``There is at least one prime number.''} This is true, since
  \(x = 2\) works.

  \begin{enumerate}
  \def\labelenumii{\alph{enumii})}
  \setcounter{enumii}{1}
  \tightlist
  \item
    At least two:
  \end{enumerate}

  \[
   \exists x \exists y \, (A(x) \land A(y) \land x \neq y)
   \]

  This means: \emph{``There exist at least two different objects such
  that \(A(x)\) holds.''}

  For example:

  Let \(A(x)\) = ``\(x\) is a prime number.'' Then the statement says:
  \emph{``There are at least two prime numbers.''} This is true, because
  \(x = 2\) and \(y = 3\) both satisfy \(A(x)\), and \(2 \neq 3\).

  \begin{enumerate}
  \def\labelenumii{\alph{enumii})}
  \setcounter{enumii}{2}
  \tightlist
  \item
    At least three:
  \end{enumerate}

  \[
   \exists x \exists y \exists z \, (A(x) \land A(y) \land A(z) \land x \neq y \land x \neq z \land y \neq z)
   \]

  This means: \emph{``There exist at least three different objects such
  that \(A(x)\) holds.''}

  For example: Let \(A(x)\) = ``\(x\) is a prime number.'' Then the
  statement says: \emph{``There are at least three prime numbers.''}
  This is true, because \(x = 2\), \(y = 3\), and \(z = 5\) all satisfy
  \(A(x)\), and they are pairwise distinct.
\item
  At most (maximum number). This means there are no more than \(n\)
  different objects satisfying \(A(x)\).

  \begin{enumerate}
  \def\labelenumii{\alph{enumii})}
  \tightlist
  \item
    At most one:
  \end{enumerate}

  \[
   \forall x \forall y \, ((A(x) \land A(y)) \rightarrow x = y)
   \]

  This means: \emph{``If \(x\) and \(y\) both satisfy \(A\), then they
  must be the same object.''}

  For example:

  Let \(A(x)\) = ``\(x\) is the even prime number.'' Then the statement
  says: \emph{``There is at most one even prime.''} This is true, since
  \(2\) is the only even prime, and no two distinct numbers can both
  satisfy \(A(x)\).

  \begin{enumerate}
  \def\labelenumii{\alph{enumii})}
  \setcounter{enumii}{1}
  \tightlist
  \item
    At most two:
  \end{enumerate}

  \[
   \forall x \forall y \forall z \, ((A(x) \land A(y) \land A(z)) \rightarrow (x = y \lor x = z \lor y = z))
   \]

  This means: \emph{``If \(x\), \(y\), and \(z\) all satisfy \(A\), then
  at least two of them are the same.''} So there cannot be three
  distinct objects satisfying \(A(x)\).

  For example:

  Let \(A(x)\) = ``\(x\) is a square root of \(4\).'' Then the statement
  says: \emph{``There are at most two square roots of \(4\).''} This is
  true, because the only solutions are \(x = 2\) and \(x = -2\).

  \begin{enumerate}
  \def\labelenumii{\alph{enumii})}
  \setcounter{enumii}{2}
  \tightlist
  \item
    At most three:
  \end{enumerate}

  \[
   \forall x \forall y \forall z \forall w \, ((A(x) \land A(y) \land A(z) \land A(w)) \rightarrow (x = y \lor x = z \lor x = w \lor y = z \lor y = w \lor z = w))
   \]

  This means: \emph{``If \(x\), \(y\), \(z\), and \(w\) all satisfy
  \(A\), then at least two of them must be the same.''}\\
  So there cannot be four distinct objects satisfying \(A(x)\).

  For example:

  Let \(A(x)\) = ``\(x\) is a primary color of light.'' Then the
  statement says: \emph{``There are at most three primary colors of
  light.''} This is true, since the only options are red, green, and
  blue.
\item
  Exactly (precise count). This means there are exactly \(n\) distinct
  objects satisfying \(A(x)\), not more, not less.

  \begin{enumerate}
  \def\labelenumii{\alph{enumii})}
  \tightlist
  \item
    Exactly one:
  \end{enumerate}

  \[
   \exists x \, (A(x) \land \forall y (A(y) \rightarrow y = x))
   \]

  This means: \emph{``There exists exactly one object such that \(A(x)\)
  holds.''}

  For example:

  Let \(A(x)\) = ``\(x\) is the even prime number.'' Then the statement
  says: \emph{``There is exactly one even prime number.''} This is true,
  since \(2\) is the only even prime.

  \begin{enumerate}
  \def\labelenumii{\alph{enumii})}
  \setcounter{enumii}{1}
  \tightlist
  \item
    Exactly two:
  \end{enumerate}

  \[
   \exists x \exists y \, (A(x) \land A(y) \land x \neq y \land \forall z (A(z) \rightarrow (z = x \lor z = y)))
   \]

  This means: \emph{``There exist exactly two distinct objects such that
  \(A(x)\) holds.''}

  For example:

  Let \(A(x)\) = ``\(x\) is a square root of \(4\).'' Then the statement
  says: \emph{``There are exactly two square roots of \(4\).''} This is
  true, since the solutions are \(2\) and \(-2\), and no others.

  \begin{enumerate}
  \def\labelenumii{\alph{enumii})}
  \setcounter{enumii}{2}
  \tightlist
  \item
    Exactly three:
  \end{enumerate}

  \[
   \exists x \exists y \exists z \, (A(x) \land A(y) \land A(z) \land x \neq y \land y \neq z \land x \neq z \land \forall w (A(w) \rightarrow (w = x \lor w = y \lor w = z)))
   \]

  This means: \emph{``There exist exactly three distinct objects such
  that \(A(x)\) holds.''}

  For example:

  Let \(A(x)\) = ``\(x\) is a primary color of light.'' Then the
  statement says: \emph{``There are exactly three primary colors of
  light.''} This is true, since the primary colors are red, green, and
  blue, no more, no less.
\end{enumerate}
