Multi Place Predicates

As we have seen in the previous chapter, a formula like this is called a
\emph{One-Place}:

\[\forall x((A(x) \rightarrow B(x)))\]

However, not every mathematical statement deals with only one variable
and predicate. That's why we use \emph{Multi-Place Predicates}. With
multi-place predicates, we can create more expressive formulas. For
example:

\subsubsection{1. Two-Place Predicates}\label{two-place-predicates}

\[R(x,y)\]

Expresses a relation between two objects.

\begin{enumerate}
\def\labelenumi{\arabic{enumi}.}
\item
  If there exists a number greater than \(x\), then there exists a
  number greater than \(x+1\)
  \[\forall x \, (\exists y \, G(y,x) \rightarrow \exists z \, G(z,x+1))\]
\item
  If there exists a number \(y\) equal to \(x\), then \(x\) equals \(y\)
  \[\forall x \, (\exists y \, E(x,y) \rightarrow E(y,x))\]
\item
  If \(x=y\) and there exists \(z\) such that \(y=z\), then \(x=z\)
  \[\forall x \forall y \, ((E(x,y) \land \exists z \, E(y,z)) \rightarrow \exists z \, E(x,z))\]
\item
  If \(x\) divides \(y\), and there exists \(z\) divisible by \(y\),
  then \(x\) divides \(z\)
  \[\forall x \forall y \, (D(x,y) \land \exists z \, D(y,z) \rightarrow \exists z \, D(x,z))\]
\end{enumerate}

\subsubsection{2. Three-Place Predicates}\label{three-place-predicates}

\[R(x,y,z)\]

Expresses a relation among three objects.

\begin{enumerate}
\def\labelenumi{\arabic{enumi}.}
\item
  For every number \(x\), there exists a number \(y\) such that \(x\) is
  less than \(y\). \[\forall x \, \exists y \, L(x,y)\]
\item
  For all numbers \(p\), if \(p\) is prime, then there exists a number
  \(q\) such that \(q\) = \(p + 2\).
  \[\forall p \, (\mathbb{P}(p) \rightarrow \exists q \, (q = p+2))\]
\item
  For all numbers \(n\), if \(n\) is even, then there exists a number
  \(m\) such that \(m = n/2\).
  \[\forall n \, (\text{Even}(n) \rightarrow \exists m \, (m = n/2))\]
\item
  For all \(x\), if \(x\) is a natural number, then there exists a \(y\)
  in the integers such that \(x\) is not equal to \(y\).
  \[\forall x \, (x \in \mathbb{N} \rightarrow \exists y \, (y \in \mathbb{Z} \land x \neq y))\]
\end{enumerate}

\subsubsection{3. Four-Place Predicates}\label{four-place-predicates}

\[Q(w,x,y,z)\]

Expresses a relation among four objects.

\begin{enumerate}
\def\labelenumi{\arabic{enumi}.}
\item
  For every \(a\) and \(b\), there exists a \(c\) such that \(a+b+c\) is
  even
  \[\forall a \forall b \, \exists c \exists d \, (Q(a,b,c,d) \land \text{Even}(d))\]
\item
  If \(a,b,c\) are positive, then there exists \(d>0\) such that
  \(a+b+c=d\)
  \[\forall a \forall b \forall c \, ((a>0 \land b>0 \land c>0) \rightarrow \exists d \, Q(a,b,c,d) \land d>0)\]
\item
  For every \(x,y\), there exists \(z\) such that \(x \times y + z\) is
  even
  \[\forall x \forall y \, \exists z \exists w \, (P(x,y,z,w) \land \text{Even}(w))\]
\item
  If \(x,y,z>0\), then there exists \(w>0\) such that
  \(x \times y + z = w\)
  \[\forall x \forall y \forall z \, ((x>0 \land y>0 \land z>0) \rightarrow \exists w \, P(x,y,z,w) \land w>0)\]
\end{enumerate}
