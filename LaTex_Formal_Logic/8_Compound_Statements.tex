Compound Statements

~~~~~~ As we have learned in the Well-Formed Formula part, by using
connectives such as \(\neg\) , \(\land\), \(\lor\), \(\rightarrow\),
\(\leftrightarrow\), we can create more complex combinations. Then, we
can also calculate their truth values. For example:

\subsubsection{1. Example 1}\label{example-1}

Let

\[(a \lor b) \land c\]

Where:

− \(a\): \(2\) is prime

− \(b\): \(2\) is even

− \(c\): \(2\) is divisible by some even number

Then, we want to check every possible truth value for \(a\), \(b\), and
\(c\) using a truth table.

First, we evaluate the \((a \lor b)\) expression:

Since \(a\) is \emph{True} and \(b\) is also \emph{True}, the output for
\((a \lor b)\) is \emph{True}, or \((a \lor b) = 1\).

Next, we check the value of \(c\). Since it is true that \(2\) is
divisible by some even number, the value of \(c\) is also \emph{True}.
Hence, \(c = 1\).

Finally, by combining \((a \lor b)\) and \(c\) with the conjunction
operator (\(\land\)), we know that when both inputs are \emph{True}, the
output will also be \emph{True}.

\begin{longtable}[]{@{}lllll@{}}
\toprule\noalign{}
\(a\) & \(b\) & \(c\) & \((a \lor b)\) & \((a \lor b) \land c\) \\
\midrule\noalign{}
\endhead
\bottomrule\noalign{}
\endlastfoot
1 & 1 & 1 & 1 & 1 \\
1 & 1 & 0 & 1 & 0 \\
1 & 0 & 1 & 1 & 1 \\
1 & 0 & 0 & 1 & 0 \\
0 & 1 & 1 & 1 & 1 \\
0 & 1 & 0 & 1 & 0 \\
0 & 0 & 1 & 0 & 0 \\
0 & 0 & 0 & 0 & 0 \\
\end{longtable}

~~~~~~ Furthermore, because we already know that \(2\) is prime, even,
and divisible by some even number, the output is obviously \emph{True}
for the \((a \lor b) \land c\) combination. Moreover, since the
expression is already known to be \emph{True}, we do not need to
consider the rest of the possible values in the truth table; in this
case, we only take the first row.

\subsubsection{2. Example 2}\label{example-2}

Let

\[ 
\begin{align*}
a &= \displaystyle\left[ x = 3, y = 5 \to 2^x \cdot 2^y = 2^5 \right]\\
b &= \left[p = 8, q = 6 \to \frac{2^p}{q^q} < \frac{2^p}{3^9}\right]\\
c &= \left[\left[3^{1/2}\right]^2 \cdot \left[2^{1/2}\right]^2 = 2^2+2 \right] 
\end{align*}\]

We ask for

\[(a \to b) \land c\]

since

\[ a = 1, \quad b= 1, \quad c= 1\]

We conclude that

\[\boxed{((a \to b) \land c )= True}\]

\subsubsection{3. Example 2}\label{example-2-1}

Let

\[
\begin{align*}
d &= \left[p=4 \;\to\; 2^p \cdot 2^5 = 2^9 \right] &
f &= \left[\frac{2^4 \cdot 3^6}{2^3 \cdot 3^2}\right]^3 = 2^3 \cdot 3^{12} \\[1mm]
e &= \left[p=3 \;\to\; (3\pi)^p = 27\pi^3 \right] &
g &= \left[\frac{1^{-1}}{5^{-6}} \right] = 5^6
\end{align*}
\]

We ask for

\[
((d \leftrightarrow e) \lor f) \land g
\]

Since

\[
d = 1, \quad e = 1, \quad f = 1 , \quad g = 1
\]

Therefore

\[
\boxed{(((d \leftrightarrow e) \lor f) \land g )= True}
\]

\subsubsection{4. Example 4}\label{example-4}

Let

\[
\begin{align*}
h &= \displaystyle\left[ x=2, \; y=5 \;\to\; \left[\frac{8 \cdot x^3 \cdot y^{-4}}{16 \cdot y^{-1/4}}\right]\right] 
      \;\to\; \neg \neg \left[\frac{4}{\sqrt[4]{5^{15}}}\right] \\[2mm]
i &= \displaystyle\left[\frac{2^5 \cdot 3^{-10}}{2^5 \cdot 3^{-4}}\right]^{1/2}
      \left[\frac{4^{1/4} \cdot 5^{1/2}} {4^{1/2} \cdot 5^{-1/2}}\right]^{1/2} \to \sqrt{\frac{5}{3^6 \cdot \sqrt{2}}}\\
j &= \displaystyle\left[p = 3, q = 3 \;\to\; \left[ 5 \sqrt{2^p} \cdot 3 \sqrt[q]{3} 
      \leftrightarrow 15 \cdot 2 \sqrt{2} \cdot \sqrt[3]{3} \right]\right]
\end{align*}
\]

We ask for

\[h \lor \neg \neg (i \land j)\]

Since

\[
h = 1, \quad i = 1, \quad f = 1 , \quad j = 1
\]

Hence

\[
\boxed{(h \lor \neg \neg (i \land j)) = True}
\]

\subsubsection{5. Example 5}\label{example-5}

Let

\[ \begin{align*}
k &= \sqrt{2} + \sqrt{6} < \sqrt{15}\\
l &= \sqrt{8} + \sqrt{10} < \sqrt{40}\\
m &= \sqrt{5} + \sqrt{7} \ge \sqrt{26}\\
\end{align*}
\]

We ask for

\[(k \land l) \to m\]

since

\[
k = 1, \quad l = 1, \quad m = 0 ,
\]

Thus

\[
\boxed{(k \land l) \to m = False}
\]
