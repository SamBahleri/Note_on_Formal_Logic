Definite Descriptions

\subsection{1. Russell's analysis}\label{russells-analysis}

The idea of definite descriptions was introduced by
\href{https://en.wikipedia.org/wiki/Bertrand_Russell}{Bertrand Russell}
(1905) in his classic paper \emph{On Denoting}. A definite description
is a phrase of the form \emph{``the F''}, intended to pick out a
\emph{unique object} with property \(F\).

Russell analyzed such phrases not as names, but as logical
constructions:

\[
\text{“The } F \text{ is G"}\;\;\equiv\;\; \exists x \, \big( F(x) \land \forall y \, (F(y) \to y = x) \land G(x) \big)
\]

Russell's aim was to handle cases where a description refers to
something that does not actually exist. For example, consider the
chapter 2, about the mathematical statement, that clerly we can evaluate
the value whether its True or False:

\(− 2 < 5\)

− ``The smallest prime number is \(2\)'':

\[
\exists! x \, \big( (Prime(x) \land \forall y \, (Prime(y) \land y < x \;\to\; y = x)) \;\land\; x = 2 \big)
\]

Clearly, ``number'' does not exist in physical reality. To see this more
clearly, consider the following statement:

\[
\text{Agent A: “The Eiffel Tower exists.”}
\]

When we are in discussion with someone who claims that such a tower
exists, the natural response is to ask further questions about its
\emph{location}. By verifying its position, we can reach agreement that
the tower truly exists in reality. Similarly, we can examine the claim
about the number \(2\): that \(2 < 5\) and that \(2\) is the smallest
prime number. The question is: does \(2\) exist in the same way as the
Eiffel Tower? Can we point to its location, or provide map coordinates
to prove the position of ``number \(2\)'' in the world? It is obvious
that such a number does not exist empirically. Numbers are not objects
in space, but abstract entities whose existence is logical rather than
physical.

In this case, Russell's analysis is adequate to treat \emph{non-existent
objects}. It allows us to symbolize descriptions like \emph{``the
smallest prime number''} or even \emph{``the current king of France''}
without collapsing into nonsense. However, the real problem lies in the
\emph{conclusion}: should we pay attention at all to things that do not
exist?

After all, expressions such as:

\[
\forall x \, A(x)
\]

are perfectly valid in terms of \emph{syntax}. But in the end, the
crucial question is what \(x\) is meant to represent. In other words,
when we evaluate a claim, we are not only concerned with the syntactic
form, but also with its \emph{semantic content}.Russell's analysis
handles the logical structure, but the question of existence pulls us
back toward interpretation and meaning.

For example, recall Russell's formalization:

\[
\text{“The } F \text{ is G"}\;\;\equiv\;\; \exists x \, \big( F(x) \land \forall y \, (F(y) \to y = x) \land G(x) \big)
\]

Nevertheless, suppose that there are no \(F\)s in the domain. On
Russell's analysis, both \emph{``the \(F\) is \(G\)''} and \emph{``the
\(F\) is non-\(G\)''} turn out to be false.

Therefore, The negation of Russell's analysis is:

\[
\lnot \exists x \, \big( F(x) \land \forall y (F(y) \to y = x) \land G(x) \big)
\]

Which is equivalent to:

\[
\forall x \, \lnot \big( F(x) \land \forall y (F(y) \to y = x) \land G(x) \big)
\]

\subsection{2. P. F. Strawson's Critic}\label{p.-f.-strawsons-critic}

In response to this tricky expression,
\href{https://en.wikipedia.org/wiki/P._F._Strawson}{P. F. Strawson}
suggested that such sentences should not be regarded as \emph{false}
exactly, but rather as involving \emph{presupposition failure}, and thus
need to be treated as neither true nor false.

For example, consider the sentence:

\[\text{“The present king of France is bald.”}\]

Russell's analysis would say this is simply false, since there is no
such king. But Strawson argued that the problem is different, the
sentence \emph{presupposes} that there \emph{is} a present king of
France. Since that presupposition fails, the sentence is not properly
true or false at all.

Another example:

\[\text{“The smallest prime greater than 10 is even.”}\]

This presupposes that there exists a \emph{smallest prime greater than
10}. But since there are infinitely many primes, no such smallest prime
exists. So, according to Strawson, the sentence is neither true nor
false, but suffers from presupposition failure.

\subsection{3. Keith Donnellan's Idea}\label{keith-donnellans-idea}

If P. F. Strawson focuses only on intuitions, then there is room to
disagree with his idea. For example, how can we ever be completely
certain about what we see or mean in context? In response to this kind
of concern, \href{https://en.wikipedia.org/wiki/Keith_Donnellan}{Keith
Donnellan} introduces the role of intention, and offers a famous example
involving two men drinking:

Two men stand in the corner:

− A very tall man drinking what looks like a gin martini.

− A very short man drinking what looks like a pint of water.

Seeing them, Malika says: ``The gin-drinker is very tall!''

Now suppose that the very tall man is actually drinking water from a
martini glass; whereas the very short man is drinking a pint of neat
gin. By Russell's analysis, Malika has said something false. But don't
we want to say that Malika has said something true?

We can agree that, from Malika's point of view, she intended to pick out
a particular man and say something true of him (that he was tall). On
Russell's analysis, however, she actually picked out a different man
(the short one) and consequently said something false about him.

So, should we judge Malika's sentence as true because her intention was
to refer to the tall man? Or should we judge it as false because her
logical description did not match the facts?
