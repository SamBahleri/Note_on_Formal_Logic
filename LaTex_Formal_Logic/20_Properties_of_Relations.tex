Properties of Relations

Let \(R \subseteq A^2\) be a binary relation on a set \(A\):

\begin{enumerate}
\def\labelenumi{\arabic{enumi}.}
\item
  Reflexivity:
  \[R \subseteq A^2 \leftrightarrow \forall x \, (A(x) \to xRx)\]
\item
  Irreflexivity:
  \[R \subseteq A^2 \leftrightarrow \forall x \, (A(x) \to \neg (xRx))\]
\item
  Asymmetry:
  \[R \subseteq A^2 \leftrightarrow \forall x \forall y \, (A(x,y) \to (xRy \to \neg (yRx)))\]
\item
  Symmetry:
  \[R \subseteq A^2 \leftrightarrow \forall x \forall y \, (A(x,y) \to (xRy \to yRx))\]
\item
  Antisymmetry:
  \[R \subseteq A^2 \leftrightarrow \forall x \forall y \, (A(x,y) \to ((xRy \land yRx) \to x = y))\]
\item
  Transitivity:
  \[R \subseteq A^2 \leftrightarrow \forall x \forall y \forall z \, (A(x,y,z) \to ((xRy \land yRz) \to xRz))\]
\item
  Connectivity:
  \[R \subseteq A^2 \leftrightarrow \forall x \forall y \, (A(x,y) \to (x \neq y \to (xRy \lor yRx)))\]
\end{enumerate}

\subsection{1. Reflexivity}\label{reflexivity}

\[\forall x \in A, \ xRx\]

Example:

The relation \(\leq\) on \(\mathbb{R}\) is reflexive because
\(3 \leq 3\). Likewise, the subset relation \(\subseteq\) is reflexive:
\(\{1\} \subseteq \{1\}\). In both cases, every element is related to
itself.

\subsection{2. Irreflexivity}\label{irreflexivity}

\[\forall x \in A, \ \neg(xRx)\]

Example:

The relation \(<\) on \(\mathbb{N}\) is irreflexive because no number is
less than itself: \(5 < 5\) is false, so \(\neg(5<5)\).

\subsection{3. Asymmetry}\label{asymmetry}

\[\forall x,y \in A,\ xRy \to \neg(yRx)\]

Example:

The relation \(<\) on \(\mathbb{N}\) is asymmetric because \(2 < 3\),
but \(3 < 2\) is false. There is no pair such that both \(xRy\) and
\(yRx\) hold. Therefore, \(<\) is asymmetric. Any asymmetric relation is
automatically irreflexive.

\subsection{4. Symmetry}\label{symmetry}

\[\forall x,y \in A, \ xRy \to yRx\]

Example:

Let

\[A = \{1,2,3\} \quad \text{and} \quad R = \{1R2, 2R1, 2R3, 3R2\}\]

This relation is symmetric because:

\[1R2  \quad \text{and} \quad 2R1\]

\[2R3 \quad \text{and} \quad 3R2\]

Every pair has its mirror image in the relation. Therefore, \(R\) is
symmetric.

\subsection{5. Antisymmetry}\label{antisymmetry}

\[\forall x,y \in A,\ (xRy \land yRx) \to x = y\]

Example 1 (on \(\mathbb{N}\)):

Let \(R = \leq\). If \(x \leq y\) and \(y \leq x\), then \(x = y\).
Therefore, \(\leq\) is antisymmetric.

\subsection{6. Transitivity}\label{transitivity}

\[\forall x,y,z \in A,\ (xRy \land yRz) \to xRz\]

Example:

The relation \(<\) on \(\mathbb{N}\) is transitive:

\[2 < 4 \land 4 < 5 \to 2 < 5\]

\subsection{7. Connectivity}\label{connectivity}

\[\forall x,y \in A,\ x \neq y \to (xRy \lor yRx)\]

Example 1 (on \(\mathbb{N}\)):

The relation \(\leq\) is connected because for any distinct \(x,y\),
either \(x \leq y\) or \(y \leq x\).

Example 2 (on \(<\)):

The relation \(<\) on \(\mathbb{N}\) is connected because for
\(x \ne y\), either \(x < y\) or \(y < x\).

Importantly, in evaluating the relations, it is also necessary to learn
the order. Specifically, If \(L(x,y)\) is a binary predicate, then
\(L(a,b)\) and \(L(b,a)\) are not the same formula unless the semantics
of \(L\) itself makes them equivalent.

Example:

If \(L(x,y)\) means ``\(x\) loves \(y\)'', then:

− \(L(a,b) =\) ``\(a\) loves \(b\)''

− \(L(b,a) =\) ``\(b\) loves \(a\)''

These are logically distinct, unless we assume love is symmetric. So we
usually use this bracket symbol to stress the order
\(\langle a, b \rangle\). Nevertheless, in general, for FOL, we can
still use (,), but keep in mind that the order matters.
