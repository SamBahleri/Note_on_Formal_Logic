Ven Diagram

One illustrative or intuitive way to represent whether a formula is
valid or not is by using a Venn diagram. In this case, Venn diagrams can
be used to represent the validity of syllogisms and propositions.
However, for this chapter, we will only study Venn diagrams for
propositions.

\subsubsection{1. Ven Diagram for
Conjunction}\label{ven-diagram-for-conjunction}

A Venn diagram for conjunction is used to show the overlap between sets
\(A\) and \(B\), which means both \(A\) and \(B\) are true. The
overlapping area represents:

\[
A \wedge B
\]

\begin{verbatim}
<img src="/2_Logic_Blog/Branch_of_Logic/Formal_Logic/Ven Dagram/Ven_Picture/Conjunction.png" alt="Diagram" class="responsive-image">
<div class="image-caption">
    Picture 1: Illustration of Conjunction
</div>
\end{verbatim}

\subsubsection{2. Venn Diagram for
Disjunction}\label{venn-diagram-for-disjunction}

A Venn diagram for disjunction is used to show the total area covered by
sets \(A\) and \(B\), including both individual and overlapping parts.
This represents \(A \vee B\), meaning either \(A\), \(B\), or both are
true.

\[
A \vee B
\]

\begin{verbatim}
<img src="/2_Logic_Blog/Branch_of_Logic/Formal_Logic/Ven Dagram/Ven_Picture/Disjunction.png" alt="Diagram" class="responsive-image">
<div class="image-caption">
    Picture 2: Illustration of Disjunction
</div>
\end{verbatim}

\subsubsection{3. Ven Diagram for
Negation}\label{ven-diagram-for-negation}

A Venn diagram for negation is used to show the area outside of circle
\(A\), meaning everything not in \(A\). This represents that \(A\) is
false.

\[
\neg A
\]

\begin{verbatim}
<img src="/2_Logic_Blog/Branch_of_Logic/Formal_Logic/Ven Dagram/Ven_Picture/Negation.png" alt="Diagram" class="responsive-image">
<div class="image-caption">
    Picture 3: Illustration of Negation
</div>
\end{verbatim}

\subsubsection{4. Venn Diagram for
Implication}\label{venn-diagram-for-implication}

A Venn diagram for implication is used to show that if \(A\) is true,
then \(B\) must also be true. Visually, this means the part of \(A\)
that is not in \(B\) is excluded, so \(A\) is entirely inside \(B\), or
the area where \(A\) is outside \(B\) is empty.

\[
A \to B
\]

\begin{verbatim}
<img src="/2_Logic_Blog/Branch_of_Logic/Formal_Logic/Ven Dagram/Ven_Picture/Implication.png" alt="Diagram" class="responsive-image">
<div class="image-caption">
    Picture 4: Illustration of Implication
</div>
\end{verbatim}

Moreover, that with equivalent formulas we can also show the ilustartion
with the Ven Diagram. For example:

\subsubsection{5. More example}\label{more-example}

Example 1.

We can also represent equivalent formulas with Venn diagrams. For
example, by De Morgan's Law: the Venn diagram for \(\neg (P \wedge Q)\)
shows the area outside the overlap of \(P\) and \(Q\). It includes
everything except the region where both \(P\) and \(Q\) are true.

\[
\neg (P \wedge Q) \;\;\equiv\;\; (\neg P) \vee (\neg Q)
\]

\begin{verbatim}
<img src="/2_Logic_Blog/Branch_of_Logic/Formal_Logic/Ven Dagram/Ven_Picture/De Morgans.png" alt="Diagram" class="responsive-image">
<div class="image-caption">
    Picture 5: Illustration of De Morgan’s Law
</div>
\end{verbatim}

Example 2.

The Venn diagram for \(\neg P \wedge Q\) shows the area where \(Q\) is
true but \(P\) is false. It includes the part inside \(Q\) but outside
\(P\).

\begin{verbatim}
<img src="/2_Logic_Blog/Branch_of_Logic/Formal_Logic/Ven Dagram/Ven_Picture/Not P and Q.png" alt="Diagram" class="responsive-image">
<div class="image-caption">
    Picture 6: Illustration of $\neg P \wedge Q$
</div>
\end{verbatim}
