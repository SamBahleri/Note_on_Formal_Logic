Scope of Deductive and Inductive

\subsection{1. Scope of Deductive}\label{scope-of-deductive}

\subsubsection{1.1. Validity}\label{validity}

A valid argument is one that is based on the principles of deduction, in
which the conclusion logically follows from the premises. It's important
to note that validity here refers not to the truth of the premises, but
solely to the logical relationship between the premises and the
conclusion.

Example 1:

\$

\begin{aligned}
\text{Premise 1:} \ & \text{All humans are living beings} \\
\text{Premise 2:} \ & \text{Sam is a human} \\
\text{Conclusion:} \ & \text{Therefore, Sam is a living being}
\end{aligned}

\$

The example above is an argument that is structurally, categorically,
and definitionally correct. Thus, the truth of the argument is
acceptable because both the logical structure and the content of the
premises align and support a valid conclusion.

Example 2:

\$

\begin{aligned}
\text{Premise 1:} \ & \text{All birds can talk} \\
\text{Premise 2:} \ & \text{A pigeon is a bird} \\
\text{Conclusion:} \ & \text{Therefore, a pigeon can talk}
\end{aligned}

\$

We can observe that this second example is structurally valid, but
factually incorrect in its content. This is because not all birds can
talk. Therefore, an argument can be valid when judged solely by the
structure of the premises; however, when examining the content, an
argument might be valid but not sound. This brings us to the concept of
soundness.

\subsubsection{1.2. Soundness}\label{soundness}

A sound argument is one that is both valid in structure (logical form)
and true in content (factual). Thus, the conclusion of a sound argument
can be guaranteed to be true or accurate. In other words, to build a
sound argument, we must first ensure that the premises are factually
correct, and then organize them into a logically valid structure.

Example 1:

\$

\begin{aligned}
\text{Premise 1:} \ & \text{All mammals are warm-blooded creatures} \\
\text{Premise 2:} \ & \text{Dolphins are mammals} \\
\text{Conclusion:} \ & \text{Therefore, dolphins are warm-blooded creatures}
\end{aligned}

\$

This example is both structurally valid and factually accurate in its
premises. Therefore, we can conclude that a sound argument follows this
pattern:

\[
\text{Sound Argument} = \text{Factual Premises} + \text{Valid Form}
\]

From the concepts of validity and soundness, we can summarize the
following:

\begin{enumerate}
\def\labelenumi{\arabic{enumi})}
\item
  An argument can be valid in logical form even if its premises are not
  factually correct. In this case, even though the logical structure is
  correct, the conclusion cannot be guaranteed true, because the content
  of the argument is not factual.
\item
  A sound argument is an accurate and trustworthy argument because its
  logical structure is valid, and all its premises are factually true.
  Therefore, a sound argument always produces a conclusion that is
  certainly true.
\end{enumerate}

\subsection{2. Scope of Inductive}\label{scope-of-inductive}

\subsubsection{2.1. Strong Argument}\label{strong-argument}

An inductive argument is considered strong if its premises provide a
high degree of probability in support of the conclusion. The
characteristics of a strong inductive argument are similar in structure
to deductive arguments, but the conclusion remains uncertain or not
absolutely guaranteed.

Example:

\$

\begin{aligned}
\text{Premise 1:} \ & \text{90 percent of the students in this class passed the formal logic exam} \\
\text{Premise 2:} \ & \text{Noel is a student in this class} \\
\text{Conclusion:} \ & \text{Most likely, Noel passed the formal logic exam}
\end{aligned}

\$

As we can observe, the premises resemble those of a deductive argument.
However, the conclusion is not definite, we still need to verify whether
Noel indeed passed the exam. This brings us to the next concept: the
cogent argument.

\paragraph{2.1.1. Cogent}\label{cogent}

The term cogent is closely related to strong, but it includes an
additional requirement. The characteristics of a cogent argument are:

\begin{enumerate}
\def\labelenumi{\alph{enumi}.}
\item
  The premises are actually true in reality
\item
  The argument is structuraly like deductive argument
\item
  Then, the conclusion is plausible and well-supported
\end{enumerate}

Example:

\$

\begin{aligned}
\text{Premise 1:} \ & \text{Cats have whiskers} \\
\text{Premise 2:} \ & \text{Cats have whiskers} \\
\text{Conclusion:} \ & \text{Therefore, Luna probably has whiskers}
\end{aligned}

\$

The argument above is cogent, because we intuitively agree that cats
generally have whiskers. However, given the nature of inductive
reasoning, which moves from general observations to specific instances,
we still need to confirm whether Luna, the cat in question, indeed has
whiskers. In short, a cogent argument is a strong inductive argument
with true premises.

\subsubsection{2.2. Weak Argument}\label{weak-argument}

A weak inductive argument is one in which the premises do not provide
strong support for the conclusion, or the conclusion does not logically
follow from the given premises. Even if the premises are true, they fail
to give a high probability that the conclusion is also true.

Example:

\$

\begin{aligned}
\text{Premise 1:} \ & \text{Dogs are cute and make good pets} \\
\text{Premise 2:} \ & \text{Cats are cute and make good pets} \\
\text{Premise 3:} \ & \text{Pandas are cute and make good pets} \\
\text{Conclusion:} \ & \text{Therefore, every cute animal makes a good pet}
\end{aligned}

\$

This argument is clearly weak, because the premises only refer to a few
cute animals. Moreover, the use of the phrase ``every cute animal'' in
the conclusion is an overgeneralization. Not all cute animals make good
pets, since there are many other factors to consider, such as
temperament, size, habitat, and potential danger.

Below is a table of indicators that we should pay attention to, both
when constructing an argument and when reading or listening to one. The
purpose is to help us distinguish between deductive and inductive
arguments, whether in social or academic contexts.

Category

Deductive Argument

Inductive Argument

Certainty of Conclusion

certainly, always, cannot be false, must, necessarily

possibly, likely, may, could, probably

Common Indicators

therefore, thus, it must be, hence, as a result

likely, most likely, apparently, it could be, tends to

Purpose of Argument

to prove truth with absolute certainty/logically

to show probability or tendency

Logical Structure

if\ldots{} then\ldots, all\ldots, none\ldots, every\ldots{}

most\ldots, often\ldots, usually\ldots, on average\ldots{}

Type of Conclusion

conclusion is definitely true if premises are true

conclusion is a prediction or generalization
