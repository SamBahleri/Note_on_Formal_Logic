Semantic Tableaux

Semantic Tableaux, commonly called a truth tree, is an effective method
for demonstrating the validity of a formula. The Semantic Tableaux
method was developed by
\href{https://en.wikipedia.org/wiki/Evert_Willem_Beth}{Evert Willem
Beth}, a Dutch logician, in the 1950s. The process is similar to
Fitch-style natural deduction, but Semantic Tableaux uses branches
(arrows) on each proposition to indicate contradictions.

\subsection{1. Conjunction
Decomposition}\label{conjunction-decomposition}

\begin{verbatim}
<img src="/2_Logic_Blog/Branch_of_Logic/Formal_Logic/Trees/1.png" alt="Diagram" class="responsive-image">
<div class="image-caption">
    Picture 1: Conjunction Decomposition
</div>
\end{verbatim}

\subsection{2. Negated Conjunction
Decomposition}\label{negated-conjunction-decomposition}

\begin{verbatim}
<img src="/2_Logic_Blog/Branch_of_Logic/Formal_Logic/Trees/2.png" alt="Diagram" class="responsive-image">
<div class="image-caption">
    Picture 2: Negated Conjunction Decomposition
</div>
\end{verbatim}

\subsection{3. Disjunction
Decomposition}\label{disjunction-decomposition}

\begin{verbatim}
<img src="/2_Logic_Blog/Branch_of_Logic/Formal_Logic/Trees/3.png" alt="Diagram" class="responsive-image">
<div class="image-caption">
    Picture 3: Disjunction Decomposition
</div>
\end{verbatim}

\subsection{4. Negated Disjunction
Decomposition}\label{negated-disjunction-decomposition}

\begin{verbatim}
<img src="/2_Logic_Blog/Branch_of_Logic/Formal_Logic/Trees/4.png" alt="Diagram" class="responsive-image">
<div class="image-caption">
    Picture 4: Negated Disjunction Decomposition
</div>
\end{verbatim}

\subsection{5. Double Negation}\label{double-negation}

\begin{verbatim}
<img src="/2_Logic_Blog/Branch_of_Logic/Formal_Logic/Trees/5.png" alt="Diagram" class="responsive-image">
<div class="image-caption">
    Picture 5: Double Negation
</div>
\end{verbatim}

\subsection{6. Conditional
Decomposition}\label{conditional-decomposition}

\begin{verbatim}
<img src="/2_Logic_Blog/Branch_of_Logic/Formal_Logic/Trees/6.png" alt="Diagram" class="responsive-image">
<div class="image-caption">
    Picture 6: Conditional Decomposition
</div>
\end{verbatim}

\subsection{7. Negated Conditional
Decomposition}\label{negated-conditional-decomposition}

\begin{verbatim}
<img src="/2_Logic_Blog/Branch_of_Logic/Formal_Logic/Trees/7.png" alt="Diagram" class="responsive-image">
<div class="image-caption">
    Picture 7: Negated Conditional Decomposition
</div>
\end{verbatim}

\subsection{8. Biconditional
Decomposition}\label{biconditional-decomposition}

\begin{verbatim}
<img src="/2_Logic_Blog/Branch_of_Logic/Formal_Logic/Trees/8.png" alt="Diagram" class="responsive-image">
<div class="image-caption">
    Picture 8: Biconditional Decomposition
</div>
\end{verbatim}

\subsection{9. Negated Biconditional
Decomposition}\label{negated-biconditional-decomposition}

\begin{verbatim}
<img src="/2_Logic_Blog/Branch_of_Logic/Formal_Logic/Trees/9.png" alt="Diagram" class="responsive-image">
<div class="image-caption">
    Picture 9: Negated Biconditional Decomposition
</div>
\end{verbatim}
