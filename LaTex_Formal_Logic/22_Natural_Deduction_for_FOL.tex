Natural Deduction for FOL

The way we perform deductions in first-order logic (FOL) is essentially
similar to propositional logic (PL), for example, we use rules like
Modus Ponens (\(MP\)) and Modus Tollens (\(MT\)). However, in FOL the
process is generally longer because we must also handle quantifiers and
carefully substitute variables before applying these rules.

\subsubsection{\texorpdfstring{1. Universal elimination
(\(\forall E\))}{1. Universal elimination (\textbackslash forall E)}}\label{universal-elimination-forall-e}

\begin{verbatim}
<img src="/2_Logic_Blog/Branch_of_Logic/Formal_Logic/FOL_ND/1.png" alt="Diagram" class="responsive-image">
<div class="image-caption">
    Picture 1:  Universal elimination
</div>
\end{verbatim}

Universal elimination allows us to instantiate a universally quantified
statement with any specific term. If something is true for all objects
in the domain, then it must be true for any particular object we choose.

Given: \[\forall x\, P(x)\] We can conclude:
\[P(a)  \quad \text{for any term $a$ } \]

The rule can be written as: \[\frac{\forall x\, P(x)}{\therefore P(a)}\]

\subsubsection{\texorpdfstring{2. Existential Introduction
(\(\exists I\))}{2. Existential Introduction (\textbackslash exists I)}}\label{existential-introduction-exists-i}

\begin{verbatim}
<img src="/2_Logic_Blog/Branch_of_Logic/Formal_Logic/FOL_ND/2.png" alt="Diagram" class="responsive-image">
<div class="image-caption">
    Picture 2: Existential Introduction
</div>
\end{verbatim}

Given: \[P(a)\] We can conclude: \[\exists x\, P(x)\]

The rule can be written as: \[\frac{P(a)}{\therefore \exists x\, P(x)}\]

\subsubsection{\texorpdfstring{3. Universal Introduction
(\(\forall I\))}{3. Universal Introduction (\textbackslash forall I)}}\label{universal-introduction-forall-i}

\begin{verbatim}
<img src="/2_Logic_Blog/Branch_of_Logic/Formal_Logic/FOL_ND/3.png" alt="Diagram" class="responsive-image">
<div class="image-caption">
    Picture 3: Universal Introduction
</div>
\end{verbatim}

Universal introduction allows us to conclude a universal statement if we
can prove something about an arbitrary object. The key constraint is
that the object must be completely arbitrary, it cannot depend on any
specific assumptions about particular objects.

From a proof that establishes \(P(a)\) where \(a\) is arbitrary, we can
conclude: \[\forall x\, P(x)\]

The rule can be written as:
\[\frac{\text{[Arbitrary } a\text{]} \quad P(a)}{\therefore \forall x\, P(x)}\]

\subsubsection{\texorpdfstring{4. Existential elimination
(\(\exists E\))}{4. Existential elimination (\textbackslash exists E)}}\label{existential-elimination-exists-e}

\begin{verbatim}
<img src="/2_Logic_Blog/Branch_of_Logic/Formal_Logic/FOL_ND/4.png" alt="Diagram" class="responsive-image">
<div class="image-caption">
    Picture 4:  Existential elimination
</div>
\end{verbatim}

Existential elimination allows us to use an existential statement in a
proof. Since we know something exists with a certain property, we can
temporarily assume we have such an object (giving it a fresh name) and
see what follows.

Given: \[\exists x\, P(x)\]

We can assume \(P(a)\) for a fresh name \(a\), derive some conclusion
\(Q\), and then conclude \(Q\) without the assumption.

The rule can be written as:
\[\frac{\exists x\, P(x) \quad \text{[Assume } P(a)\text{]} \quad Q}{\therefore Q}\]

\subsubsection{\texorpdfstring{5. Identity Introduction
(\(=I\))}{5. Identity Introduction (=I)}}\label{identity-introduction-i}

\begin{verbatim}
<img src="/2_Logic_Blog/Branch_of_Logic/Formal_Logic/FOL_ND/5.png" alt="Diagram" class="responsive-image">
<div class="image-caption">
    Picture 5:  Identity Introduction
</div>
\end{verbatim}

Identity introduction states that every object is identical to itself.
This is a logical truth that requires no premises.

We can always conclude: \[a = a \quad \text{for any term} \quad a\]

The rule can be written as: \[\frac{}{\therefore a = a}\]

This reflects the reflexive property of identity.

\subsubsection{\texorpdfstring{5. Identity elimination
(\(=E\))}{5. Identity elimination (=E)}}\label{identity-elimination-e}

\begin{verbatim}
<img src="/2_Logic_Blog/Branch_of_Logic/Formal_Logic/FOL_ND/6.png" alt="Diagram" class="responsive-image">
<div class="image-caption">
    Picture 6:  Identity elimination
</div>
\end{verbatim}

Identity elimination, also known as Leibniz's Law, states that identical
objects have identical properties. If two terms refer to the same
object, we can substitute one for the other in any context.

Given: \[a = b \quad \text{and} \quad P(a)\] We can conclude: \[P(b)\]

The rule can be written as: \[\frac{a = b \quad P(a)}{\therefore P(b)}\]

\subsubsection{6. Conversion of Quantifiers (CQ)
Rules}\label{conversion-of-quantifiers-cq-rules}

Recall the Quantifier Equivalences, these derived rules show the
relationship between quantifiers and negation:

\begin{enumerate}
\def\labelenumi{\arabic{enumi}.}
\item
  CQ1: \[\forall x\, \neg P(x) \leftrightarrow \neg \exists x\, P(x)\]
  \emph{Everything is not P} is equivalent to \emph{Nothing is P}
\item
  CQ2: \[\neg \exists x\, P(x) \leftrightarrow \forall x\, \neg P(x)\]
  \emph{Nothing is P} is equivalent to \emph{Everything is not P}
\item
  CQ3: \[\exists x\, \neg P(x) \leftrightarrow \neg \forall x\, P(x)\]
  \emph{Something is not P} is equivalent to \emph{Not everything is P}
\item
  CQ4: \[\neg \forall x\, P(x) \leftrightarrow \exists x\, \neg P(x)\]
  \emph{Not everything is P} is equivalent to \emph{Something is not P}
\end{enumerate}
