Boolean, DNF, and CNF

\subsection{1. Boolean}\label{boolean}

In short, the application of Boolean Algebra is essentially the same as
Formal Logic. The main difference lies in the symbols used. For example,
in Formal Logic, the symbol \(\land\) is used for logical AND, whereas
in Boolean Algebra we often use \(\,\cdot\,\) (dot) for AND. Logical OR,
represented by \(\lor\) in Formal Logic, corresponds to \(+\) in Boolean
Algebra. Negation, \(\neg x\) in Formal Logic, is written as \(\bar{x}\)
in Boolean Algebra. Sometimes, you may also encounter the AND and OR
symbols inside circles: \(\oplus\) for XOR (exclusive OR) and \(\odot\)
or \(\otimes\) for AND, depending on the text. These are just
alternative notations and do not change the underlying logic.

Furthermore, it is important to note that Boolean Algebra is not the
same as conventional mathematics. Specifically, Boolean Algebra operates
strictly on binary values (\(1\) and \(0\)). General mathematics
typically deals with a wider range of numbers, including integers,
fractions, and complex numbers.

\begin{longtable}[]{@{}
  >{\raggedright\arraybackslash}p{(\linewidth - 2\tabcolsep) * \real{0.4545}}
  >{\raggedright\arraybackslash}p{(\linewidth - 2\tabcolsep) * \real{0.5455}}@{}}
\toprule\noalign{}
\begin{minipage}[b]{\linewidth}\raggedright
Property
\end{minipage} & \begin{minipage}[b]{\linewidth}\raggedright
Expression
\end{minipage} \\
\midrule\noalign{}
\endhead
\bottomrule\noalign{}
\endlastfoot
Binary Operation \(+\) & \(1 + 0 = 1\) \\
Commutative Property of \(+\) & \(1 + 0 = 0 + 1\) \\
Associative Property of \(+\) & \((1 + 0) + 1 = 1 + (0 + 1)\) \\
Binary Operation \(\cdot\) & \(1 \cdot 1 = 1\) \\
Commutative Property of \(\cdot\) & \(1 \cdot 0 = 0 \cdot 1\) \\
Associative Property of \(\cdot\) &
\((1 \cdot 0) \cdot 1 = 1 \cdot (0 \cdot 1)\) \\
Distributive Properties &
\(1 \cdot (0 + 1) = (1 \cdot 0) + (1 \cdot 1)\) \\
Special Combinations & \(a = a \cdot (a + b)\) \\
Unary Operation \(\bar{}\) (Negation) & \(\bar{1} = 0, \bar{0} = 1\) \\
Identity with \(+\) & \(a + 0 = a\) \\
Annihilation with \(\cdot\) & \(a \cdot 0 = 0\) \\
Proof using Commutativity of \(+\) & \(a + b = b + a\) \\
Proof using Commutativity of \(\cdot\) & \(a \cdot b = b \cdot a\) \\
Identity Property of \(+\) & \(a + 0 = a\) \\
Zero for \(+\) & \(0 + a = a\) \\
Commutative Property for Zero & \(0 + a = a + 0\) \\
Identity with \(\cdot\) & \(a \cdot 1 = a\) \\
Idempotent Law for \(+\) & \(a + a = a\) \\
Idempotent Law for \(\cdot\) & \(a \cdot a = a\) \\
Annihilation with \(+\) & \(a + 1 = 1\) \\
Annihilation with \(\cdot\) & \(a \cdot 0 = 0\) \\
Double Negation & \(\overline{\overline{a}} = a\) \\
De Morgan's Law for \(+\) &
\(\overline{a \cdot b} = \bar{a} + \bar{b}\) \\
De Morgan's Law for \(\cdot\) &
\(\overline{a + b} = \bar{a} \cdot \bar{b}\) \\
\end{longtable}

Let \(k \geq 1\). A \(k\)-ary Boolean function \(f\) takes as input
\(k\) True/False values and outputs a True/False value; namely, it is a
mapping.

\[f : \{1, 0\}^k \rightarrow \{1, 0\}\]

Example:

\[f_{p_1 \cdot p_2}(x_1, x_2) = \begin{cases} 1 & \text{if } x_1 \text{ and } x_2 \text{ both equal } 1 \\ 0 & \text{otherwise} \end{cases}\]

Let \(k \geq 1\) and let \(A\) be a propositional formula that uses (at
most) the variables \(p_1, \ldots, p_k\). The \(k\)-ary Boolean function
\(f_A(x_1, x_2, \ldots, x_k)\) is defined by

\[f_A(x_1, \ldots, x_k) = \varphi(A), \text{ where } \varphi(p_i) = x_i \text{ for } i = 1, 2, \ldots, k\]

Construction:

\[f(x_1, x_2) = \begin{cases} 1 & \text{if } x_1 = 1 \text{ or } x_2 = 0 \\ 0 & \text{otherwise} \end{cases}\]

\begin{longtable}[]{@{}lll@{}}
\toprule\noalign{}
\(x_1\) & \(x_2\) & \(f(x_1, x_2)\) \\
\midrule\noalign{}
\endhead
\bottomrule\noalign{}
\endlastfoot
\(1\) & \(1\) & \(1\) \\
\(1\) & \(0\) & \(1\) \\
\(0\) & \(1\) & \(0\) \\
\(0\) & \(0\) & \(1\) \\
\end{longtable}

\[(p_1 \cdot p_2) + (p_1 \cdot \bar p_2) + (\bar p_1 \cdot \bar p_2)\]

\subsubsection{1.2. General Construction}\label{general-construction}

For \(1 \leq i \leq 2^k\) and \(1 \leq j \leq k\):

Form Disjunctions (Clauses):

\[\psi_i = \displaystyle\bigvee_{j=1}^k L_{i,j} := L_{i,1} + L_{i,2} + \cdots + L_{i,k}\]

Form Conjunctions (Clauses):

\[\Sigma_i = \displaystyle\bigwedge_{j=1}^k L_{i,j} := L_{i,1} \cdot L_{i,2} \cdot \cdots \cdot L_{i,k}\]

Example:

Let's take two propositional variables:\\
\[p_1, p_2\]

Possible truth assignments:

\begin{longtable}[]{@{}lcc@{}}
\toprule\noalign{}
\(i\) & \(p_1\) & \(p_2\) \\
\midrule\noalign{}
\endhead
\bottomrule\noalign{}
\endlastfoot
1 & 1 & 1 \\
2 & 1 & 0 \\
3 & 0 & 1 \\
4 & 0 & 0 \\
\end{longtable}

For each row \(i\), we will form literals \(L_{i,1}\) and \(L_{i,2}\):
1. If \(\varphi_i(p_j) = 1\), then \(L_{i,j} = p_j\) 2. If
\(\varphi_i(p_j) = 0\), then \(L_{i,j} = \bar p_j\)

Conjunction: \[
C_i = L_{i,1} \cdot L_{i,2}
\]

\begin{longtable}[]{@{}lcl@{}}
\toprule\noalign{}
\(i\) & \((p_1, p_2)\) & \(C_i\) \\
\midrule\noalign{}
\endhead
\bottomrule\noalign{}
\endlastfoot
1 & (1, 1) & \(p_1 \cdot p_2\) \\
2 & (1, 0) & \(p_1 \cdot \bar p_2\) \\
3 & (0, 1) & \(\bar p_1 \cdot p_2\) \\
4 & (0, 0) & \(\bar p_1 \cdot \bar p_2\) \\
\end{longtable}

Each \(C_i\) corresponds to \emph{one minterm}, True for exactly one
assignment.

Disjunction:

\[
D_i = L_{i,1} + L_{i,2}
\]

\begin{longtable}[]{@{}lcl@{}}
\toprule\noalign{}
\(i\) & \((p_1, p_2)\) & \(D_i\) \\
\midrule\noalign{}
\endhead
\bottomrule\noalign{}
\endlastfoot
1 & (1, 1) & \(p_1 + p_2\) \\
2 & (1, 0) & \(p_1 + \bar p_2\) \\
3 & (0, 1) & \(\bar p_1 + p_2\) \\
4 & (0, 0) & \(\bar p_1 + \bar p_2\) \\
\end{longtable}

Each \(D_i\) corresponds to \emph{one maxterm}, False for exactly one
assignment.

\subsection{2. Disjunctive Normal Form
(DNF)}\label{disjunctive-normal-form-dnf}

The idea of DNF is that we can reduce all connectives in propositional
logic (PL) to only \(+ \ \text{(Disjucntion)}\),
\(.\ \text{(Conjunction)}\), and \$\bar x ~\text{(negation)} \$ in order
to determine the truth value of a sentence.

For example:

\[(s \cdot a) + (\bar m \cdot b) + (\bar h \cdot \bar l) + (r \cdot \bar i)\]

Moreover, for convenience we use \((\pm L)\) to indicate that \(L\) is
an atomic sentence which may or may not be prefaced with an occurrence
of negation. DNF example:

\[(\pm L_1 \cdot \dots \cdot \pm L_{i+1}) + (\pm L_j \cdot \dots \cdot \pm L_{k+1})\]

In detail:

Let \(L\) be a sentence containing three atomic sentences \(a, b, c\).

\(a\)

\(b\)

\(c\)

\(L\)

1

1

1

1

1

1

0

0

1

0

1

1

1

0

0

0

0

1

1

0

0

1

0

0

0

0

1

1

0

0

0

1

From the truth table, we observe that \(L\) is true on lines
\(1, 3, 7, 8\). For each of these lines, we construct a conjunction that
captures the exact truth value assignment on that line:

\begin{enumerate}
\def\labelenumi{\arabic{enumi}.}
\tightlist
\item
  Line 1: \((a \cdot b \cdot c)\)
\item
  Line 3: \((a \cdot \bar b \cdot c)\)
\item
  Line 7: \((\bar a \cdot \bar b \cdot c)\)
\item
  Line 8: \((\bar a \cdot \bar b \cdot \bar c)\)
\end{enumerate}

Therefore, a DNF representation logically equivalent to \(L\) is:

\[(a \cdot b \cdot c) + (a \cdot \bar b \cdot c) + (\bar a \cdot \bar b \cdot c) + (\bar a \cdot \bar b \cdot \bar c)\]

This gives us a sentence in DNF which is true on exactly those lines
where one of the disjuncts is true, namely, lines 1, 3, 7, and 8. Hence,
the DNF sentence has exactly the same truth table as \(L\). In other
words, we have produced a sentence in DNF that is logically equivalent
to \(L\), which is precisely the goal of normalization.

\subsubsection{2.1. General Construction}\label{general-construction-1}

Let \(L\) be a sentence containing atomic sentences \(x_1, \dots, x_n\).
We aim to construct a new sentence, \(G\), that is \emph{logically
equivalent} to \(L\) but written in DNF. Intuitively, \(G\) will serve
as a \emph{truth-functional reconstruction} of \(L\): it will reproduce
the truth value of \(L\) on every possible assignment of truth values to
\(x_1, \dots, x_n\). That is, for each possible line of the truth table
for \(L\), \(G\) will be true if and only if \(L\) is true on that line.
Hence, \(G\) acts as the DNF counterpart of \(L\), the formula that
expresses \(L\)'s truth behavior in canonical disjunctive form.

\begin{longtable}[]{@{}cccccc@{}}
\toprule\noalign{}
Line & \(x_1\) & \(x_2\) & \(\dots\) & \(x_n\) & \(L\) \\
\midrule\noalign{}
\endhead
\bottomrule\noalign{}
\endlastfoot
1 & 1 & 1 & \(\dots\) & 1 & \(L_1\) \\
2 & 1 & 1 & \(\dots\) & 0 & \(L_2\) \\
3 & 1 & 0 & \(\dots\) & 1 & \(L_3\) \\
⋮ & ⋮ & ⋮ & ⋮ & ⋮ & ⋮ \\
\(2^n\) & 0 & 0 & \(\dots\) & 0 & \(L_{2^n}\) \\
\end{longtable}

Case 1:

If \(L\) is false on every line of its truth table, then \(L\) is a
contradiction. In that case, a contradiction of the form is in DNF and
is logically equivalent to \(L\).

Case 2:

If \(L\) is true on at least one line of its truth table, we proceed as
follows. For each line \(i\) where \(L = 1\), let \(\psi_i\) be a
conjunction of the form:

\[\psi_i = \displaystyle\bigwedge_{m=1}^{n} (\pm x_m)\]

where

\[
(\pm x_m) =
\begin{cases}
x_m, & \text{if } x_m = 1 \text{ on line } i, \\
\bar x_m, & \text{if } x_m = 0 \text{ on line } i.
\end{cases}
\]

That is, \(x_m\) appears unnegated in \(\psi_i\) if it is true on line
\(i\), and negated if it is false on line \(i\). Thus, \(\psi_i\) is
true on, and only on, line \(i\) of the truth table.

Now, let \(i_1, i_2, \dots, i_j\) be the indices of all lines where
\(L = 1\). Then we define \(G\) as the disjunction of all such
\(psi_i\):

\[G = \psi_{i_1} + \psi_{i_2} + \dots + \psi_{i_j} = \displaystyle\bigvee_{k=1}^{j} \left[ \bigwedge_{m=1}^{n} (\pm x_m) \right]\]

Since \(G\) is a disjunction of conjunctions of literals, \(G\) is in
DNF. By construction, \(L\) and \(G\) are true on exactly the same lines
of the truth table, and therefore

\[L \equiv G\]

Example:

Let \(S\) contain three atomic sentences \(x_1, x_2, x_3\).

\(x_1\)

\(x_2\)

\(x_3\)

\(S\)

1

1

1

1

1

1

0

0

1

0

1

1

1

0

0

0

0

1

1

0

0

1

0

0

0

0

1

1

0

0

0

1

\(S = 1\) on lines 1, 3, 7, and 8.

\begin{enumerate}
\def\labelenumi{\arabic{enumi}.}
\tightlist
\item
  For line 1:\(f_1 = (x_1 \cdot x_2 \cdot x_3)\)
\item
  For line 3:\(f_3 = (x_1 \cdot \bar x_2 \cdot x_3)\)
\item
  For line 7:\(f_7 = (\bar x_1 \cdot \bar x_2 \cdot x_3)\)
\item
  For line 8:\(f_8 = (\bar x_1 \cdot \bar x_2 \cdot \bar x_3)\)
\end{enumerate}

Form the disjunction: \[B = f_1 + f_3 + f_7 + f_8\]

Hence,

\[\big[(x_1 \cdot x_2 \cdot x_3) + (x_1 \cdot \bar x_2 \cdot x_3) + (\bar x_1 \cdot \bar x_2 \cdot x_3) + (\bar x_1 \cdot \bar x_2 \cdot \bar x_3)\big]\]

We conclude that, \[S \equiv B\]

\subsection{3. Conjunctive Normal Form
(CNF)}\label{conjunctive-normal-form-cnf}

Shortly, the formal definition of CNF is also analogous to the
definition of DNF.

For example:

\[(s + a) \cdot (\bar m + b) \cdot (\bar h + \bar l) \cdot (r + \bar i)\]

and

\[(\pm T_1 + \dots + \pm P_m) \cdot (\pm T_i + \dots + \pm T_k)\]

In detail:

Consider a sentence \(T\) with atomic sentences \(p, q, r\).

\(p\)

\(q\)

\(r\)

\(B\)

1

1

1

1

1

1

0

0

1

0

1

1

1

0

0

0

0

1

1

1

0

1

0

0

0

0

1

1

0

0

0

1

The truth table shows \(T\) is false on lines \(2, 4, 6\). We build a
clause for each false row by taking the disjunction of literals that
would make that row true:

\begin{enumerate}
\def\labelenumi{\arabic{enumi}.}
\tightlist
\item
  Line 2: \((\bar p + \bar q + r)\)
\item
  Line 4: \((\bar p + q + r)\)
\item
  Line 6: \((p + \bar q + r)\)
\end{enumerate}

A CNF formula logically equivalent to \(T\):

\[(\bar p + \bar q + r) \cdot (\bar p + q + r) \cdot (p + \bar q + r)\]

Each clause eliminates exactly one false row. The conjunction ensures
all three clauses hold simultaneously, making the formula false
precisely where \(M\) is false and true elsewhere.

\subsubsection{3.1. General Construction}\label{general-construction-2}

Given sentence \(L\) with atomic sentences \(y_1, \dots, y_n\), we
construct an equivalent sentence \(H\) in CNF. The strategy: identify
where \(L\) fails and build clauses that fail at exactly those points.

\begin{longtable}[]{@{}cccccc@{}}
\toprule\noalign{}
Line & \(y_1\) & \(y_2\) & \(\dots\) & \(y_n\) & \(L\) \\
\midrule\noalign{}
\endhead
\bottomrule\noalign{}
\endlastfoot
1 & 1 & 1 & \(\dots\) & 1 & \(L_1\) \\
2 & 1 & 1 & \(\dots\) & 0 & \(L_2\) \\
3 & 1 & 0 & \(\dots\) & 1 & \(L_3\) \\
⋮ & ⋮ & ⋮ & ⋮ & ⋮ & ⋮ \\
\(2^n\) & 0 & 0 & \(\dots\) & 0 & \(L_{2^n}\) \\
\end{longtable}

Case 1:

When \(L\) is true everywhere, it is a tautology. Any tautological
formula in CNF serves as an equivalent, such as \((p + \bar{p})\) or any
logically equivalent tautology.

Case 2:

When \(L\) is false on some lines, for each line \(j\) where \(L = 0\),
construct clause \(\phi_j\):

\[\phi_j = \displaystyle\bigvee_{t=1}^{n} (\pm y_t)\]

where

\[
(\pm y_t) =
\begin{cases}
\bar y_t, & \text{if } y_t = 1 \text{ on line } j, \\
y_t, & \text{if } y_t = 0 \text{ on line } j.
\end{cases}
\]

Each literal is the opposite of its truth value on line \(j\), making
\(\phi_j\) false only on line \(j\).

Let \(j_1, j_2, \dots, j_s\) index all lines where \(L = 0\). Define:

\[H = \phi_{j_1} \cdot \phi_{j_2} \cdot \dots \cdot \phi_{j_s} = \displaystyle\bigwedge_{w=1}^{s} \left[ \bigvee_{t=1}^{n} (\pm y_t) \right]\]

\(H\) is in CNF. Since \(H\) is false exactly where \(L\) is false:

\[L \equiv H\]

Example:

Let \(R\) have atomic sentences \(y_1, y_2, y_3\).

\(y_1\)

\(y_2\)

\(y_3\)

\(R\)

1

1

1

1

1

1

0

0

1

0

1

1

1

0

0

1

0

1

1

0

0

1

0

1

0

0

1

0

0

0

0

1

\(R = 0\) on lines 2, 5, and 7.

\begin{enumerate}
\def\labelenumi{\arabic{enumi}.}
\tightlist
\item
  For line 2: \(g_2 = (\bar y_1 + \bar y_2 + y_3)\)
\item
  For line 5: \(g_5 = (y_1 + \bar y_2 + \bar y_3)\)
\item
  For line 7: \(g_7 = (y_1 + y_2 + \bar y_3)\)
\end{enumerate}

Form the conjunction: \[C = g_2 \cdot g_5 \cdot g_7\]

Hence,

\[\big[(\bar y_1 + \bar y_2 + y_3) \cdot (y_1 + \bar y_2 + \bar y_3) \cdot (y_1 + y_2 + \bar y_3)\big]\]

We conclude that, \[R \equiv C\]

Example 1:

Let

\[I = a \to \bar b\]

First, we rewrite using the definition of implication:

\[I = (a \to \bar b \equiv \bar a + \bar b)\]

Truth Table

\(a\)

\(b\)

\(\bar a\)

\(\bar b\)

\(I\)

1

1

0

0

0

1

0

0

1

1

0

1

1

0

1

0

0

1

1

1

Hence,

\begin{enumerate}
\def\labelenumi{\arabic{enumi}.}
\item
  DNF Construction:

  \begin{enumerate}
  \def\labelenumii{\arabic{enumii}.}
  \tightlist
  \item
    Line 2: \((a \cdot \bar b)\)
  \item
    Line 3: \((\bar a \cdot b)\)
  \item
    Line 4: \((\bar a \cdot \bar b)\)
  \end{enumerate}

  Therefore,

  \[I= (a \cdot \bar b) + (\bar a \cdot b) + (\bar a \cdot \bar b)\]
\item
  CNF Construction:

  \begin{enumerate}
  \def\labelenumii{\arabic{enumii}.}
  \tightlist
  \item
    Line 1: \((\bar a + \bar b)\)
  \end{enumerate}

  Thus,

  \[I= (\bar a + \bar b)\]
\item
  Both forms are logically equivalent:

  \[I \equiv \bar a + \bar b\]
\end{enumerate}

Example 2:

Let

\[X = \neg(p \equiv q)\]

First, we rewrite using the definition of biconditional:

\[p \leftrightarrow q \equiv (p \cdot q) + (\bar{p} \cdot \bar{q})\]

Then, applying negation:

\[
\neg(p \leftrightarrow q) \equiv \neg[(p \cdot q) + (\bar{p} \cdot \bar{q})]
\]

Using De Morgan's law:

\[
X = (\bar{p} + \bar{q}) \cdot (p + q)
\]

Truth Table

\(p\)

\(q\)

\(\bar p\)

\(\bar q\)

\(p \equiv q\)

\(X\)

1

1

0

0

1

0

1

0

0

1

0

1

0

1

1

0

0

1

0

0

1

1

1

0

\begin{enumerate}
\def\labelenumi{\arabic{enumi}.}
\item
  DNF Construction

  \(X = 1\) on lines 2 and 3.

  \begin{enumerate}
  \def\labelenumii{\arabic{enumii}.}
  \tightlist
  \item
    Line 2: \((p \cdot \bar{q})\)
  \item
    Line 3: \((\bar{p} \cdot q)\)
  \end{enumerate}

  Therefore,

  \[
   X = (p \cdot \bar{q}) + (\bar{p} \cdot q)
   \]
\item
  CNF Construction

  \(X = 0\) on lines 1 and 4.

  \begin{enumerate}
  \def\labelenumii{\arabic{enumii}.}
  \tightlist
  \item
    Line 1: \((\bar{p} + \bar{q})\)
  \item
    Line 4: \((p + q)\)
  \end{enumerate}

  Thus,

  \[
   X = (\bar{p} + \bar{q}) \cdot (p + q)
   \]
\item
  Both forms are logically equivalent

  \[
   X \equiv (p \cdot \bar{q}) + (\bar{p} \cdot q) \equiv (\bar{p} + \bar{q}) \cdot (p + q)
   \]
\end{enumerate}
