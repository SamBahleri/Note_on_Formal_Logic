Deductive and Inductive

\subsection{1. Deductive}\label{deductive}

In terms of etymology, the word ``deductive'' comes from the Latin verb
deducere, which means ``to lead down or away.'' This reflects the
process of reasoning from general principles to specific conclusions.
Moreover, in terms of terminology, specifically in logic and philosophy,
deductive reasoning refers to a method of reasoning that moves from
general premises to specific conclusions. We have already seen this in
the Introduction material. In short, deductive reasoning is a form of
reasoning based on necessity. If we list them out, several of the topics
below involve forms of reasoning or conclusions that are necessarily
true.

\begin{enumerate}
\def\labelenumi{\arabic{enumi}.}
\item
  Logic

  As we have seen in the Introduction material, a statement is a
  declaration that contains truth, whether it is true or false.
  Specifically, if the premises are false, then the conclusion will also
  be untrue; and conversely, if the premises are true, then the
  deduction must be valid. In this case, the conclusion is vital.

  Example:

  \$

  \begin{aligned}
   \text{Premise 1:} \ & \text{All humans are living beings} \\
   \text{Premise 2:} \ & \text{Socrates is a human} \\
   \text{Conclusion:} \ & \text{Therefore, Socrates is a living being}
   \end{aligned}

  \$

  Example:

  \$

  \begin{aligned}
   \text{Premise 1:} \ & \text{If } A, \text{ then } B \\
   \text{Premise 2:} \ & A \text{ occurs} \\
   \text{Conclusion:} \ & \text{Therefore, } B \text{ occurs}
   \end{aligned}

  \$
\item
  Definition

  Deductive reasoning can also be observed through definitions.

  Example:

  \begin{enumerate}
  \def\labelenumii{\arabic{enumii}.}
  \item
    Truth is truth.
  \item
    All triangles are flat shapes with three sides.
  \item
    A bachelor is an adult man who is not married.
  \end{enumerate}
\item
  Sylogistic

  As we've seen in the Introduction material, syllogistic reasoning is
  reasoning where the conclusion is necessarily true due to elements
  such as categories (living beings) and definitions.

  Example:

  \$

  \begin{aligned}
   \text{Premise 1:} \ & \text{All humans are living beings} \\
   \text{Premise 2:} \ & \text{Socrates is a human} \\
   \text{Conclusion:} \ & \text{Therefore, Socrates is a living being}
   \end{aligned}

  \$
\end{enumerate}

\subsection{2. Inductive}\label{inductive}

In terms of etymology, the word ``inductive'' comes from the Latin verb
inducere, meaning ``to lead into'' or ``to bring in.'' This reflects the
process of reasoning that moves from specific observations or examples
toward general conclusions or principles. Furthermore, in terms of
terminology, inductive reasoning refers to a method of reasoning in
which generalizations are formed based on repeated observations or
patterns. We have also seen this form of reasoning in the Introduction
material, specifically in the section on informal logic. In short,
inductive reasoning is a type of reasoning based on probability. If we
were to list them, several topics below are forms of reasoning or
conclusions that are probabilistic in nature.

\begin{enumerate}
\def\labelenumi{\arabic{enumi}.}
\item
  Logic

  Although inductive reasoning is frequently associated with informal
  logic or arguments lacking rigid formal structure, in practice, formal
  logical structures, such as modus ponens, can also be used in
  inductive reasoning, depending on the content of the premises. Example
  (Modus Ponens):

  \$

  \begin{aligned}
   \text{Premise 1:} \ & \text{If Paul studies hard, then he will be accepted into a university} \\
   \text{Premise 2:} \ & \text{Paul studies hard} \\
   \text{Conclusion:} \ & \text{Therefore, he will be accepted into a university}
   \end{aligned}

  \$

  Structurally, this argument takes the form of modus ponens, which is a
  deductive form. However, the first premise does not deliver logical
  certainty, but rather a high probability based on past experience or
  generalization. Thus, while the form is deductive, the content is
  inductive, meaning the conclusion is not guaranteed, only probable.
  This shows that inductive reasoning doesn't have to be informal; it
  can take a formal structure, as long as the truth of the premises is
  not absolute.
\item
  Analogy

  One common type of inductive reasoning is reasoning by analogy. In
  this form, one concludes that because two things share some
  similarities, they may also share other characteristics. A famous
  example is the ``watchmaker analogy'' used by William Paley to support
  the argument for the existence of God as the designer of the universe.

  Form of Reasoning:

  \$

  \begin{aligned}
   \text{Premise 1:} \ & \text{A watch, with its complexity and order, shows evidence of being designed by a watchmaker} \\
   \text{Premise 2:} \ & \text{The universe also shows similar complexity and order} \\
   \text{Conclusion:} \ & \text{Therefore, the universe is also likely designed by a creator/designer}
   \end{aligned}

  \$

  This kind of reasoning is clearly inductive, because it is based on
  probability. As stated above, just because two things share some
  features does not necessarily mean they share all features. Such
  reasoning, while it may sound convincing, is often considered a weak
  argument.
\item
  Generalization

  Generalization is a type of inductive reasoning characterized by
  drawing conclusions from only a few instances or observations.

  Example:

  \$

  \begin{aligned}
   \text{Premise 1:} \ & \text{Swan A is white} \\
   \text{Premise 2:} \ & \text{Swan B is white} \\
   \text{Premise 3:} \ & \text{Swan C is white} \\
   \text{Conclusion:} \ & \text{Therefore, all swans are white}
   \end{aligned}

  \$
\item
  Causation

  Although the law of causality feels intuitively persuasive and is
  widely used in everyday life and science, it is actually a form of
  inductive reasoning, for several reasons:

  \begin{enumerate}
  \def\labelenumii{\arabic{enumii})}
  \item
    Causality is based on experience and repeated observation of
    specific patterns.

    The philosopher David Hume is well-known for this view. He reasoned
    that we never directly experience causality; instead, we only
    observe that one event regularly follows another. In other words,
    what we perceive as a causal relationship is merely a constant
    conjunction of events. Therefore, the link we presume between cause
    and effect is just a habit of the mind, not something objectively
    featured of reality.
  \item
    Causality assumes the consistency of nature, or that the future will
    resemble the past.

    For instance, we believe the sun will rise tomorrow because it has
    always risen in the past. But this belief is based purely on past
    experience, not on any inherent truth about the universe itself.
    Hence, causal reasoning is practically a prediction rooted in
    experience, and therefore, it is inductive.
  \end{enumerate}
\end{enumerate}

\subsection{3. Deduction vs/and
Induction}\label{deduction-vsand-induction}

Comprehending this fundamental difference is crucial when producing
formal statements, specifically, as we've studied in the distinction
between formal and informal logic. Regarding what we call ``true'',
formal logic focuses on the form of the argument, while informal logic
focuses on the content of the argument. In this context, deductive
reasoning is typically formal, whereas inductive reasoning is generally
informal. In practice, however, a conclusion should be evaluated not
only based on the form of the argument, but also on the truth of the
premises and the relevance of the content (whether the premises
correspond to reality). Therefore, we cannot rely solely on the rigid
rules we have constructed; we must also take into account our actual
experience of reality. Furthermore, apprehending deduction and induction
is vital for clearly clarifying the boundaries between what is logically
true or false and what is empirically valid or not.

Deductive rules give us logical assurance, since the method is
predetermined, conclusions follow necessarily from the premises.
Induction, on the other hand, offers probability based on experience. It
is also essential to acknowledge that deductive rules are the result of
human consensus, agreements we have made, whereas induction reflects
reality itself. In other words, we comprehend phenomena only to the
extent that we can perceive and interpret them. We assemble patterns to
the degree that they can be applied, even if we do not fully understand
the underlying reality. This marks the epistemological limits of human
knowledge. Essentially, both approaches are equally fundamental, because
it is through both that humans can adapt to reality.
