Limit of Logic

\subsection{1. Contextual Meaning}\label{contextual-meaning}

On one hand, formal logic is precise in systematically expressing
sentences. On the other hand, it tends to flatten nuance into rigid
structures, and as a result, it lacks the richness and flexibility
inherent in natural language.

Consider the following syllogistic reasoning:

\$

\begin{aligned}
\text{Premise 1:} \ & \text{All foxes are carnivores} \\
\text{Premise 2:} \ & \text{King Herod is a fox} \\
\text{Conclusion:} \ & \text{Therefore, King Herod is a carnivore}
\end{aligned}

\$

In syllogistic reasoning, the form is clearly valid, but the content is
not. Here, formal reasoning, especially when translated into symbols,
offers precision in structure, but it does not capture the semantic or
contextual dimension. Specifically, in this case, Jesus metaphorically
referred to King Herod as a ``fox''. The statement was meant as a
figurative critique rather than a literal claim, which formal logic
cannot fully represent. In this case, formal logic operates in only one
dimension, it captures structural validity but fails to account for
meaning, context, or figurative nuance.

Moreover, consider the formula:

\[
\forall x \, (P(x) \rightarrow Q(x))
\]

This is valid in first-order logic, but the interpretation of \(x\) and
the predicates \(P\) and \(Q\) is context-dependent.

\subsection{2. Vagueness}\label{vagueness}

Natural language is full of vague terms, such as ``some,'' ``soon,'' and
``enough.'' Classical logic, however, cannot naturally capture this
vagueness because it requires precision. When we make these terms
precise, we lose part of the expressive subtlety that makes language
rich. For example, if we try to formalize ``enough'' in classical logic,
we might define it as a quantity greater than or equal to a specific
threshold \(x\). But in natural language, ``enough'' is
context-dependent, enough food for one person might not be enough for
others. Forcing precision in this way risks losing the flexibility and
adaptability that give language its nuance.

\subsection{3. Presuppositions}\label{presuppositions}

Consider definite descriptions, although formal logic can represent the
underlying intuitions, it may produce interpretations that differ from
human intentions. This double-layered meaning is difficult to capture
fully within a formal system. Yet at the same time, we also claim that,
for our formal reasoning to remain pure, the logic must not depend on
agents or subjectivity.

\subsection{4. Idiomatic Layer}\label{idiomatic-layer}

Similarly, formal logic is precise but cannot capture the idiomatic
expressions, which often carry non-literal meanings.

For example:

\[\text{“Rain cat and dog”}\]

A formal translation can interpret this syntactically, but it cannot
capture the real-world meaning, that it is raining heavily, without
extra context.

\subsection{5. Naturalness}\label{naturalness}

Natural language is rich, flexible, and expressive, but also messy.
Formal logic, by contrast, is precise but brittle. Yet, the key point is
that natural language, despite being abstract, conveys subtleties and
nuances that formal logic often cannot.

For example, consider the sentence:

\[\text{“Every student who studies hard will likely succeed.”}\]

A formal translation might be:

\[
\forall x \, (\text{Student}(x) \wedge \text{StudiesHard}(x) \rightarrow \text{Succeeds}(x))
\]

This captures the syntactic structure and a strict logical relation.
However, the original sentence conveys probabilistic nuance and
contextual subtleties that the formal statement ignores, showing that
natural language expresses meaning beyond strict logical forms.
