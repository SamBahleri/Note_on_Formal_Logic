Semantic Concepts for FOL

\subsubsection{1. Validity}\label{validity}

\[\text{A formula is valid if it's true in every interpretation}\]

Example: \[\forall x \in \{1,2,3\} \, (P(x) \rightarrow P(x))\]

This is valid because \emph{if 1 has property \(P\), then 1 has property
\(P\)} is always true, regardless of what \(P\) represents.

\subsubsection{2. Satisfiability}\label{satisfiability}

\[\text{A formula is satisfiable if there exists at least one interpretation that makes it true}\]

Example: \[\exists x \in \{1,2,3\} \, (x > 1 \wedge x < 3)\]

This is satisfiable because we can choose \(x = 2\), making \emph{2
\textgreater{} 1 and 2 \textless{} 3} true.

\subsubsection{3. Contradiction}\label{contradiction}

\[\text{A formula is unsatisfiable if no interpretation can make it true}\]

Example: \[\exists x \in \{1,2,3\} \, (x > 3 \wedge x < 1)\]

This is unsatisfiable because no number in our domain can be both
greater than 3 and less than 1.

\subsubsection{4. Entailment}\label{entailment}

\[\Gamma \models \phi \text{ if } \phi \text{ is true whenever all formulas in } \Gamma \text{ are true}\]

Example:
\[\{\forall x \in \{1,2,3\} \, (x < 3), 2 \in \{1,2,3\}\} \models (2 < 3)\]

The premises \emph{all numbers in our set are less than 3} and \emph{2
is in our set} entail \emph{2 is less than 3}.

\subsubsection{5. Non-Entailment}\label{non-entailment}

\[\Gamma \not\models \phi \text{ if there exists an interpretation where } \Gamma \text{ is true but } \phi \text{ is false}\]

Example:
\[\{\exists x \in \{1,2,3\} \, (x > 1)\} \not\models \forall x \in \{1,2,3\} \, (x > 1)\]

\emph{Some number is greater than 1} doesn't entail \emph{all numbers
are greater than 1} because 1 itself isn't greater than 1.

\subsubsection{6. Joint Satisfiability}\label{joint-satisfiability}

\[\text{Formulas are jointly satisfiable if some interpretation makes them all true simultaneously}\]

Example:
\[\{P(1), \neg P(2), \exists x \, P(x)\} \text{ where domain } = \{1,2\}\]

We can make P true for 1, false for 2, and \emph{something has property
\(P\)} true, all consistent.

\subsubsection{7. Logical Equivalence}\label{logical-equivalence}

\[\phi \equiv \psi \text{ if } \phi \text{ and } \psi \text{ have the same truth value in every interpretation}\]

Example:
\[\forall x \in \{1,2,3\} \, \neg(x > 2) \equiv \forall x \in \{1,2,3\} \, (x \leq 2)\]

\emph{No number is greater than 2} is equivalent to \emph{every number
is less than or equal to 2}.

\subsubsection{8. Contingency}\label{contingency}

\[\text{A formula is contingent if it's true in some interpretations and false in others}\]

Example: \[\exists x \in \{1,2,3\} \, (x = 2)\]

This is contingent, true when our domain includes 2, false if we change
the domain to \(\{4,5,6\}\).

\subsubsection{9. Expressibility}\label{expressibility}

\[\text{A property is expressible if there exists a formula that captures exactly that property}\]

Example:
\[\text{Even}(x) := (x = 2) \vee (x = 4) \text{ over domain } \{1,2,3,4\}\]

The property \emph{being even} is expressible because we can write a
formula true exactly for even numbers.

\subsubsection{10. Definability}\label{definability}

\[\text{A relation is definable if it can be expressed using existing predicates and logical operators}\]

Example:
\[\text{Between}(x,y,z) := (x < y < z) \vee (z < y < x) \text{ over } \{1,2,3,4,5\}\]

Using the less-than relation, we can define \emph{\(y\) is between \(x\)
and \(z\)} without introducing new predicates.

\subsubsection{11. Implicit Definability}\label{implicit-definability}

\[\text{A relation is uniquely determined by certain conditions without explicit construction}\]

Example:
\[\forall x,y \, (\text{Succ}(x,y) \leftrightarrow (x < y \wedge \neg\exists z \, (x < z < y)))\]

The successor relation in \(\{1,2,3\}\) is implicitly defined as the
unique immediate-next relation.

\subsubsection{12. Finite Expressibility}\label{finite-expressibility}

\[\text{Properties expressible only over finite domains of specific size}\]

Example:
\[\exists x \exists y \exists z \, (x \neq y \neq z \wedge \forall w \, (w = x \vee w = y \vee w = z))\]

\emph{Having exactly 3 elements} can be expressed for domain
\(\{1,2,3\}\) but doesn't work for arbitrary domains.

\subsubsection{13. Cardinality
Constraints}\label{cardinality-constraints}

\[\text{Expressing numerical constraints on elements satisfying properties}\]

Example:
\[\exists x \exists y \, (x \neq y \wedge \text{Prime}(x) \wedge \text{Prime}(y)) \text{ over } \{2,3,4,5,6\}\]

This expresses \emph{at least two distinct elements are prime} using
existential quantification with inequality.
