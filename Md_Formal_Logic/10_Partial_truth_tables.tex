Partial truth tables

The idea of \emph{partial truth tables} is that we don't always need to
list every possible input for a proposition in order to determine
whether it is a tautology, contradiction, or something else. Instead, we
can strategically test only the cases that matter. One common method is
Reductio ad Absurdum (\(raa\)), in which we assume the opposite of what
we want to prove and then show that this assumption leads to a
contradiction.

\subsubsection{1. Example 1: Tautology}\label{example-1-tautology}

Consider the formula:

\[ (a \land (a \rightarrow b)) \rightarrow b \]

To test whether this is a tautology, we assume the formula is false. An
implication \(x \rightarrow y\) is false only when \(x\) is true and
\(y\) is false.

Firstly, we assume the implication is false:

\(a\)

\(\land\)

\((a \rightarrow b)\)

\(\rightarrow\)

\(b\)

0

Second, since the implication is false, we can immediately set
\(b = \text{False}\). Moreover, the implication has two parts:

\[ (a \land (a \rightarrow b)) \quad \text{and} \quad b \]

By definition of implication, for the whole formula to be false the
consequent \(b\) must be false and the antecedent
\((a \land (a \rightarrow b))\) must be true. Therefore, we place
\emph{True} under the Conjunction (\(\land\)) row.

\(a\)

\(\land\)

\((a \rightarrow b)\)

\(\rightarrow\)

\(b\)

1

0

0

Third, a conjunction is true only when both inputs are true. Thus, \(a\)
must be true and \((a \rightarrow b)\) must also be true. However, since
we already assumed \(b\) is false, the expression \((a \rightarrow b)\)
becomes a contradiction.

\(a\)

\(\land\)

\((a \rightarrow b)\)

\(\rightarrow\)

\(b\)

1

1

1

0

0

Finally, as we can see, \((a \rightarrow b)\) would only be true when
\(b\) is true. Yet from the beginning we assumed \(b\) is false. This
creates a contradiction, so the statement cannot be false under any
assignment of truth values. Therefore, the formula is always true, which
means it is a tautology.

\subsubsection{2. Example 2:
Contradiction}\label{example-2-contradiction}

To show a formula is a contradiction, we try to make the formula true by
assigning truth values. If no assignment makes it true, it is a
contradiction.

Consider this formula:

\[ a \land \neg a \]

First, suppose \(a = 1\). Then \(\neg a = 0\).

\(a\)

\(\neg a\)

\(a \land \neg a\)

1

0

0

Second, suppose \(a = 0\). Then \(\neg a = 1\).

\(a\)

\(\neg a\)

\(a \land \neg a\)

0

1

0

Ultimately, in both cases, the output of \(a \land \neg a\) is
\emph{False}. Therefore, the formula \(a \land \neg a\) is never true
and is a contradiction.

\subsubsection{3. Example 3: Equivalence}\label{example-3-equivalence}

To show two formulas are equivalent, we check whether they always have
the same truth value under every possible assignment of inputs.

Consider this equivalence:

\[ a \rightarrow b \;\;\equiv\;\; \neg a \lor b \]

Suppose \(a = 1\) and \(b = 1\).

\(a\)

\(b\)

\(a \rightarrow b\)

\(\neg a\)

\(\neg a \lor b\)

1

1

1

0

1

Also, suppose \(a = 1\) and \(b = 0\).

\(a\)

\(b\)

\(a \rightarrow b\)

\(\neg a\)

\(\neg a \lor b\)

1

0

0

0

0

Suppose \(a = 0\) and \(b = 1\).

\(a\)

\(b\)

\(a \rightarrow b\)

\(\neg a\)

\(\neg a \lor b\)

0

1

1

1

1

Finally, suppose \(a = 0\) and \(b = 0\).

\(a\)

\(b\)

\(a \rightarrow b\)

\(\neg a\)

\(\neg a \lor b\)

0

0

1

1

1

As a result, in every possible case, the columns for \(a \rightarrow b\)
and \(\neg a \lor b\) match exactly. Therefore, the two formulas are
logically equivalent.

Full truth table:

\(a\)

\(b\)

\(\neg a\)

\(a \rightarrow b\)

\(\neg a \lor b\)

1

1

0

1

1

1

0

0

0

0

0

1

1

1

1

0

0

1

1

1

Note that to show that a formula is an entailment, we must present a
complete truth table. We cannot just assume a few simple inputs, as is
sometimes done with tautologies or contradictions.

\subsubsection{4. Example 4: Validity}\label{example-4-validity}

To check for validity using a partial truth table, we assume the
conclusion is false and see if all premises can still be true. If it is
impossible, the argument is valid.

Consider the argument:

\begin{enumerate}
\def\labelenumi{\arabic{enumi}.}
\tightlist
\item
  \(a \rightarrow b\)\\
\item
  \(b \rightarrow c\)
\end{enumerate}

We want to prove that \(a \rightarrow c\).

First, assume the conclusion is false. To make \(a \rightarrow c\)
false, recall that a conditional statement \(x \rightarrow y\) is false
only when \(x = 1\) and \(y = 0\).\\
Thus, we set:

\[
a = 1, \quad c = 0
\]

Next, we try to keep the premises true. For \(a \rightarrow b\) to be
true with \(a = 1\), we must set:

\[
b = 1
\]

\(a\)

\(b\)

\(c\)

\(a \rightarrow b\)

\(b \rightarrow c\)

\(a \rightarrow c\)

1

1

0

1

0

0

\begin{enumerate}
\def\labelenumi{\arabic{enumi}.}
\tightlist
\item
  \(a \rightarrow b\) is \emph{True}.\\
\item
  \(b \rightarrow c\) is \emph{False}.\\
\item
  \(a \rightarrow c\) is \emph{False} (as intended).
\end{enumerate}

Recall that for validity we require the premises
\[a \rightarrow b \quad \text{and} \quad b \rightarrow c\] to be
\emph{True}.

However, when we force the conclusion to be false (\(a=1, c=0\)), one of
the premises (\(b \rightarrow c\)) inevitably becomes false as well.
Thus, it is impossible to make all the premises true while keeping the
conclusion false. Therefore, the argument is valid.
